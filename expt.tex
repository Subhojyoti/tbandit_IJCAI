\begin{figure}
    \centering
    \begin{tabular}{cc}
    \subfigure[0.32\textwidth][Experiment $1$: Experiment with Arithmetic Progression]
    {
    		\pgfplotsset{
		tick label style={font=\Huge},
		label style={font=\Huge},
		legend style={font=\Large},
		}
        \begin{tikzpicture}[scale=0.4]
      	\begin{axis}[
		xlabel={timestep},
		ylabel={Error Percentage},
		grid=major,
        %clip mode=individual,grid,grid style={gray!30},
        clip=true,
        %clip mode=individual,grid,grid style={gray!30},
  		legend style={at={(0.5,1.2)},anchor=north, legend columns=3} ]
      	% UCB
		\addplot table{results/budgetTestAP1/APT1_comp_subsampled.txt};
		\addplot table{results/budgetTestAP1/UA1_comp_subsampled.txt};
		\addplot table{results/budgetTestAP1/UCBEM1_comp_subsampled.txt};
		\addplot table{results/budgetTestAP1/UCBEMV1_comp_subsampled.txt};
		\addplot table{results/budgetTestAP1/AugUCB1_comp_subsampled.txt};
		%\addplot table{results/budgetTestAP/UA1.txt};
      	\legend{APT,Unif Alloc,UCBE($1$,UCBEV($1$),AugUCB}
      	\end{axis}
      	\end{tikzpicture}
  		\label{Fig:budgetExpt1}
    }
    %\\
    &
    \subfigure[0.32\textwidth][Experiment $2$: Experiment with $4$ Group Setting ]
    {
    		\pgfplotsset{
		tick label style={font=\Huge},
		label style={font=\Huge},
		legend style={font=\Large},
		}
        \begin{tikzpicture}[scale=0.4]
        \begin{axis}[
		xlabel={timestep},
		ylabel={Error Percentage},
        %clip mode=individual,grid,grid style={gray!30},
       	grid=major,
       	clip=true,
  		legend style={at={(0.5,1.2)},anchor=north, legend columns=3} ]
      	% UCB
		\addplot table{results/budgetTestGR1/APT1_comp_subsampled.txt};
		\addplot table{results/budgetTestGR1/UA1_comp_subsampled.txt};
		\addplot table{results/budgetTestGR1/UCBEM1_comp_subsampled.txt};
		\addplot table{results/budgetTestGR1/UCBEMV1_comp_subsampled.txt};
		\addplot table{results/budgetTestGR1/AugUCB1_comp_subsampled.txt};
		%\addplot table{results/budgetTestGP/UA1.txt};
        \legend{APT,Unif Alloc,UCBE($1$),UCBEV1$(1)$,AugUCB}
      	\end{axis}
      	\end{tikzpicture}
   		\label{Fig:budgetExpt2} 
    }
%    &
%    \subfigure[Experiment $3$: Threshold Bandit with $3$ Group Setting ]
%    {
%    		\pgfplotsset{
%		tick label style={font=\Large},
%		label style={font=\Large},
%		%legend style={font=\footnotesize}
%		}
%        \begin{tikzpicture}[scale=0.5]
%        \begin{axis}[
%		xlabel={timestep},
%		ylabel={Error Percentage},
%        %clip mode=individual,grid,grid style={gray!30},
%		grid=major,
%		clip=true,
%  		legend style={at={(0.5,-0.2)},anchor=north, legend columns=3} ]
%        % UCB
%		\addplot table{results/budgetTestGR/APT1.txt};
%		\addplot table{results/budgetTestGR/AugUCB1.txt};
%		\addplot table{results/budgetTestGR/UCBE_1_41.txt};
%		\addplot table{results/budgetTestGR/UCBE_11.txt};
%		\addplot table{results/budgetTestGR/UCBE_2561.txt};
%		\addplot table{results/budgetTestGR/UA1.txt};
%        \legend{APT,AugUCBE,UCBE($0.25$),UCBE(1),UCBE(256),Unif Alloc}
%      	\end{axis}
%      	\label{Fig:budgetExpt3}
%        \end{tikzpicture}
%    }
    \end{tabular}
    \caption{Experiments with thresholding bandit}
    \label{fig:budgetExpt}
\end{figure}


	In this section we compare the empirical performance of AugUCB against APT, Uniform Allocation, UCBE and UCBEV algorithm. The threshold $\tau$ is set at $0.5$ for all experiments. Each algorithm is run independently $500$ times for $15000$ timesteps and the output set of arms suggested by the algorithms at every timestep is recorded. The output is considered erroneous if the correct set of arms is not $i=\lbrace 6,7,8,9,10 \rbrace$ (true for all the experiments). The error percentage over $500$ runs is plotted against $15000$ timesteps. For the uniform allocation algorithm, for each $t=1,2,..,T$ the arms are sampled uniformly. For UCBE algorithm  (\cite{audibert2009exploration}) which was built for single best arm identification, we modify it according to \cite{locatelli2016optimal} to suit the goal of finding arms above the threshold $\tau$. So the exploration parameter $a$ in UCBE is set to $a_{i}=\frac{T-K}{H_1}$. Then for each timestep $t=1,2,..,T$ we pull the arm that maximizes $\lbrace |\hat{r}_{i} -\tau|-\sqrt{\frac{a_{i}}{n_{i}}} \rbrace$, where $n_{i}$ is the number of times the arm $i$ is pulled till $t-1$ timestep. Also, APT is run with $\epsilon=0$, which denotes the precision with which the algorithm suggests the best set of arms. So when $\epsilon$ is  set to $0$ APT has to suggest the exact set of arms above the threshold. For AugUCB we take $\psi=\frac{T}{\log K}$, $\rho=\frac{1}{8}$ and $\rho_v=\frac{2}{3}$ as in Corollary \ref{Result:Corollary:1}.
	
	The first experiment is conducted on a testbed of $100$ arms involving Gaussian reward distribution with expected rewards of the arms $r_{1:4}=0.2+(0:3)*0.05$, $r_{5}=0.45$, $r_{6}=0.55$, $r_{7:10}=0.65+(0:3)*0.05$ and $r_{11:100}$=0.4. The means of first $10$ arms are set as arithmetic progression. Variance is set as $\sigma_{i=1:6 \text{ and } 11:100}^{2}=0.5$ and $\sigma_{i=7:10}=0.6$. The means in the testbed are chosen in such a way that any algorithm has to spend a significant amount of budget to explore all the arms and variance is chosen in such a way that the arms above $\tau$ have high variance. In this experiment we see that AugUCB performs better than all the other algorithms mentioned. Only UCBE($1$) and UCBEV($1$) beats AugUCB and that is because they have access to the exact problem complexity. The result is shown in Figure \ref{Fig:budgetExpt1}.
	
	The second experiment is conducted on a testbed of $100$ arms with the means divided into $4$ groups. Again the testbed involves Gaussian reward distributions with expected rewards of the arms as $r_{1:3}=0.1$, $r_{4:7}=\lbrace 0.35,0.45,0.55,0.65\rbrace$, $r_{8:10}=0.9$ and  $r_{11:100}=0.4$. Also $\sigma_{i=1:7 \text{ and } 11:100}^{2}=0.5$ and $\sigma^{2}_{i=8:10}=0.6$. AugUCB, APT, Uniform Allocation, UCBE and UCBEV with the same settings as experiment $1$ are run on this testbed. The result is shown in Figure \ref{Fig:budgetExpt2}. Here, also we see that AugUCB beats APT.  
	
%	The second experiment is conducted on a testbed of $10$ arms with the means set as Geometric Progression. The testbed involves Bernoulli reward distribution with expected rewards of the arms as $r_{1:4}=0.4-(0.2)^{1:4}$, $r_{5}=0.45$, $r_{6}=0.55$ and $r_{7:10}=0.6+(0.2)^{5-(1:4)}$. AugUCB, APT, Uniform Allocation and UCBE with the same settings  as experiment $1$ are run on this testbed. The result is shown in Figure \ref{Fig:budgetExpt2}. Here, we see that AugUCB beats APT with only UCBE($1$) performing at par with AugUCB. 

%\begin{figure}
%\centering
  %\begin{tabular}{c}
  %&
  %%%%%%Expt4
  %\begin{subfigure}{0.45\textwidth}
 %\tabl{c}{\scalebox{0.7}{
% \begin{tikzpicture}
%      \begin{axis}[
%	xlabel={timestep},
%	ylabel={Error Percentage},
%       clip mode=individual,grid,grid style={gray!30},
%  legend style={at={(0.5,-0.2)},anchor=north, legend columns=4} ]
%      % UCB
%\addplot table[x index=0,y index=1,col sep=tab,each nth point={10}] {results/budgetTestGR/APT.txt};
%\addplot table[x index=0,y index=1,col sep=tab,each nth point={10}] {results/budgetTestGR/AugUCB.txt};
%\addplot table[x index=0,y index=1,col sep=tab,each nth point={10}] {results/budgetTestGR/UA.txt};
%\addplot table[x index=0,y index=1,col sep=tab,each nth point={10}] {results/budgetTestGR/UCBE_1.txt};
%\addplot table[x index=0,y index=1,col sep=tab,each nth point={10}] {results/budgetTestGR/UCBE_256.txt};
%\addplot table[x index=0,y index=1,col sep=tab,each nth point={10}] {results/budgetTestGR/UCBE_1_4.txt};
%      \legend{APT,AugUCBE,Unif Alloc,UCBE(1),UCBE(256),UCBE($\frac{1}{4}$)}
%      %\legend{ClusUCB (NC, p=1),ClusUCB (C, p=4),ClusUCB(C, p=10) ,MOSS, ClusUCB(C, p=5, NAE), ClusUCB(C, p=10, NAE)}
%      %\legend{ClusUCB(1,A),ClusUCB(4,B),ClusUCB(10,B), MOSS,ClusUCB(5,S), ClusUCB(10,A)}
%      \end{axis}
%      \end{tikzpicture}
%      %}\\}
%			\caption{Experiment $3$: Threshold Bandit with $3$ Group Setting }
%  \label{Fig:budgetExpt3}
  %\end{subfigure}
  %\end{tabular}
%\end{figure}

%	The third experiment is conducted on a testbed of $10$ arms with the means divided into $3$ groups. Again the testbed involves Bernoulli reward distribution with expected rewards of the arms as $r_{1:3}=0.1$, $r_{4:7}=\lbrace 0.35,0.45,0.55,0.65\rbrace$ and $r_{8:10}=0.9$. AugUCB, APT, Uniform Allocation and UCBE with the same settings as experiment $1$ are run on this testbed. The result is shown in Figure \ref{Fig:budgetExpt3}. Here, also we see that AugUCB beats APT.  
