In this paper we propose the Algorithm AugUCB which is an anytime action elimination algorithm suited for the TBP problem. It combines the approach of UCB-Improved, CCB (\cite{liu2016modification}) and APT algorithm. Our algorithm is also a variance-aware algorithm which takes into account the empirical variance of the arms. We also address an open problem raised in \cite{auer2010ucb} of coming up with an algorithm that can eliminate arms based on variance. Both CSAR and APT are not variance-aware algorithms. Theoretically our result is more closer to CSAR and is weaker than APT. But empirically we show that for a large action set when the variance of the arms lying above $\tau$ are high, our algorithm performs better than all other algorithms, except the algorithm UCBEV which has access to the underlying problem complexity and also is a variance aware algorithm. Irrespective of this case AugUCB also employs elimination of arms based on mean estimation only and is the first such algorithm which uses elimination by both mean and variance estimation simultaneously. AugUCB requires three input parameters and the exact choices for these parameters are derived in Theorem \ref{Result:Theorem:1}. Also, unlike SAR or CSAR, AugUCB does not have explicit accept or reject set rather the arm elimination conditions simply removes arm(s) if it is sufficiently sure that the mean of the arms are very high or very low about the threshold based on mean and variance estimation thereby re-allocating the remaining budget among the surviving arms. This although is a tactic similar to SAR or CSAR, but here at any round, an arbitrary number of arms can be accepted or rejected thereby improving upon SAR and CSAR which accepts/rejects one arm in every round. At the end of the budget $T$ the algorithm outputs all the arms whose $\hat{r}_{i}$ is above the threshold $\tau$ thereby making this an anytime algorithm whereby we need not finish every round. 

\begin{table}
\caption{Expected Loss for different bandit algorithms}
\label{tab:regret-bds}
\begin{center}
\begin{tabular}{|p{1.3cm}|p{6.33cm}|}
\toprule
Algorithm  & Upper Bound on Expected Loss \\
\midrule
APT         &$\exp(-\frac{T}{64H_1}+2\log((\log(T)+1)K))$ \\\midrule
CSAR		&$K^2\exp(-\frac{T-K}{72\log(K)H_2})$ \\\midrule
AugUCB      &
$\exp\left( -\frac{T\log ( 2K\log K)}{ 16 H_2 K\log K} + \log\left(K\left(\log_2\frac{T}{e}\right)\right)\right)$\newline
 $+$\newline
 $ \exp\left(- \frac{T\log ( \frac{3}{4} K\log K)}{64H_{2}^{\sigma} K(\log K)}
 + 2\log\left(K\left(\log_2\frac{T}{e}\right)\right) \right) $
%$\exp\big( -\frac{T\log (2 K\sqrt{\log K})}{2 H_2 K (\log K)^{3/2}} + \log(K(\log_2\frac{T}{e}+1))\big)
%+ \exp \big(- \frac{5T\log ( K\sqrt{\log K})}{H_{2}^{\sigma} K(\log K)^{3/2}} $
%\newline$ + \log(K\big(\log_2\frac{T}{e}+1))\big)$ 
\\\bottomrule
\end{tabular}
\end{center}
\end{table}

  