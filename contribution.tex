In this paper we propose the Algorithm AugUCB which is an action elimination algorithm suited for the TBP problem. It combines the approach of UCB-Improved, CCB (\cite{liu2016modification}) and APT algorithm. Our algorithm is a variance-aware algorithm which takes into account the empirical variance of the arms. We also address an open problem raised in \cite{auer2010ucb} of coming up with an algorithm that can eliminate arms based on variance. Both CSAR and APT are not variance-aware algorithms. The expected loss of various algorithms is shown in Table \ref{tab:regret-bds}. The terms $H_1, H_2, H_{CSAR,2}, H_1^{\sigma}$ and $H_2^{\sigma}$ signifies problem complexity and are defined in section \ref{results}. Theoretically, we can compare the first term (containing $H_2$) of our expected loss and see that for all $K\geq 4$, $ H_2 \log(\frac{3}{16} K\log K) > (\log K)H_{CSAR,2}\geq H_1 $ and hence our result is weaker than CSAR and APT. The term containing $H_{2}^{\sigma}$ is comparable to the similar terms (containing $H_1^{\sigma}$) for the error probability of Gap-EV(\cite{gabillon2011multi} or UGapE-V(\cite{gabillon2012best}) algorithm which we modify to perform in the TBP problem. The error probability of Gap-EV for single bandit multi-armed case is given by $6TK\exp(-\frac{1}{512}\frac{T-2K}{H_1^{\sigma}})$ where $\log(\frac{3}{16} K\log K) H_2^{\sigma} > H_1^{\sigma}$ and hence our algorithm is weaker with respect to Gap-EV for single  multi-armed bandit scenario. But Gap-EV algorithm needs the complexity factor $H_1^{\sigma}$ as input for optimal performance (which is not a realistic scenario) whereas AugUCB requires no such complexity factor as input. 

\begin{table}[h]
\caption{Expected Loss for different bandit algorithms}
\label{tab:regret-bds}
\begin{center}
\begin{tabular}{|p{1.3cm}|p{6.33cm}|}
\toprule
Algorithm  & Upper Bound on Expected Loss \\
\midrule
APT         &$\exp(-\frac{T}{64 H_1}+2\log((\log(T)+1)K))$ \\\midrule
CSAR		&$K^2\exp(-\frac{T-K}{72\log(K)H_{CSAR,2}})$ \\\midrule
%Gap-EV		&$6TK\exp(-\frac{1}{512}\frac{T-2K}{H_1^{\sigma}})$ \\\midrule
AugUCB      &
$\exp\left( -\frac{T}{ 64 H_2 a} + \log\left(K\left(\log_2\frac{T}{e}+1\right)\right)\right)+$\newline
 $ \exp\left(- \frac{T}{4096 H_{2}^{\sigma} a}
 + \log\left(2K\left(\log_2\frac{T}{e}+1\right)\right) \right) $\newline
 where $a=\log(\frac{3}{16} K\log K)$
\\\bottomrule
\end{tabular}
\end{center}
\end{table}
Empirically we show that for a large number of arms when the variance of the arms lying above $\tau$ are high, our algorithm performs better than all other algorithms, except the algorithm UCBEV (modified from Gap-EV for TBP) which has access to the underlying problem complexity and also is a variance-aware algorithm. Irrespective of this case AugUCB also employs elimination of arms based on mean estimation only and is the first such algorithm which uses elimination by both mean and variance estimation simultaneously. AugUCB requires three input parameters and the exact choices for these parameters are derived in Theorem \ref{Result:Theorem:1}. Also, unlike SAR or CSAR, AugUCB does not have explicit accept or reject set rather the arm elimination conditions simply removes arm(s) if it is sufficiently sure that the mean of the arms are very high or very low about the threshold based on mean and variance estimation thereby re-allocating the remaining budget among the surviving arms. This although is a tactic similar to SAR or CSAR, but here at any round, an arbitrary number of arms can be accepted or rejected thereby improving upon SAR and CSAR which accepts/rejects one arm in every round. Also their round lengths are non-adaptive and they pull all the arms equal number of times in each round. 
%At every timestep AugUCB pulls the arm that minimizes thereby making this an anytime algorithm whereby we need not finish every round. 
The remainder of the paper is organized as follows. In section \ref{algorithm} we present our AugUCB algorithm. 
Section \ref{results} contains our main theorem on expected loss, while section \ref{expt} contains the numerical experiments. We finally draw our conclusions in section \ref{conclusion}.
%in section \ref{notation} we introduce the notations and the