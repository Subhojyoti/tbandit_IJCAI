%The algorithm is presented below:-

%%%%%%%%%%%%%%%% alg-custom-block %%%%%%%%%%%%
\algblock{ArmElim}{EndArmElim}
\algnewcommand\algorithmicArmElim{\textbf{\em Arm Elimination by Mean Estimation}}
 \algnewcommand\algorithmicendArmElim{}
\algrenewtext{ArmElim}[1]{\algorithmicArmElim\ #1}
\algrenewtext{EndArmElim}{\algorithmicendArmElim}
\algtext*{EndArmElim}

\algblock{ArmElimV}{EndArmElimV}
\algnewcommand\algorithmicArmElimV{\textbf{\em Arm Elimination by Mean and Variance Estimation}}
 \algnewcommand\algorithmicendArmElimV{}
\algrenewtext{ArmElimV}[1]{\algorithmicArmElimV\ #1}
\algrenewtext{EndArmElimV}{\algorithmicendArmElimV}
\algtext*{EndArmElimV}

\algblock{ResetParam}{EndResetParam}
\algnewcommand\algorithmicResetParam{\textbf{\em Reset Parameters}}
 \algnewcommand\algorithmicendResetParam{}
\algrenewtext{ResetParam}[1]{\algorithmicResetParam\ #1}
\algrenewtext{EndResetParam}{\algorithmicendResetParam}
\algtext*{EndResetParam}

%%%%%%%%%%%%%%%%%%%%%%%%%%%%%%%%%%%%%%%%%%%%%%%%%%%%%%%%%%%%%%%%%%%%%%%%
%Old Algorithm
%%%%%%%%%%%%%%%%%%%%%%%%%%%%%%%%%%%%%%%%%%%%%%%%%%%%%%%%%%%%%%%%%%%%%%%%

%\begin{algorithm}[th!]
%\caption{AugmentedUCB}
%\label{alg:augucb}
%\begin{algorithmic}
%\State {\bf Input:} Time horizon $T$, exploration parameters $\rho_{\mu}$, $\rho_v$ and $\psi$, threshold $\tau$.
%\State {\bf Initialization:} Set $B_{0}:=A$, $M=\left\lfloor \frac{1}{2}\log_{2} \frac{T}{e}\right\rfloor $, $m:=0$, $\epsilon_{0}:=1$, $\ell_{0}=\left\lceil \frac{2\psi\log( T\epsilon_{0}^{2})}{\epsilon_{0}} \right\rceil$ and $N_{0}=K\ell_{0} $.
%\State Pull each arm once
%\State \For{$t=K+1,..,T$}
%\State Pull arm $i\in\argmin_{j\in B_{m}}\bigg\lbrace |\hat{r}_{j} - \tau | - 2s_{j}\bigg\rbrace$
%\State $t:=t+1$ 
%\ArmElim
%\State For each arm $i \in B_{m}$, remove arm ${i}$ from $B_{m}$ if
%\begin{align*}
%\hat{r}_{i} + c_i  < \tau - c_i \mbox{ or } \hat{r}_{i} - c_i  > \tau + c_i \\
%\text{where $c_i=\sqrt{\frac{\rho_{\mu}\psi\log{( T\epsilon_{m}^{2})}}{2 n_{i}}}$}
%\end{align*}
%\EndArmElim
%\ArmElimV
%\State For each arm $i \in B_{m}$, remove arm ${i}$ from $B_{m}$ if
%\begin{align*}
%\hat{r}_{i} + s_i  < \tau - s_i \mbox{ or } \hat{r}_{i} - s_i  > \tau + s_i \\
%\text{where $s_i=\sqrt{\frac{\rho_v\psi\hat{V}_{i}\log{( T\epsilon_{m}^{2})}}{4 n_{i}} + \frac{\rho_v\psi \log{(T\epsilon_{m}^{2})}}{4 n_{i}}}$}
%\end{align*}
%\EndArmElimV
%\State \If{$t\geq N_{m}$ and $m \leq M$}
%\ResetParam
%\State $\epsilon_{m+1}:=\frac{\epsilon_{m}}{2}$
%\State $B_{m+1} := B_{m}$
%\State $\ell_{m+1}:=\left\lceil \frac{2\psi\log( T\epsilon_{m+1}^{2})}{\epsilon_{m+1}} \right\rceil$
%\State $N_{m+1} := t + |B_{m+1}|\ell_{m+1}$
%\State $m := m+1$
%\EndResetParam
%\EndIf
%\EndFor
%\State Output $\hat{S}_{\tau}=\lbrace i: \hat{r}_{i}\geq \tau \rbrace$.
%\end{algorithmic}
%\end{algorithm}

%%%%%%%%%%%%%%%%%%%%%%%%%%%%%%%%%%%%%%%%%%%%%%%%%%%%%%%%%%%%%%%%%%%%%%%%%%%%%%%%%%%%%

%%%%%%%%%%%%%%%%%%%%%%%%%%%%%%%%%%%%%%%%%%%%%%%%%%%%%%%%%%%%%%%%%%%%%%%%%%%%%%%%%
%New 	Algorithm
%%%%%%%%%%%%%%%%%%%%%%%%%%%%%%%%%%%%%%%%%%%%%%%%%%%%%%%%%%%%%%%%%%%%%%%%%%%%%%%%%


%%%%%%%%%%%%%%%%%%%%%%
%Notations moved here
%%%%%%%%%%%%%%%%%%%%%%
\label{notation}
\textbf{Notations and assumptions:} $\mathcal{A}$ denotes the set of arms, and $|\mathcal{A}|=K$ is the number of arms in $\mathcal{A}$. 
%Arms generic arm is indexed by $i,j\in\mathcal{A}$. 
For arm $i\in\mathcal{A}$, we use $r_{i}$ to denote the true mean of the distribution from which the rewards are sampled, while $\hat{r}_{i}(t)$ denotes the estimated mean at time $t$. Formally, using $n_i(t)$ to denote the number of times arm $i$ has been pulled until time $t$, we have $\hat{r}_{i}(t)=\frac{1}{n_{i}(t)}\sum_{z=1}^{n_i(t)} X_{i,z}$, where $X_{i,z}$ is the reward sample received when arm $i$ is pulled for the $z$-th time. %
Similarly, we use $\sigma_{i}^{2}$ to denote the true variance of the reward distribution corresponding to arm $i$, while $\hat{v}_{i}(t)$ is the estimated variance, i.e., $\hat{v}_{i}(t)=\frac{1}{n_i(t)}\sum_{z=1}^{n_{i}(t)}(X_{i,z}-\hat{r}_{i})^{2}$. Whenever there is no ambiguity about the underlaying  time index $t$, for simplicity we neglect $t$ from the notations and simply use  $\hat{r}_i, \hat{v}_i,$ and $n_i, $ to denote the respective quantities.  Let  $\Delta_{i}=|\tau-r_{i}|$ denote the distance of the true mean from the threshold $\tau$.



%The average estimated payoff for any arm is denoted by  whereas the true mean of the distribution  is denoted by . The optimal arm is denoted by $*$. The '*' superscript is used to denote anything related to optimal arm. 

 % $n_{i}$ denotes the number of times the arm $i$ has been pulled. $\psi $ denotes the exploration regulatory factor and $\rho_\mu ,\rho_v$ as arm elimination parameters. $\hat{V}_{i}=\frac{1}{n_i}\sum_{t=1}^{n_{i}}(x_{i,t}-r_{i})^{2}$ denotes the empirical variance and $x_{i,t}$ is the reward obtained at timestep t for arm $i$. Also   denotes the true variance of the arm $i$. 
 
Finally, we assume that all the reward distributions 
%from which rewards are sampled are identical and independent 
are $1$-sub-Gaussian (note that,  $1$-sub-Gaussian includes Gaussian distributions with variance less than $1$, distributions supported on an interval of length less than 2, etc). Further, the rewards are assumed to take values in the interval $[0,1]$.
%. In our case, we  assume that all rewards are .

%Also we define $\Delta_{i}=r^{*} - r_{i}$ and $\hat{\Delta}_{i}=\hat{r}^{*} - \hat{r}_{i}$. In all cases $\min_{i\in A}{\Delta_{i}}$ is denoted by $\Delta$.
%and the optimal arm is denoted by $*$. The '*' superscript is used to denote anything related to optimal arm
%\paragraph*{}It is assumed that the distribution from which rewards are sampled are identical and independent sub-Gaussian distributions. Throughout the paper, we assume that the distributions $v_{i}$ are sub-Gaussian that is $\int e^{\lambda(x - r)} v_{i} (dx) ≤ e^{\lambda /2}, \forall \lambda \in \mathbb{R}$. Note that these include Gaussian distributions with variance less than 1 and distributions supported on an interval of length less than 2. All the experiments are also conducted with sub-Gaussians having variance as 1. Together with a Chernoff-Hoeffding bound, the sub-Gaussian assumption implies the following concentration inequality, valid for any
%$u > 0$,
%\newline
%\hspace*{8em}$\mathbb{P}\lbrace \hat{r}_{i} - r^{*} > u\rbrace \leq exp(-\dfrac{su^{2}}{2}) $
%\newline
%where s is the number of pulls of $a_{i}$. T is the horizon over which this entire algorithm runs.  $A^{'}$ at any round $m$ denotes the arms still not eliminated.
%\paragraph*{}The paper is organized as follows. We first present the algorithm in section 6. We then provide the proofs of Phase1 which includes regret-bound calculation and arm deletion conditions in section 7. In section 8, we provide proofs for Phase2 along with early stopping conditions. Section 9 deals with regret bound and then we provide error probability and error bounds in section 10. Experimental results are provided in section 11, and we conclude in section 12.
%%%%%%%%%%%%%%%%%%%%%%
\textbf{The Algorithm:} The Augmented-UCB (AugUCB) algorithm is presented in Algorithm~\ref{alg:augucb}.
AugUCB is essentially based on the arm elimination method of the UCB-Improved \cite{auer2010ucb}, but adapted to the thresholding bandit setting proposed in \cite{locatelli2016optimal}. However, unlike the UCB improved (which is based on mean estimation) our algorithm employs \emph{variance estimates} (as in \cite{audibert2009exploration}) for arm elimination; to the best of our knowledge this is the first variance-aware  algorithm for the thresholding bandit problem. Further, we allow for arm-elimination at each time-step, which is in contrast to the earlier work (e.g., \cite{auer2010ucb,chen2014combinatorial}) where the arm elimination task is deferred to the end of the respective exploration rounds. The details are presented below.

% In algorithm \ref{alg:augucb}, hence referred to as AugUCB, we have two exploration parameters, $\rho_{\mu}$ and $\rho_v$ which are the arm elimination parameters. $\psi_{m}$ is the exploration regulatory factor. 
%The main approach is based on the UCB-Improved algorithm with modifications suited for the thresholding bandit problem. 
The active set $B_{0}$ is initialized with all the arms from $\mathcal{A}$. We divide the entire budget $T$ into rounds/phases like in UCB-Improved, CCB, SAR and CSAR. At every time-step AugUCB checks for arm elimination conditions, while updating parameters at the end of each round. As suggested by \cite{liu2016modification} to make AugUCB to overcome too much early exploration, we no longer pull all the arms equal number of times in each round. Instead, we choose an arm in the active set $B_m$ that minimizes $(|\hat{r}_{i} - \tau |-2s_i)$ where 
%$\min_{i\in B_{m}}\big\lbrace |\hat{r}_{i} - \tau | - 2\sqrt{\frac{\rho_v\psi_m \hat{V}_{i} \log ( T \epsilon_{m})}{4 n_{i}} + \frac{\rho_v\psi_m \log{( T\epsilon_{m})}}{4 n_{i}}} \big\rbrace $
\begin{small}
\begin{align*}
s_i & = \sqrt{\frac{\rho\psi_m (\hat{v}_{i}+1) \log ( T \epsilon_{m})}{4 n_{i}}} %+ \frac{\rho\psi_m \log{( T\epsilon_{m})}}{4 n_{i}}}.
\end{align*}
\end{small} 
with $\rho$ being the arm elimination parameter and $\psi_{m}$ being the exploration regulatory factor.
%  in the active set $B_{m}$. 
The above condition ensures that an arm closer to the threshold $\tau$ is pulled; 
%and with suitable choice of $\rho_{\mu}$ and $\rho_v$ we can fine tune the exploration. 
parameter $\rho$ can be used to fine tune the elimination interval.
The choice of exploration factor, $\psi_m$,
% $\psi_m=\frac{T\epsilon_m}{(\log(\frac{3}{16} K\log K))^{2}}$ 
comes directly from \cite{audibert2010best} and \cite{bubeck2011pure} where it is  stated that in pure exploration setup, the exploring factor must be linear in $T$ (so that an exponentially small probability of error is achieved) rather than being logarithmic in $T$ (which is more suited for minimizing cumulative regret).

\begin{algorithm}[t!]
\caption{AugUCB}
\label{alg:augucb}
\begin{algorithmic}
\State {\bf Input:} Time budget $T$; parameter $\rho$; 
% $\rho_{\mu}$, $\rho_v$ 
  threshold $\tau$
\State {\bf Initialization:} $B_{0}=\mathcal{A}$; $m=0$; $\epsilon_{0}=1$;
\begin{small}
\begin{align*}
M&=\left\lfloor \frac{1}{2}\log_{2} \frac{T}{e}\right\rfloor; 
\hspace{2mm}\psi_{0}=\frac{T\epsilon_{0}}{128\Big(\log(\frac{3}{16}K\log K)\Big)^2}; \\
\ell_{0}&=\left\lceil \frac{2\psi_0\log( T\epsilon_{0})}{\epsilon_{0}} \right\rceil;
\hspace{2mm}N_{0}=K\ell_{0}
\end{align*}
\end{small}
%$M=\left\lfloor \frac{1}{2}\log_{2} \frac{T}{e}\right\rfloor $,  
%$\psi_{0}=\frac{T\epsilon_{0}}{(\log(\frac{3}{16}K\log K)^2}$,
% $\ell_{0}=\left\lceil \frac{2\psi\log( T\epsilon_{0})}{\epsilon_{0}} \right\rceil$ and 
% $N_{0}=K\ell_{0} $. Pull each arm once.
\State Pull each arm once
\vspace{-2mm}
\State \For{$t=K+1,..,T$}
\State Pull arm $j\in\argmin_{i\in B_{m}}\Big\lbrace |\hat{r}_{i} - \tau | - 2s_{i}\Big\rbrace$
%\State $t\leftarrow t+1$ 
\vspace{-4mm}
%\ArmElim
%\State For each arm $i \in B_{m}$, remove arm ${i}$ from $B_{m}$ if
%\begin{align*}
%\hat{r}_{i} + c_i  < \tau - c_i \mbox{ or } \hat{r}_{i} - c_i  > \tau + c_i \\
%\text{where $c_i=\sqrt{\frac{\rho_{\mu}\psi_{m}\log{( T\epsilon_{m})}}{2 n_{i}}}$}
%\end{align*}
%\EndArmElim
%\ArmElimV
%\State \For{$i\in B_m$}
%\State For each arm $i \in B_{m}$, remove arm ${i}$ from $B_{m}$ if
\State \For{$i\in B_m$}
\vspace{-4mm}
\State \If{$(\hat{r}_{i} + s_i  < \tau - s_i)$ or $(\hat{r}_{i} - s_i > \tau + s_i)$}
\State $B_m\leftarrow B_m\backslash\{i\}$\hspace{4mm} (Arm deletion)
\EndIf
\EndFor
%\begin{align*}
%\hat{r}_{i} + s_i  < \tau - s_i,\hspace{1mm} \mbox{ or } \hspace{1mm}\hat{r}_{i} - s_i  > \tau + s_i \\
%% \text{where $s_i=\sqrt{\frac{\rho\psi_{m}\hat{v}_{i}\log{( T\epsilon_{m})}}{4 n_{i}} + \frac{\rho\psi_{m} \log{(T\epsilon_{m})}}{4 n_{i}}}$}
%\end{align*}
%\EndFor
%\EndArmElimV
\vspace{-2mm}
\State \If{$t\geq N_{m}$ and $m \leq M$}
%\ResetParam
\State \textbf{Reset Parameters}
\State $\epsilon_{m+1}\leftarrow\frac{\epsilon_{m}}{2}$
\State $B_{m+1} \leftarrow B_{m}$
\State $\psi_{m+1}\leftarrow \frac{T\epsilon_{m+1}}{128(\log(\frac{3}{16}K\log K))^{2}}$
\State $\ell_{m+1}\leftarrow\left\lceil \frac{2\psi_{m+1}\log( T\epsilon_{m+1})}{\epsilon_{m+1}} \right\rceil$
\State $N_{m+1} \leftarrow t + |B_{m+1}|\ell_{m+1}$
\State $m \leftarrow m+1$
%\EndResetParam
\EndIf
\EndFor
\State \textbf{Output:} $\hat{S}_{\tau}=\lbrace i: \hat{r}_{i}\geq \tau \rbrace$.
\end{algorithmic}
\end{algorithm}


%Also because of the said condition, like \cite{liu2016modification} we also claim that AugUCB is an anytime algorithm.
