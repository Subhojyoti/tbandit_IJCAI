%%%% ijcai17.tex

\typeout{IJCAI-17 Instructions for Authors}

% These are the instructions for authors for IJCAI-17.
% They are the same as the ones for IJCAI-11 with superficical wording
%   changes only.

\documentclass{article}
% The file ijcai17.sty is the style file for IJCAI-17 (same as ijcai07.sty).
\usepackage{ijcai17}

% Use the postscript times font!
\usepackage{times}

\usepackage{macros}

\usepackage{latexsym} 


\title{Thresholding Bandits with Augmented UCB}
\author{Author names withheld}

\begin{document}

\maketitle

\begin{abstract}
In this paper we propose the Augmented-UCB (AugUCB) algorithm for a specific version of the thresholding bandit problem (TBP), where the objective is to identify a set of arms whose quality is above a threshold. A key feature of AugUCB is that it uses both mean and variance estimates to eliminate arms that have been sufficiently explored; to the best of our knowledge this is the first algorithm to employ such an approach for the considered TBP.  Theoretically, we obtain an upper bound on the loss (probability of mis-classification) incurred by AugUCB. Although UCBEV in literature provides a better guarantee, it is important to emphasize that UCBEV has access to problem complexity (whose computation requires arms' mean and variances), and hence is not realistic in practice; this is in contrast to AugUCB whose implementation does not require any such complexity inputs. We conduct extensive simulation experiments to validate the performance of AugUCB. Through our simulation work, we establish that AugUCB, owing to its utilization of variance estimates, performs significantly better than the state-of-the-art APT, CSAR and other non variance-based algorithms.


%%%%%%%
%In this paper we propose the Augmented-UCB (AugUCB) algorithm for the fixed-budget setting of a specific combinatorial, pure-exploration, stochastic multi-armed bandit setup called the thresholding bandit problem. Our algorithm is based on arm elimination, employing both mean and variance estimates and to our knowledge this is the first algorithm to employ such an approach in this setting. Through  simulation experiments we establish that our algorithm, owing to its utilization of variance estimates in arm elimination, performs significantly better than the state-of-the-art APT and CSAR algorithms, particularly when a large number of arms with different means and variances are involved. Theoretically, our algorithm is not comparable with APT or CSAR which use just mean estimation. AugUCB provides a weaker guarantee (in terms of an upper bound on the expected loss) than UCBEV, a variant of GapE-V \cite{gabillon2011multi} algorithm, modified for thresholding bandit problem. However, UCBEV requires access to the problem complexity (which is not realistic), while AugUCB requires no such complexity parameters as input. 
%%%%%%%%%%%



%is an anytime arm elimination variance-aware algorithm, and is the first of its kind which employs arm elimination 
%In earlier works, it was seen that algorithms using variance estimates outperform other competing algorithms.

%We propose the  Augmented-UCB (AugUCB) algorithm for the thresholding bandit problem, which is an instance of the combinatorial fixed-budget pure-exploration stochastic multi-armed bandit setup.
%the latter ones since it is a variance-aware algorithm. In the considered test cases comprising large number of arms, our algorithm has consistently performed much better than the state-of-the art APT and CSAR algorithms.  
\end{abstract}

%\begin{keywords}
%Multi-Armed Bandit, Regret, Exploration-exploitation, UCB
%\end{keywords}

\section{Introduction}
\label{intro}
In this paper, we study a specific combinatorial pure exploration problem called thresholding bandit problem (TBP) in the stochastic multi-armed bandit (MAB) setting. MAB problems are classic sequential decision making problems where the learner is provided with a set of actions (or arms) whose rewards are i.i.d samples from the distribution specific to the arm $i\in A$ and whose expected mean is denoted by $r_{i},\forall i\in A$. The learner's job is to identify the best arm whose expected mean is denoted by $r^{*}$. So, at very timestep the learner selects an arm $i$ and hence faces the \emph{exploration-exploitation dilemma} whereby it could pull the arm which has the highest observed mean reward (or $\hat{r}_{i}$) till now (exploitation) or to explore other arms, with the prospect of finding superior performance which was previously unobserved (exploration).

%In the stochastic multi-armed bandit setting a learning agent is required to choose from a set of decisions or arms at every round. The agent is then presented with a reward for that round, which is an independent draw from a stationary distribution specific to the arm selected. The agent, however, does not know the mean of the distributions associated with each arm, denoted by $r_{i}$, including the optimal arm which will give it the best reward, denoted by $r^{*}$. The agent attempts to make arm choices that will maximize some performance measure by keeping track of the reward that has been gathered from previous selections of the arm, for each arm. This is called the estimated mean reward of an arm denoted by $\hat{r}_{i}$. The bandit problem can be conceptualized as a sequential decision making process where the agent is at each round presented with an \emph{exploration-exploitation dilemma}. The agent could pull the arm which has the highest observed mean reward till now (exploitation) or to explore other arms, with the prospect of finding superior performance which was previously unobserved (exploration). 

%
%	Formally, let $r_i$, $i=1,\ldots,K$ denote the mean rewards of the $K$ arms and $r^* = \max_i r_i$ the optimal mean reward. The objective in some of the stochastic bandit problem is to minimize the cumulative regret, which is defined as follows:
%\begin{align*}
%R_{T}=r^{*}T - \sum_{i\in A} r_{i}N_{i}(T),
%\end{align*}
%where $T$ is the number of rounds, $N_{i}(T)=\sum_{m=1}^T I(I_m=i)$ is the number of times the algorithm chose arm $i$ up to round $T$.
%The expected regret of an algorithm after $T$ rounds can be written as,
%
%\begin{align*}
%\E[R_{T}]= \sum_{i=1}^K \E[N_i(T)] \Delta_i,
%\end{align*}
%where $\Delta_{i}=r^{*}-r_{i}$ denotes the gap between the means of the optimal arm and of the $i$-th arm. 

	In the pure exploration thresholding bandit setup the goal is different than minimizing the cumulative regret, that is the total loss suffered by the learner for not selecting the optimal arm throughout the time horizon $T$. Here the learning algorithm is provided with a threshold $\tau$ and it has to output all such arms $i$ whose $r_{i}$ is above $\tau$ after $T$ rounds. This is a specific instance of combinatorial pure exploration where the learning algorithm can explore as much as possible given a fixed horizon $T$ and not be concerned with the usual exploration-exploitation dilemma. Formally we can define a set $S_{\tau}=\lbrace i\in A: r_{i}\geq \tau \rbrace$ and the complementary set $S_{\tau}^{C}=\lbrace i\in A: r_{i} < \tau \rbrace$. Also we define $\hat{S}_{\tau}=\hat{S}_{\tau}(T)\subset A$ and its complementary set $\hat{S}_{\tau}^{C}$ as the recommendation of the learning algorithm after $T$ rounds. Given such sets exists, the performance of the learning agent is measured by how much accuracy it can discriminate between $S_{\tau}$ and $S_{\tau}^{C}$ after time horizon $T$. The loss $\Ls$ is defined as:-
\begin{align*}
\Ls (T) = I\big(\lbrace S_{\tau}\cap \hat{S}_{\tau}^{C}\neq \emptyset\rbrace    \cup    \lbrace\hat{S}_{\tau}\cap S_{\tau}^{C}\neq \emptyset\rbrace\big)
\end{align*}			
The goal of the learning agent is to minimize $\Ls (T)$. So, the expected loss after $T$ rounds is,
\begin{align*}
\E[\Ls(T)] = \Pb\big(\lbrace S_{\tau}\cap \hat{S}_{\tau}^{C} \neq \emptyset \rbrace  \cup   \lbrace \hat{S}_{\tau}\cap S_{\tau}^{C} \neq \emptyset\rbrace\big)
\end{align*}
which we can say is the probability of making mistake, that is whether the learning agent at the end of round $T$ rejects arms from $S_{\tau}$ or accepts arms from $S_{\tau}^{C}$ in its final recommendation. 

%Also, we are looking at an anytime algorithm, so the knowledge of $T$ may not be known to the learner.

%\subsection{Motivation}
%\label{motivation}
%%%%%%%%%%%%%%%%%%%%%%%%%%
%Original Version
%%%%%%%%%%%%%%%%%%%%%%%%%%
%The above TBP formulation has several applications, for instance, from areas ranging from anomaly detection and classification (see  \citet{locatelli2016optimal}) to industrial application. Particularly in industrial applications a learners objective is to choose (i.e., keep in  operation) all machines whose productivity is above a threshold. The TBP also finds applications in mobile communications (see \citet{audibert2010best})  where the users are to be allocated only those channels whose quality is above an acceptable threshold.


%New Version
The above TBP formulation has several applications, for instance, from areas ranging from anomaly detection and classification (see  \citet{locatelli2016optimal}) to industrial applications as well as in mobile communications (see \citet{audibert2010best})  where the users are to be allocated only those channels whose quality is above an acceptable threshold.
\subsection{Related Work}
\label{prevRes}
%Bandit problems and their variants have been studied in various domains under different names. An original motivation to the problem is seen in \cite{thompson1933likelihood}, where the author looks at a clinical trials problem of administering two treatments to patients who come in sequentially. Subsequently, various studies such as \cite{robbins1952some} and \cite{lai1985asymptotically}, popularize this broad problem statement under the topic of  \textit{sequential design of experiments}. The later go on  prove an asymptotic lower bound for the regret. The use of an Upper Confidence Bounds(UCB) as a strategy to select arms is first mentioned in the seminal paper by \cite{auer2002finite}, which is also accompanied with a proof on the regret bound. UCB policies are simple to implement and computationally efficient as they do not require large amounts of memory or complex policies of deciding which action to take at which state. A further modification of the UCB algorithm is proposed in \cite{auer2010ucb}, where a round based variant of the UCB algorithm  called UCB Revisited/UCB-Improved. In this manuscript we will call this as UCB-Revisited. All of these papers considered only the stationary distribution version of the bandit problem, that is the distribution from which the rewards are sampled remain fixed over time, but each arm could have a different mean and variance. This is the same context studied in our paper. It is also noteworthy that some UCB algorithms make no explicit variance estimation, such as UCB-1, UCB-2 by \cite{auer2002finite}, and UCB-Revisited by \cite{auer2010ucb}. However, other contributions such as UCB-Normal by \cite{auer2002finite} and UCB-Variance by \cite{audibert2009exploration} do make explicit variance estimates.  Similar to the algorithms UCB-1, UCB-2 and UCB-Revisited, this study also does not use any variance estimation. In fact, the experiments we consider in our study assume that the rewards for each arm come from unique stationary distribution which have different means but the same variance. Various other modifications to the problem statement have also been studied. One example is \citep{bubeck2013bounded}, which studies the stochastic multi-armed bandit problem where the value $r^{*}$ of the optimal arm is known, as a well as a positive lower bound on the smallest positive gap that is the $\min_{i\in A}{\Delta}_{i}, \forall i\in A$.
% 
%
%\paragraph{}The use of round based or phased algorithms, which is at the kernel of the proposed algorithm, can be seen in other studies belonging to the stochastic bandit literature. One variant was originally proposed by \cite{even2006action}, where they come up with algorithms called Successive Elimination and Median Elimination and provide PAC(Probably Approximately Correct) guarantees for them. In PAC guarantee algorithms the learning agent comes up with an $\epsilon$-optimal arm with $\delta$ error probability. Even UCB algorithms can be divided into non-round based and round-based; UCB1 falls under the former category, whereas UCB-Revisited falls under the latter. In the non-round based class of UCB algorithms the agent behaves greedily over a reward function combined with an confidence interval term which creates a fine balance between exploration and exploitation. However, in the round based algorithms, this balance is achieved in another way. Typically, in these algorithms the learning agent deletes some sub-optimal arm(s) at each round and is left with just one arm at the end, which is outputed with a certain error probability. Other approaches to phase wise elimination have also been explored under modified problem statements, an example of this is the Successive Reject algorithm which looks at the budgeted bandit scenario \cite{audibert2010best}. 
%
%\paragraph{}A phase-wise separation of exploration-exploitation is also tackled in \cite{perchet2015batched} which points out that FDA (Food and Drug Administration-USA) also use these type of methods to conduct clinical trials that is having a pilot trial, full trial and then diffusion into larger population. The paper also comes up with a Explore-Then-Commit policy in a batched bandit scenario. 
%%\cite{bubeck2013bounded}, studied  the stochastic multi-armed bandit problem when one knows the value $r^{*}$ of an optimal arm, as a well as a positive lower bound on the smallest positive gap that is the $\min_{i\in A}{\Delta}_{i}, \forall i\in A$.

	A significant amount of work has been done on the stochastic multi-armed bandit setting regarding minimizing cumulative regret with a single optimal arm. For a survey of such works we refer the reader to \cite{bubeck2012regret}. An early work involving a bandit setup is \cite{thompson1933likelihood}, where the author deals with the problem of choosing between two treatments to administer on patients who come in sequentially. Following the seminal work of  \cite{robbins1952some}, bandit algorithms have been extensively studied in a variety of applications. From a theoretical standpoint, an asymptotic lower bound for the regret was established in \cite{lai1985asymptotically}. Several other works such as \cite{auer2002finite},  \cite{audibert2009minimax} and \cite{auer2010ucb} have shown results for minimizing cumulative regret in stochastic bandit setup whereas works such as \cite{auer2002nonstochastic} have concentrated on adversarial bandit setup.
	
	
%	 An early work involving a bandit setup is \cite{thompson1933likelihood}, where the author deals the problem of choosing between two treatments to administer on patients who come in sequentially. Following the seminal work of  \cite{robbins1952some}, bandit algorithms have been extensively studied in a variety of applications. 
%From a theoretical standpoint, an asymptotic lower bound for the regret was established in \cite{lai1985asymptotically}. In particular, it was shown that for any consistent allocation strategy, we have
%$\liminf_{T \to \infty}\frac{\E[R_{T}]}{\log T}\geq\sum_{\{i:r_{i}<r^{*}\}}\dfrac{(r^{*}-r_{i})}{D(p_{i}||p^{*})},$
%where $D(p_{i}||p^{*})$ is the Kullback-Leibler divergence between the reward densities $p_{i}$ and $p^{*}$, corresponding to arms with mean $r_{i}$ and $r^{*}$, respectively.

%	There have been several algorithms with strong regret guarantees. The foremost among them is UCB1 by  \cite{auer2002finite}, which has a regret upper bound of $O\bigg(\dfrac{K\log T}{\Delta}\bigg)$, where $\Delta = \min_{i:\Delta_i>0} \Delta_i$. This result is asymptotically order-optimal for the class of distributions considered. However, the worst case gap independent regret bound of UCB1  can be as bad as $O \bigg(\sqrt{TK\log T}\bigg)$.  In \cite{audibert2009minimax}, the authors propose the MOSS algorithm and establish that the worst case regret of MOSS is $O\bigg(\sqrt{TK}\bigg)$ which improves upon UCB1 by a factor of order $\sqrt{\log T}$. However, the gap-dependent regret of MOSS is  $O\left(\dfrac{K^{2}\log\left(T\Delta^{2}/K\right)}{\Delta}\right)$ and in certain regimes, this can be worse than even UCB1 (see \cite{audibert2009minimax},\cite{lattimore2015optimally}). The UCB-Improved algorithm, proposed in \cite{auer2010ucb}, is a round-based algorithm\footnote{An algorithm is \textit{round-based} if it pulls all the arms equal number of times in each round and then proceeds to eliminate one or more arms that it identifies to be sub-optimal.} variant of UCB1 that has a gap-dependent regret bound of $O\bigg(\dfrac{K\log T\Delta^{2}}{\Delta}\bigg)$, which is better than that of UCB1. On the other hand, the worst case regret of UCB-Improved is $O\bigg(\sqrt{TK\log K}\bigg)$. 

	In the pure exploration setup, a significant amount of research has been done on finding the best arm(s) from a set of arms. The pure exploration setup has been explored in mainly two settings:-
\begin{enumerate}
\item Fixed Budget setting: In this setting the learning algorithm has to suggest the best arm(s) within a fixed number of attempts that is given as an input. The objective here is to maximize the probability of returning the best arm(s). One of the foremost papers to deal with single best arm identification is \cite{audibert2009exploration} where the authors come up with the algorithm UCBE and Successive Reject(SR) with simple regret guarantees. The relationship between cumulative regret and simple regret is proved in \cite{bubeck2011pure} where the authors prove that minimizing the simple regret necessarily results in maximizing the cumulative regret. In the combinatorial fixed budget setup \cite{gabillon2011multi} come up with Gap-E and Gap-EV algorithm which suggests the best $m$ (given as input) arms at the end of the budget with high probability. Similarly, \cite{bubeck2013multiple} comes up with the algorithm Successive Accept Reject(SAR) which is an extension of the SR algorithm. SAR is a round based algorithm whereby at the end of round an arm is either accepted or rejected based on certain conditions till the required top $m$ arms are suggested at the end of the budget with high probability. 
\item Fixed Confidence setting: In this setting the the learning algorithm has to suggest the best arm(s) with a fixed (given as input) confidence with as less number of attempts as possible. The single best arm identification has been handled in \cite{even2006action} where they come up with an algorithm called Successive Elimination (SE) which comes up with an arm that is $\epsilon$ close to the optimal arm. In the combinatorial setup recently \cite{kalyanakrishnan2012pac} have suggested the LUCB algorithm which on termination returns $m$ arms which are atleast $\epsilon$ close to the true top $m$ arms with $1-\delta$ probability.
\end{enumerate}	

	Apart from these two settings some unified approach has also been suggested in \cite{gabillon2012best} which proposes the algorithms UGapEb and UGapEc which can work in both the above two settings. A similar combinatorial setup was also explored in \cite{chen2014combinatorial} where the authors come up with more similarities and dissimilarities between these two settings in a more general setup. In their work, the learning algorithm, called Combinatorial Successive Accept Reject (CSAR) is similar to SAR with a more general setup. The thresholding bandit problem is a specific instance of the pure exploration setup of \cite{chen2014combinatorial}. In the latest work in \cite{locatelli2016optimal} the algorithm Anytime Parameter-Free Thresholding (APT) algorithm comes up with a better anytime guarantee than CSAR for the thresholding bandit problem.

\subsection{Our Contribution}
\label{contribution}
In this paper we propose the Algorithm AugUCB which is an action elimination algorithm suited for the TBP problem. It combines the approach of UCB-Improved, CCB (\cite{liu2016modification}) and APT algorithm. Our algorithm is a variance-aware algorithm which takes into account the empirical variance of the arms. We also address an open problem raised in \cite{auer2010ucb} of coming up with an algorithm that can eliminate arms based on variance. Both CSAR and APT are not variance-aware algorithms. The expected loss of various algorithms is shown in Table \ref{tab:regret-bds}. The terms $H_1, H_2, H_{CSAR,2}, H_1^{\sigma}$ and $H_2^{\sigma}$ signifies problem complexity and are defined in section \ref{results}. The term containing $H_{2}^{\sigma}$ is comparable to the similar terms (containing $H_1^{\sigma}$) for the error probability of Gap-EV(\cite{gabillon2011multi} algorithm which we modify to perform in the TBP problem and name it as UCBEV. The error probability of UCBEV for single bandit multi-armed case is given in Table \ref{tab:regret-bds}. We see that $\log(\frac{3}{16} K\log K) H_2^{\sigma} > H_1^{\sigma}$ and hence our algorithm is weaker with respect to UCBEV for single  multi-armed bandit scenario. But UCBEV algorithm needs the complexity factor $H_1^{\sigma}$ as input for optimal performance (which is not a realistic scenario) whereas AugUCB requires no such complexity factor as input. 
%Theoretically, we can compare the first term (containing $H_2$) of our expected loss and see that for all $K\geq 4$, $ H_2 \log(\frac{3}{16} K\log K) > (\log K)H_{CSAR,2}\geq H_1 $ and hence our result is weaker than CSAR and APT.

\begin{table}[h]
\caption{Expected Loss for different bandit algorithms}
\label{tab:regret-bds}
\begin{center}
\begin{tabular}{|p{1.3cm}|p{6.33cm}|}
\toprule
Algorithm  & Upper Bound on Expected Loss \\
\midrule
APT         &$\exp(-\frac{T}{64 H_1}+2\log((\log(T)+1)K))$ \\\midrule
CSAR		&$K^2\exp(-\frac{T-K}{72\log(K)H_{CSAR,2}})$ \\\midrule
UCBEV		&$\exp\left(-\frac{1}{512}\frac{T-2K}{H_1^{\sigma}} + \log\left(6KT\right)\right)$ \\\midrule
AugUCB      &$ \exp\left(- \frac{T}{4096 H_{2}^{\sigma} a} + \log\left(2KT\right) \right) $\newline
 where $a=\log(\frac{3}{16} K\log K)$
\\\bottomrule
\end{tabular}
\end{center}
\end{table}
Empirically we show that for a large number of arms when the variance of the arms lying above $\tau$ are high, our algorithm performs better than all other algorithms, except the algorithm UCBEV which has access to the underlying problem complexity and also is a variance-aware algorithm. AugUCB requires one input parameter and the exact choice for the parameter is derived in Theorem \ref{Result:Theorem:1}. Also, unlike SAR or CSAR, AugUCB does not have explicit accept or reject sets rather the arm elimination condition simply removes arm(s) if it is sufficiently sure that the mean of the arms are very high or very low about the threshold based on mean and variance estimation thereby re-allocating the remaining budget among the surviving arms. This although is a tactic similar to SAR or CSAR, but here at any round, an arbitrary number of arms can be accepted or rejected thereby improving upon SAR and CSAR which accepts/rejects one arm in every round. Also their round lengths are non-adaptive and they pull all the arms equal number of times in each round. 
%At every timestep AugUCB pulls the arm that minimizes thereby making this an anytime algorithm whereby we need not finish every round. 
%Irrespective of this case AugUCB also employs elimination of arms based on mean estimation only and is the first such algorithm which uses elimination by both mean and variance estimation simultaneously.

The remainder of the paper is organized as follows. In section \ref{algorithm} we present our AugUCB algorithm. 
Section \ref{results} contains our main theorem on expected loss, while section \ref{expt} contains simulation experiments. We finally draw our conclusions in section \ref{conclusion}.
%in section \ref{notation} we introduce the notations and the
%
%\section{Notation Used and Assumptions}
%\label{notation}
%\textbf{Notation and assumptions:} $\mathcal{A}$ denotes the set of arms, and $|\mathcal{A}|=K$ is the number of arms in $\mathcal{A}$. 
%Arms generic arm is indexed by $i,j\in\mathcal{A}$. 
For arm $i\in\mathcal{A}$, we use $r_{i}$ to denote the true mean of the distribution from which the rewards are sampled, while $\hat{r}_{i}(t)$ denotes the estimated mean at time $t$. Formally, using $n_i(t)$ to denote the number of times arm $i$ has been pulled until time $t$, we have $\hat{r}_{i}(t)=\frac{1}{n_{i}(t)}\sum_{z=1}^{n_i(t)} X_{i,z}$, where $X_{i,z}$ is the reward sample received when arm $i$ is pulled for the $z$-th time. %
Similarly, we use $\sigma_{i}^{2}$ to denote the true variance of the reward distribution corresponding to arm $i$, while $\hat{v}_{i}(t)$ is the estimated variance, i.e., $\hat{v}_{i}(t)=\frac{1}{n_i(t)}\sum_{z=1}^{n_{i}(t)}(X_{i,z}-\hat{r}_{i})^{2}$. Whenever there is no ambiguity about the underlaying  time index $t$, for simplicity we neglect $t$ from the notations and simply use  $\hat{r}_i, \hat{v}_i,$ and $n_i, $ to denote the respective quantities.  Let  $\Delta_{i}=|\tau-r_{i}|$ denote the distance of the true mean from the threshold $\tau$. Along the lines of \cite{locatelli2016optimal} we assume that all the reward distributions 
%from which rewards are sampled are identical and independent 
are $1$-sub-Gaussian (note that,  $1$-sub-Gaussian includes Gaussian distributions with variance less than $1$, distributions supported on an interval of length less than 2, etc). Also, the rewards are assumed to take values in $[0,1]$.
%. In our case, we  assume that all rewards are .

%Also we define $\Delta_{i}=r^{*} - r_{i}$ and $\hat{\Delta}_{i}=\hat{r}^{*} - \hat{r}_{i}$. In all cases $\min_{i\in A}{\Delta_{i}}$ is denoted by $\Delta$.
%and the optimal arm is denoted by $*$. The '*' superscript is used to denote anything related to optimal arm
%\paragraph*{}It is assumed that the distribution from which rewards are sampled are identical and independent sub-Gaussian distributions. Throughout the paper, we assume that the distributions $v_{i}$ are sub-Gaussian that is $\int e^{\lambda(x - r)} v_{i} (dx) ≤ e^{\lambda /2}, \forall \lambda \in \mathbb{R}$. Note that these include Gaussian distributions with variance less than 1 and distributions supported on an interval of length less than 2. All the experiments are also conducted with sub-Gaussians having variance as 1. Together with a Chernoff-Hoeffding bound, the sub-Gaussian assumption implies the following concentration inequality, valid for any
%$u > 0$,
%\newline
%\hspace*{8em}$\mathbb{P}\lbrace \hat{r}_{i} - r^{*} > u\rbrace \leq exp(-\dfrac{su^{2}}{2}) $
%\newline
%where s is the number of pulls of $a_{i}$. T is the horizon over which this entire algorithm runs.  $A^{'}$ at any round $m$ denotes the arms still not eliminated.
%\paragraph*{}The paper is organized as follows. We first present the algorithm in section 6. We then provide the proofs of Phase1 which includes regret-bound calculation and arm deletion conditions in section 7. In section 8, we provide proofs for Phase2 along with early stopping conditions. Section 9 deals with regret bound and then we provide error probability and error bounds in section 10. Experimental results are provided in section 11, and we conclude in section 12.
%
%
\vspace*{-1em}
\section{Augmented-UCB Algorithm}
\label{algorithm}
%The algorithm is presented below:-

%%%%%%%%%%%%%%%% alg-custom-block %%%%%%%%%%%%
\algblock{ArmElim}{EndArmElim}
\algnewcommand\algorithmicArmElim{\textbf{\em Arm Elimination by Mean Estimation}}
 \algnewcommand\algorithmicendArmElim{}
\algrenewtext{ArmElim}[1]{\algorithmicArmElim\ #1}
\algrenewtext{EndArmElim}{\algorithmicendArmElim}
\algtext*{EndArmElim}

\algblock{ArmElimV}{EndArmElimV}
\algnewcommand\algorithmicArmElimV{\textbf{\em Arm Elimination by Mean and Variance Estimation}}
 \algnewcommand\algorithmicendArmElimV{}
\algrenewtext{ArmElimV}[1]{\algorithmicArmElimV\ #1}
\algrenewtext{EndArmElimV}{\algorithmicendArmElimV}
\algtext*{EndArmElimV}

\algblock{ResetParam}{EndResetParam}
\algnewcommand\algorithmicResetParam{\textbf{\em Reset Parameters}}
 \algnewcommand\algorithmicendResetParam{}
\algrenewtext{ResetParam}[1]{\algorithmicResetParam\ #1}
\algrenewtext{EndResetParam}{\algorithmicendResetParam}
\algtext*{EndResetParam}

\begin{algorithm}[th!]
\caption{AugmentedUCB}
\label{alg:augucb}
\begin{algorithmic}
\State {\bf Input:} Time horizon $T$, exploration parameters $\rho_{\mu}$, $\rho_v$ and $\psi$, threshold $\tau$.
\State {\bf Initialization:} Set $B_{0}:=A$, $M=\bigg\lfloor \frac{1}{2}\log_{2} \frac{T}{e}\bigg\rfloor $, $m:=0$, $\epsilon_{0}:=1$, $\ell_{0}=\big\lceil \frac{2\log(\psi T\epsilon_{0}^{2})}{\epsilon_{0}} \big\rceil$ and $N_{0}=K*\ell_{0} $.
\State Pull each arm once
%\State \For{$m=0,1,..\big \lfloor \dfrac{1}{2}\log_{2} \dfrac{T}{e}\big\rfloor$}{	
%\State $N_{0}=\big\lfloor \dfrac{T}{M} \big\rfloor$
\State \For{$t=K+1,..,T$}
%\State Pull arm $i$ in $B_m$ such that $\min_{i\in B_{m}}\bigg\lbrace |\hat{r}_{i} - \tau | - \sqrt{\dfrac{\rho_v \hat{V}_{i} \log (\psi T \epsilon_{m}^{2})}{4 n_{i}} + \dfrac{\rho_v \log{(\psi T\epsilon_{m}^{2})}}{4 n_{i}}} \bigg\rbrace$, where $n_{i}$ is the number of times the arm $i$ has been pulled.
\State Pull arm $i$ in $B_m$ such that $\min_{i\in B_{m}}\bigg\lbrace |\hat{r}_{i} - \tau | - 2s_{i}\bigg\rbrace$
\State $t:=t+1$ 
\ArmElim
\State For each arm $i \in B_{m}$, remove arm ${i}$ from $B_{m}$ if
\begin{align*}
%\hat{r}_{i} + \sqrt{\dfrac{\rho_{\mu}\log{(\psi T\epsilon_{m}^{2})}}{2 n_{i}}}  < \tau -\sqrt{\dfrac{\rho_{\mu}\log{(\psi T\epsilon_{m}^{2})}}{2 n_{i}}} 
\hat{r}_{i} + c_i  < \tau - c_i
\end{align*}
\State For each arm $i \in B_{m}$, remove arm ${i}$ from $B_{m}$ if
\begin{align*}
%\hat{r}_{i} - \sqrt{\dfrac{\rho_{\mu}\log{(\psi T\epsilon_{m}^{2})}}{2 n_{i}}}  > \tau +\sqrt{\dfrac{\rho_{\mu}\log{(\psi T\epsilon_{m}^{2})}}{2 n_{i}}} 
\hat{r}_{i} - c_i  > \tau + c_i
\end{align*}
where $ c_i=\sqrt{\frac{\rho_{\mu}\log{(\psi T\epsilon_{m}^{2})}}{2 n_{i}}} $
\EndArmElim
\ArmElimV
\State For each arm $i \in B_{m}$, remove arm ${i}$ from $B_{m}$ if
\begin{align*}
%\hat{r}_{i} + \sqrt{\dfrac{\rho_v\hat{V}_{i}\log{(\psi T\epsilon_{m}^{2})}}{2 n_{i}} + \dfrac{\rho_v \log{(\psi T\epsilon_{m}^{2})}}{4 n_{i}}}  < \tau -\sqrt{\dfrac{\rho_v\hat{V}_{i}\log{(\psi T\epsilon_{m}^{2})}}{4 n_{i}} + \dfrac{\rho_v \log{(\psi T\epsilon_{m}^{2})}}{2 n_{i}}} 
\hat{r}_{i} + s_i  < \tau - s_i 
\end{align*}
\State For each arm $i \in B_{m}$, remove arm ${i}$ from $B_{m}$ if
\begin{align*}
%\hat{r}_{i} - \sqrt{\dfrac{\rho_v\hat{V}_{i}\log{(\psi T\epsilon_{m}^{2})}}{4 n_{i}} + \dfrac{\rho_v \log{(\psi T\epsilon_{m}^{2})}}{4 n_{i}}}  > \tau +\sqrt{\dfrac{\rho_v\hat{V}_{i}\log{(\psi T\epsilon_{m}^{2})}}{4 n_{i}} + \dfrac{\rho_v \log{(\psi T\epsilon_{m}^{2})}}{4 n_{i}}} 
\hat{r}_{i} - s_i  > \tau + s_i
\end{align*}
where $s_i=\sqrt{\frac{\rho_v\hat{V}_{i}\log{(\psi T\epsilon_{m}^{2})}}{4 n_{i}} + \frac{\rho_v \log{(\psi T\epsilon_{m}^{2})}}{4 n_{i}}}$
\EndArmElimV
\State \If{$t\geq N_{m}$ and $m \leq M$}
\ResetParam
\State $\epsilon_{m+1}:=\frac{\epsilon_{m}}{2}$
\State $B_{m+1} := B_{m}$
%\State $p_{m+1}:=p_{m}+1$
%\State $M := \dfrac{p+1}{p}$
\State $\ell_{m+1}:=\bigg\lceil \frac{2\log(\psi T\epsilon_{m+1}^{2})}{\epsilon_{m+1}} \bigg\rceil$
\State $N_{m+1} := t + |B_{m+1}|\ell_{m+1}$
% + \lfloorM N_{m}\rfloor $
\State $m := m+1$
\EndResetParam
\EndIf
\EndFor
\State Output $\hat{S}_{\tau}=\lbrace i: \hat{r}_{i}\geq \tau \rbrace$.
\end{algorithmic}
\end{algorithm}

%The threshold $\tau$ is also given as an input. AugUCB combines the power of UCB-Improved (\cite{auer2010ucb}), APT (\cite{locatelli2016optimal}) and SAR (\cite{gabillon2011multi}) or CSAR(\cite{chen2014combinatorial}). The choice of $M$ comes from UCB-Improved which necessarily entails that the $\epsilon_{m}\geq \sqrt{\frac{e}{T}}$. So, $M$ is the total number of rounds and is the same as UCB-Improved.


In algorithm \ref{alg:augucb}, hence referred to as AugUCB, we have three exploration parameters, $\rho_{\mu}, \rho_v$ which are the arm elimination parameters and $\psi$ which is the exploration regulatory factor. The main approach is based on UCB-Improved with modifications suited for the thresholding bandit problem. The active set $B_{0}$ is initialized with all the arms from $A$. We divide the entire budget $T$ into rounds/phases as like UCB-Improved, CCB, SAR and CSAR. After the end of each such round $m$ we eliminate arm(s) from active set $B_{m}$ and update parameters. As suggested by \cite{liu2016modification} to make AugUCB an anytime algorithm and to overcome too much early exploration, we no longer pull all the arms equal number of times in each round but pull the arm that minimizes,  
$\min_{i\in B_{m}}\big\lbrace |\hat{r}_{i} - \tau | - 2\sqrt{\frac{\rho_v \hat{V}_{i} \log (\psi T \epsilon_{m}^{2})}{4 n_{i}} + \frac{\rho_v \log{(\psi T\epsilon_{m}^{2})}}{4 n_{i}}} \big\rbrace $
in the active set $B_{m}$. This condition makes it possible to pull the arms closer to the threshold $\tau$ and with suitable choice of $\rho_{\mu},\rho_v$ and $\psi$ we can fine tune the exploration. 

%
\section{Theoretical Results}
\label{results}


\subsection{Problem Complexity}

We define problem complexity as,
\begin{align*}
H_{1} = \sum_{i=1}^{K}\dfrac{1}{\Delta_{i}^{2}} \text{ ,   } H_{2}=\max_{i\in A}\dfrac{i}{{\Delta_{i}^{2}}} \text{, where } \Delta_{i}=|r_{i}-\tau|
\end{align*}

This is same as the problem complexity defined in \cite{locatelli2016optimal} for the thresholding bandit problem and is similar to the problem complexity defined in \cite{audibert2010best} for single best arm identification. Also we know that,

\begin{align*}
H_{2}\leq H_{1}\leq \log(2K)H_{2}
\end{align*}

Also, we define $H_{1}^{\sigma}$ (\cite{gabillon2011multi}) and $H_{2}^{\sigma}$ as,
\begin{align*}
& H_{1}^{\sigma}=\sum_{i=1}^{K}\frac{\sigma_{i}+\sqrt{\sigma_{i}^{2}+(16/3)\Delta_{i}}}{\Delta_{i}^{2}}\\
& H_{2}^{\sigma}=\max_{i\in A} i\frac{3\sigma_{i}^{2} + \Delta_i}{\Delta_i^{2}}
\end{align*}

%\begin{align*}
%H_{1}^{\sigma}=\sum_{i=1}^{K}\frac{\sigma_{i}+\sqrt{\sigma_{i}^{2}+(16/3)\Delta_{i}}}{\Delta_{i}^{2}} \geq \sum_{i=1}^{K}\frac{3\sigma_{i}^{2} + \Delta_k}{\Delta_{i}^{-2}}\geq \sum_{i=1}^{K}\frac{3\sigma_{i}^{2} + \Delta_k}{\Delta_{i}^{-2}}
%\end{align*}

which also gives us that $H_{2}^{\sigma} < H_{1}^{\sigma}$.


\subsection{Theorem 1}

\begin{theorem}
\label{Result:Theorem:1}

The expected loss of the AugUCB algorithm is given by,
\begin{small}
\begin{align*}
&\E[\Ls(T)] \leq \exp\bigg( -\frac{T\log (2 K\sqrt{\log K})}{H_2 K (\log K)^2} + \log\bigg(K\big(\log_2\frac{T}{e}+1\big)\bigg)\bigg)\\
& + \exp\bigg(- \frac{33T\log ( K\sqrt{\log K})}{H_{2}^{\sigma} K(\log K)^2}  + \log\bigg(K\big(\log_2\frac{T}{e}+1\big)\bigg)\bigg) 
\end{align*}
\end{small}

for $\psi=\frac{T}{\log K}$, $\rho_{\mu}=\frac{1}{8}$ and $\rho_v=\frac{1}{3}$.

%For every $0<\eta <1$ and $\gamma > 1$, there exists time $t$ such that for all $T>t$ the simple regret of AugUCB is upper bounded by,
%\begin{small}
%\begin{align*}
%& SR_{AugUCB} \leq \sum_{i=1}^{K} \Delta_{i}\bigg\lbrace \exp\bigg(-4\rho\log (\psi T\frac{\Delta_{i}^{4}}{16\rho^{2}})-\dfrac{c_{0}\sqrt{T}}{16\rho H_{2}}\\
%& + \log \big( 16\gamma C_1\log_{2}\dfrac{T}{e} \big) \bigg) + \exp\bigg(- \dfrac{3\rho_v}{2} \bigg(\dfrac{2\sigma_{i}^{2}+\Delta_{i}+2}{6\sigma_{i}^{2}+\Delta_{i}}\bigg)\log(\psi T\frac{\Delta_{i}^{4}}{16\rho_{v}^{2}})\\
%& -\dfrac{c_{0}\sqrt{T}}{16\rho_v H_{2}} + \log\big ( 32\gamma C_2\log_{2}\dfrac{T}{e} \big)  \bigg)\bigg\rbrace
%\end{align*}
%\end{small}
%with probability at least $1-\eta$, where $c_{0}>0$ is a constant and $C_1=\dfrac{K\rho\log (\psi T \frac{\Delta_{i}^{4}}{16\rho^{2}})}{T\Delta_{i}^{2}}$ and $C_2= \dfrac{K\rho_v\log (\psi T \frac{\Delta_{i}^{4}}{16\rho_{v}^{2}})}{T\Delta_{i}^{2}}$.

\end{theorem}


\begin{proof}

According to the algorithm, the number of rounds is $m=\lbrace 0,1,2,.. M\rbrace $ where $M=\bigg\lfloor \frac{1}{2}\log_{2} \frac{T}{e}\bigg\rfloor$. So, $\epsilon_{m}\geq 2^{-M}\geq \sqrt{\frac{e}{T}}$. Also each round $m$ consists of $|B_{m}|\ell_{m}$ timesteps where $\ell_{m} = \frac{2\log(\psi T \epsilon_{m}^{2})}{\epsilon_{m}}$ and $B_{m}$ is the set of all surviving arms. 

Let $c_{i} = \sqrt{\frac{\rho_{\mu}\log{(\psi T\epsilon_{m}^{2})}}{2 n_{i}}}$ denote the confidence interval, where $n_{i}$ is the number of times an arm $i$ is pulled. Let $A^{'}=\lbrace i\in A|\Delta_{i}\geq b\rbrace$, for $b\geq \sqrt{\frac{e}{T}}$. Let $m_{i}$ be the minimum round that an arm $i$ gets eliminated such that $m_{i}=min\lbrace m| \sqrt{\rho_{\mu}\epsilon_{m}}<\frac{\Delta_{i}}{2}\rbrace$. 


Let $s_{i}=\sqrt{\frac{\rho_v \hat{V_{i}} \log{(\psi T\epsilon_{g}^{2})}}{4 n_{i}} + \frac{\rho_v \log{(\psi T\epsilon_{g}^{2})}}{4 n_{i}}}$. Let $g_{i}$ be the minimum round that an arm $i$ gets eliminated such that $g_{i}=min\lbrace g| \sqrt{\rho_{v}\epsilon_{g}}<\frac{\Delta_{i}}{2}\rbrace$. 
%In this proof sub-optimal arms refer to the arms whose $r_{i}$ is lower than the threshold $\tau$.

%At the end of any round $\max\lbrace m_{i},g_{i}\rbrace$, for any arm $i$, two cases are possible.

Let $\xi_{1}$ and $\xi_{2}$ be the event such that,
\begin{align*}
\xi_{1}&=\bigg\lbrace \forall i\in A, \forall m=0,1,2,..,M: |\hat{r_i} - r_i| \leq 2c_i\bigg\rbrace\\
\xi_{2}&=\bigg\lbrace \forall i\in A, \forall m=0,1,2,..,M: |\hat{r_i} - r_i| \leq  2s_i\bigg\rbrace
\end{align*}

So, $\xi_{1}$ and $\xi_{2}$ signifies the event till when any arm $i$ will not get eliminated from $B_m$.

%Similarly, $\xi_{2}=\lbrace \forall i\in A \text{ s.t } r_i < \tau \text{ or }\forall j\in A \text{ s.t } r_j \geq \tau; \forall t=1,2,..,T | r_i - s_i  > \tau + s_i \text{ or } r_j + s_i < \tau - s_i  \rbrace$. 

%Let $\xi_{1}$ be the event such that $\xi_{1}=\lbrace \forall i\in A \text{ s.t } r_i < \tau \text{ or }\forall j\in A 
%\text{ s.t } r_j \geq \tau; \forall t=1,2,..,T | r_i - c_i  > \tau + c_i \text{ or } r_j + c_i < \tau - c_i  \rbrace$. Similarly, $\xi_{2}=\lbrace \forall i\in A \text{ s.t } r_i < \tau \text{ or }\forall j\in A \text{ s.t } r_j \geq \tau; \forall t=1,2,..,T | r_i - s_i  > \tau + s_i \text{ or } r_j + s_i < \tau - s_i  \rbrace$. 

\subsubsection{\textit{Some arm i is not eliminated on or before round $\max\lbrace m_{i},g_{i}\rbrace$}}
For any arm $i$, if it is eliminated from active set $B_{m_{i}}$ then the below two events have to come true,
\begin{small}
\begin{align}
\hat{r}_{i} + c_{i} < \tau - c_{i}, \label{eq:armelim-casea}\\
\hat{r}_{i} - c_{i} > \tau + c_{i}, \label{eq:armelim-caseb}
\end{align}
\end{small}


For \ref{eq:armelim-casea} we can see that it eliminates arms that have performed poorly and removes them them from $B_{m_{i}}$. Similarly, \ref{eq:armelim-caseb} eliminates arms from $B_{m}$ that have performed very well compared to threshold $\tau$.

Each round consist of $|B_{m_{i}}|\ell_{m_{i}}$ timesteps. In the $m_{i}$-th round an arm $i$ can be pulled no more than $\ell_{m_{i}}$ times. So when $n_{i}=\ell_{m_{i}}$, putting the value of $\ell_{m_{i}}=\frac{2\log{(\psi T\epsilon_{m_{i}}^{2})}}{\epsilon_{m_{i}}}$ in $c_{i}$ we get, 
\begin{small}
\begin{align*}
c_{i}&=\sqrt{\frac{\rho_{\mu}\epsilon_{m_{i}}\log (\psi T\epsilon_{m_{i}}^{2})}{2 n_{i}}}
=\sqrt{\frac{\rho_{\mu}\epsilon_{m_{i}}\log (\psi T\epsilon_{m_{i}}^{2})}{2*2 \log(\psi T\epsilon_{m_{i}}^{2})}}\\
& =\frac{\sqrt{\rho_{\mu}\epsilon_{m_{i}}}}{2}
 \leq \sqrt{\rho_{\mu}\epsilon_{m_{i}+1}} < \frac{\Delta_{i}}{4} \text{, as }\rho_{\mu}\in (0,1].
\end{align*}
\end{small}


Again, for ${i} \in A^{'}$ for \ref{eq:armelim-casea} elimination condition, 
\begin{small}
\begin{align*}
\hat{r}_{i} + c_{i}&\leq r_{i} + 2c_{i} = r_{i} + 4c_{i} - 2c_{i} \\
&< r_{i} + \Delta_{i} - 2c_{i} = \tau -2c_{i} \leq \tau - c_{i}
\end{align*}
\end{small}


Also, for ${i} \in A^{'}$ for \ref{eq:armelim-caseb} elimination condition, 
\begin{small}
\begin{align*}
\hat{r}_{i} - c_{i}&\geq r_{i} - 2c_{i} = r_{i} - 4c_{i} + 2c_{i} \\
&> r_{i} - \Delta_{i} + 2c_{i}\geq \tau + 2c_{i} \geq \tau + c_{i}
\end{align*}
\end{small}


Now, arm elimination condition is being checked at every timestep, in the $m_{i}$-th round as soon as $n_{i}=\ell_{m_{i}}$, arm $i$ gets eliminated. Applying Chernoff-Hoeffding bound and considering independence of complementary of the event in \ref{eq:armelim-casea},
\begin{small}
\begin{align*}
%\mathbb{P}\lbrace\hat{r}_{i}\geq r_{i} - 2c_{i}\rbrace &\leq exp(-2(\tau + 2c_{i})^{2}n_{i})\\
\mathbb{P}\lbrace\hat{r}_{i}\geq r_{i} + 2c_{i}\rbrace &\leq \exp(-4 c_{i}^{2}n_{i})\\
&\leq \exp(-8 * \dfrac{\rho_{\mu}\log (\psi T\epsilon_{m_{i}}^{2})}{2 n_{i}} *n_{i})\\
&\leq \exp\big(-4\rho_{\mu}\log (\psi T\epsilon_{m_{i}}^{2})\big)
\end{align*}
\end{small}
  
%  \begin{align*}
%\mathbb{P}\lbrace\hat{r}_{i}\geq r_{i} - (\tau + 2c_{i})\rbrace &\leq exp(-2(\tau + 2c_{i})^{2}n_{i})\\
%&\leq \exp(-2(2\tau c_{i})^{2}n_{i}) \text{ , as} (a+b)^{2} \geq (ab)^{2} \text{ for } a,b\in[0,1]\\
%&\leq \exp(-8 * \tau^{2}\dfrac{\rho_{\mu}\log (\psi T\epsilon_{m_{i}}^{2})}{2 n_{i}} *n_{i})\\
%&\leq \exp\big(-4\tau^{2}\rho_{\mu}\log (\psi T\epsilon_{m_{i}}^{2})\big)
%  \end{align*}
  
%Similarly, $\mathbb{P}\lbrace\hat{r}_{i}\leq r_{i} + (\tau + 2c_{i})\rbrace\leq \dfrac{1}{(\psi  T\epsilon_{m_{i}}^{2})^{4\tau^{2}\rho_{\mu}}}$

Similarly, $\mathbb{P}\lbrace\hat{r}_{i}\leq r_{i} - 2c_{i}\rbrace\leq \exp\big(-4\rho_{\mu}\log (\psi T\epsilon_{m_{i}}^{2})\big)$
 
%Summing, the two up, the probability that an arm ${i}$ is not eliminated on or before $m_{i}$-th round based on the \ref{eq:armelim-casea} and \ref{eq:armelim-caseb} elimination condition is  $\bigg(\dfrac{2}{(\psi T\epsilon_{m_{i}}^{2})^{4\tau^{2}\rho_{\mu}}}\bigg)$. 

Summing, the two up, the probability that an arm ${i}$ is not eliminated on or before $m_{i}$-th round based on the \ref{eq:armelim-casea} and \ref{eq:armelim-caseb} elimination condition is  $\big(2\exp\big(-4\rho_{\mu}\log (\psi T\epsilon_{m_{i}}^{2})\big)\big)$. 


Again for any arm $i$, if it is eliminated from active set $B_{g_{i}}$ then the below two events have to come true,
\begin{small}
\begin{align}
\hat{r}_{i} + s_{i} < \tau - s_{i}, \label{eq:armelim-var-casea}\\
\hat{r}_{i} - s_{i} > \tau + s_{i}, \label{eq:armelim-var-caseb}
\end{align}
\end{small}

For \ref{eq:armelim-var-casea} we can see that it eliminates arms that have performed poorly and removes them them from $B_{g_{i}}$. Similarly, \ref{eq:armelim-var-caseb} eliminates arms from $B_{g_{i}}$ that have performed very well compared to threshold $\tau$.

%But, we know that $\epsilon_{m_{i}}=\epsilon_{g_{i}}$ and round consist of $|B_{g_{i}}|\ell_{g_{i}}$ timesteps. 

In the $g_{i}$-th round an arm $i$ can be pulled no more than $\ell_{g_{i}}$ times. So when $n_{i}=\ell_{g_{i}}$, putting the value of $\ell_{g_{i}}=\frac{2\log{(\psi T\epsilon_{g_{i}}^{2})}}{\epsilon_{g_{i}}}$ in $s_{i}$ we get, 
\begin{small}
\begin{align*}
s_{i}&=\sqrt{\dfrac{\rho_v \hat{V}_{i} \epsilon_{g_{i}}\log (\psi T\epsilon_{g_{i}}^{2})}{4 n_{i}} + \dfrac{\rho_v \log{(\psi T\epsilon_{g_{i}}^{2})}}{4 n_{i}}} \\
&\leq \sqrt{\dfrac{\rho_v \epsilon_{g_{i}}\log (\psi T\epsilon_{g_{i}}^{2})}{4*2 \log(\psi T\epsilon_{g_{i}}^{2})} + \dfrac{\rho_v \epsilon_{g_{i}} \log{(\psi T\epsilon_{g_{i}}^{2})}}{4*2 \log(\psi T\epsilon_{g_{i}}^{2})} } \text{, as }\hat{V}_{i}\in [0,1].\\
& \leq \sqrt{\dfrac{\rho_v \epsilon_{g_{i}}}{8} + \dfrac{\rho_v \epsilon_{g_{i}}}{8} } \leq \dfrac{\sqrt{\rho_v \epsilon_{g_{i}}}}{2}\\
& \leq \sqrt{\rho_v \epsilon_{g_{i}+1}} < \dfrac{\Delta_{i}}{4} \text{, as }\rho_v\in (0,1].
\end{align*}
\end{small}


Again, for ${i} \in A^{'}$ for \ref{eq:armelim-var-casea} elimination condition,
\begin{small}
\begin{align*}
\hat{r}_{i} + s_{i}&\leq r_{i} + 2s_{i} = r_{i} + 4s_{i} - 2s_{i} \\
&< r_{i} + \Delta_{i} - 2s_{i} = \tau -2s_{i} \leq \tau - s_{i}
\end{align*}
\end{small} 


Also, for ${i} \in A^{'}$ for \ref{eq:armelim-var-caseb} elimination condition, 
\begin{small}
\begin{align*}
\hat{r}_{i} - s_{i}&\geq r_{i} - 2s_{i} = r_{i} - 4s_{i} + 2s_{i} \\
&> r_{i} - \Delta_{i} + 2s_{i}\geq \tau + 2s_{i} \geq \tau + s_{i}
\end{align*}
\end{small}


Since, arm elimination condition is being checked at every timestep, in the $g_{i}$-th round as soon as $n_{i}=\ell_{g_{i}}$, arm $i$ gets eliminated. Applying Bernstein inequality and considering independence of complementary of the event in \ref{eq:armelim-var-casea},
\begin{small}
\begin{align}
&\mathbb{P}\lbrace\hat{r}_{i}\geq r_{i} + 2s_{i}\rbrace\\
&\leq \mathbb{P}\bigg\lbrace \hat{r}_{i} \geq r_{i}+ ( 2\sqrt{\dfrac{\rho_v \hat{V}_{i}\log(\psi T\epsilon_{g_{i}}^{2}) + \rho_v \log{(\psi T\epsilon_{g_{i}}^{2})}}{4n_{i}} }) \bigg\rbrace\\
&\leq \mathbb{P}\bigg\lbrace \hat{r}_{i} \geq r_{i}+ (2\sqrt{\dfrac{\rho_v [\sigma_{i}^{2}+\sqrt{\rho_{v}\epsilon_{g_{i}}} + 1]\log(\psi T\epsilon_{g_{i}}^{2})}{4n_{i}}})\bigg\rbrace \label{eq:prob_eq1}\\ 
&+ \mathbb{P}\bigg\lbrace \hat{V}_{i}\geq \sigma_{i}^{2}+\sqrt{\rho_{v}\epsilon_{g_{i}}}\bigg\rbrace \label{eq:prob_eq2}
\end{align}
\end{small}
 
 
Now, we know that in the $g_{i}$-th round,
\begin{small}
\begin{align*}
& 2\sqrt{\dfrac{\rho_v [\sigma_{i}^{2}+\sqrt{\rho_{v}\epsilon_{g_{i}}}]\log(\psi T\epsilon_{g_{i}}^{2})}{4n_{i}} + \dfrac{\rho_v \log{(\psi T\epsilon_{g_{i}}^{2})}}{4 n_{i}}}\\ &\leq  2\sqrt{\dfrac{\rho_v [\sigma_{i}^{2}+\sqrt{\rho_{v}\epsilon_{g_{i}}}]\log(\psi T\epsilon_{g_{i}}^{2})}{\frac{8\log(\psi T \epsilon_{g_{i}}^{2})}{\epsilon_{g_{i}}}} + \dfrac{\rho_v \log{(\psi T\epsilon_{g_{i}}^{2})}}{\frac{8\log(\psi T \epsilon_{g_{i}}^{2})}{\epsilon_{g_{i}}}}}\\
& \leq \dfrac{\sqrt{\rho_v \epsilon_{g_{i}}[\sigma_{i}^{2}+\sqrt{\rho_{v}\epsilon_{g_{i}}} + 1]}}{2}\leq \sqrt{\rho_v \epsilon_{g_{i}}}
\end{align*}
\end{small}


For the term in \ref{eq:prob_eq1}, by applying Bernstein inequality, we can write as,
\begin{small}
\begin{align*}
&\mathbb{P}\bigg\lbrace \hat{r}_{i}\geq r_{i} + \bigg(2\sqrt{\frac{\rho_v [\sigma_{i}^{2}+\sqrt{\rho_{v}\epsilon_{g_{i}}} + 1]\log(\psi T\epsilon_{g_{i}}^{2})}{4n_{i}}  } \bigg)\bigg\rbrace\\
&\leq \exp\bigg(- \dfrac{\bigg(2\sqrt{\frac{\rho_v [\sigma_{i}^{2}+\sqrt{\rho_{v}\epsilon_{g_{i}}}]\log(\psi T\epsilon_{g_{i}}^{2})}{4n_{i}} + \frac{\rho_v \log{(\psi T\epsilon_{g_{i}}^{2})}}{4 n_{i}}}\bigg)^{2}n_{i}}{2\sigma_{i}^{2}+\frac{4}{3}\sqrt{\frac{\rho_v [\sigma_{i}^{2}+\sqrt{\rho_{v}\epsilon_{g_{i}}}]\log(\psi T\epsilon_{g_{i}}^{2})}{4n_{i}}+\frac{\rho_v \log{(\psi T\epsilon_{g_{i}}^{2})}}{4 n_{i}}}}\bigg) \\
&\leq \exp\bigg(- \dfrac{\bigg(\rho_v [\sigma_{i}^{2}+\sqrt{\rho_{v}\epsilon_{g_{i}}} + 1]\log(\psi T\epsilon_{g_{i}}^{2})\bigg)}{2\sigma_{i}^{2}+\frac{2}{3}\sqrt{\rho_v \epsilon_{g_{i}}}} \bigg)\\
&\leq \exp\bigg(- \dfrac{3\rho_v}{2} \bigg(\dfrac{\sigma_{i}^{2}+\sqrt{\rho_{v}\epsilon_{g_{i}}}+1}{3\sigma_{i}^{2}+\sqrt{\rho_v \epsilon_{g_{i}}}}\bigg) \log(\psi T\epsilon_{g_{i}}^{2}) \bigg) 
\end{align*}
\end{small}
 
  
For the term in \ref{eq:prob_eq2}, by applying Bernstein inequality, we can write as,
\begin{small}
\begin{align*}
&\mathbb{P}\bigg\lbrace \hat{V}_{i}\geq \sigma_{i}^{2}+\sqrt{\rho_{v}\epsilon_{g_{i}}}\bigg\rbrace\\
&\leq \mathbb{P}\bigg\lbrace \dfrac{1}{n_{i}}\sum_{t=1}^{n_{i}}(x_{i,t}-r_{i})^{2}-(\hat{r}_{i}-r_{i})^{2}\geq \sigma_{i}^{2}+\sqrt{\rho_{v}\epsilon_{g_{i}}}\bigg\rbrace\\
&\leq \mathbb{P}\bigg\lbrace \dfrac{\sum_{t=1}^{n_{i}}(x_{i,t}-r_{i})^{2}}{n_{i}}\geq \sigma_{i}^{2}+\sqrt{\rho_{v}\epsilon_{g_{i}}} \bigg\rbrace\\
&\leq \mathbb{P}\bigg\lbrace \dfrac{\sum_{t=1}^{n_{i}}(x_{i,t}-r_{i})^{2}}{n_{i}}\geq \sigma_{i}^{2} +\\
&\bigg(2\sqrt{\dfrac{\rho_v [\sigma_{i}^{2}+\sqrt{\rho_{v}\epsilon_{g_{i}}}]\log(\psi T\epsilon_{g_{i}}^{2})}{4n_{i}}+\frac{\rho_v \log{(\psi T\epsilon_{g_{i}}^{2})}}{4 n_{i}}}\bigg)\bigg\rbrace\\
% &\leq \exp\bigg(- \dfrac{\bigg(2\sqrt{\frac{\rho_v [\sigma_{i}^{2}+\sqrt{\rho_{v}\epsilon_{g_{i}}}]\log(\psi T\epsilon_{g_{i}}^{2})}{2n_{i}}}\bigg)^{2}n_{i}}{2\sigma_{i}^{2}+\frac{4}{3}\sqrt{\frac{\rho_v [\sigma_{i}^{2}+\sqrt{\rho_{v}\epsilon_{g_{i}}}]\log(\psi T\epsilon_{g_{i}}^{2})}{2n_{i}}}}\bigg) \\
%&\leq \exp\bigg(- \dfrac{\bigg(\rho_v [\sigma_{i}^{2}+\sqrt{\rho_{v}\epsilon_{g_{i}}}]\log(\psi T\epsilon_{g_{i}}^{2})\bigg)}{2\sigma_{i}^{2}+\frac{4}{3}\sqrt{\rho_v \epsilon_{g_{i}}}} \bigg)\\
&\leq \exp\bigg(- \dfrac{3\rho_v}{2} \bigg(\dfrac{\sigma_{i}^{2}+\sqrt{\rho_{v}\epsilon_{g_{i}}}+1}{3\sigma_{i}^{2}+\sqrt{\rho_v \epsilon_{g_{i}}}}\bigg) \log(\psi T\epsilon_{g_{i}}^{2}) \bigg) 
\end{align*}
\end{small}
 
  
Similarly, the condition for the complementary event for the elimination case \ref{eq:armelim-var-caseb} holds such that $\mathbb{P}\lbrace\hat{r}_{i}\leq r_{i} - 2s_{i}\rbrace \leq 2\exp\bigg(- \frac{3\rho_v}{2} \bigg(\frac{\sigma_{i}^{2}+\sqrt{\rho_{v}\epsilon_{g_{i}}}+1}{3\sigma_{i}^{2}+\sqrt{\rho_v \epsilon_{g_{i}}}}\bigg) \log(\psi T\epsilon_{g_{i}}^{2}) \bigg)$.

  Summing everything up, the probability that an arm ${i}$ is not eliminated on or before $g_{i}$-th round based on the \ref{eq:armelim-var-casea} and \ref{eq:armelim-var-caseb} elimination condition is  $4\exp\bigg(- \frac{3\rho_v}{2} \bigg(\frac{\sigma_{i}^{2}+\sqrt{\rho_{v}\epsilon_{g_{i}}}+1}{3\sigma_{i}^{2}+\sqrt{\rho_v \epsilon_{g_{i}}}}\bigg) \log(\psi T\epsilon_{g_{i}}^{2}) \bigg)$. 
  
  
%\begin{enumerate}
%\item \emph{Fact 1:} From above we know that the probability of elimination of a sub-optimal arm in the $\max\lbrace m_{i}, g_{i}\rbrace$-th round being not eliminated is bounded above by $P_{m_{i}}\leq 2\exp\bigg(-4\rho\log (\psi T\epsilon_{m_{i}}^{2})\bigg) + 4\exp\bigg(- \dfrac{3\rho_v}{2} \bigg(\dfrac{\sigma_{i}^{2}+\sqrt{\rho_{v}\epsilon_{g_{i}}}+1}{3\sigma_{i}^{2}+\sqrt{\rho_v \epsilon_{g_{i}}}}\bigg) \log(\psi T\epsilon_{g_{i}}^{2}) \bigg)$.
%\item \emph{Fact 2:} From \cite{tolpin2012mcts} we know that, for every $0<\eta <1$ and $\gamma > 1$, there exists $t$ such that for all $T>t$ the probability of a sub-optimal arm $i$ being sampled in the $m_{i}$-th round is bounded by $Q_{m_{i}}\leq 2\gamma \exp(-c_{m_{i}}\dfrac{\sqrt{T}}{2})$, where $c_{m_{i}}=\dfrac{c_{0}}{2^{m_{i}}}$.
%\end{enumerate}

We start with an upper bound on the number of plays $\delta_{\max\lbrace m_{i}, g_{i}\rbrace}$ in the $\max\lbrace m_{i}, g_{i}\rbrace$-th round. We know that the total number of arms surviving in the $\max\lbrace m_{i}, g_{i}\rbrace$-th arm is, 

\begin{small}
\begin{align*}
&|B_{\max\lbrace m_{i}, g_{i}\rbrace}|=2K\exp\bigg(-4\rho_{\mu}\log (\psi T\epsilon_{m_{i}}^{2})\bigg)\\ 
& + 4K\exp\bigg(- \frac{3\rho_v}{2} \big(\frac{\sigma_{i}^{2}+\sqrt{\rho_{v}\epsilon_{g_{i}}}+1}{3\sigma_{i}^{2}+\sqrt{\rho_v \epsilon_{g_{i}}}}\big) \log(\psi T\epsilon_{g_{i}}^{2}) \bigg)
\end{align*}     
\end{small}


Again for AugUCB, we know that the number of pulls allocated for each surviving arm $i$ in the $m_{i}$-th round is $\ell_{m_{i}}=\frac{2\log (\psi T \epsilon_{m_{i}}^{2})}{\epsilon_{m_{i}}}$ or for the $g_{i}$-th round is $\ell_{g_{i}}=\frac{2\log (\psi T \epsilon_{g_{i}}^{2})}{\epsilon_{g_{i}}}$. Therefore, the proportion of plays $\delta_{\max\lbrace m_{i}, g_{i}\rbrace}$ in the $\max\lbrace m_{i}, g_{i}\rbrace$-th round can be written as,

\begin{small}
\begin{align*}
&\delta_{\max\lbrace m_{i}, g_{i}\rbrace}=(|B_{m_{i}}|.\ell_{m_{i}}) + (|B_{g_{i}}|.\ell_{g_{i}})\\
&\leq 2K\exp\bigg(-4\rho_{\mu}\log (\psi T\epsilon_{m_{i}}^{2})\bigg).\dfrac{2\log (\psi T \epsilon_{m_{i}}^{2})}{\epsilon_{m_{i}}}\\
 & + 4K\exp\bigg(- \dfrac{3\rho_v}{2} \bigg(\dfrac{\sigma_{i}^{2}+\sqrt{\rho_{v}\epsilon_{g_{i}}}+1}{3\sigma_{i}^{2}+\sqrt{\rho_v \epsilon_{g_{i}}}}\bigg) \log(\psi T\epsilon_{g_{i}}^{2})\bigg).\dfrac{2\log (\psi T \epsilon_{g_{i}}^{2})}{\epsilon_{g_{i}}} \\
& \leq \dfrac{4K\log (\psi T \epsilon_{m_{i}}^{2})}{\epsilon_{m_{i}}}\exp\bigg(-4\rho_{\mu}\log (\psi T\epsilon_{m_{i}}^{2})\bigg)\\
& + \dfrac{8K\log (\psi T \epsilon_{g_{i}}^{2})}{\epsilon_{g_{i}}}\exp\bigg(- \dfrac{3\rho_v}{2} \bigg(\dfrac{\sigma_{i}^{2}+\sqrt{\rho_{v}\epsilon_{g_{i}}}+1}{3\sigma_{i}^{2}+\sqrt{\rho_v \epsilon_{g_{i}}}}\bigg) \log(\psi T\epsilon_{g_{i}}^{2}) \bigg)
\end{align*}
\end{small}

%\begin{small}
%\begin{align*}
%&\delta_{\max\lbrace m_{i}, g_{i}\rbrace}=\dfrac{(|B_{m_{i}}|.\ell_{m_{i}})}{T} + \dfrac{(|B_{g_{i}}|.\ell_{g_{i}})}{T}\\
%&\leq \dfrac{2K}{T}\exp\bigg(-4\rho\log (\psi T\epsilon_{m_{i}}^{2})\bigg).\dfrac{2\log (\psi T \epsilon_{m_{i}}^{2})}{\epsilon_{m_{i}}}
%\\ & + \dfrac{4K}{T}\exp\bigg(- \dfrac{3\rho_v}{2} \bigg(\dfrac{\sigma_{i}^{2}+\sqrt{\rho_{v}\epsilon_{g_{i}}}+1}{3\sigma_{i}^{2}+\sqrt{\rho_v \epsilon_{g_{i}}}}\bigg) \log(\psi T\epsilon_{g_{i}}^{2})\bigg).\dfrac{2\log (\psi T \epsilon_{g_{i}}^{2})}{\epsilon_{g_{i}}} \\
%& \leq \dfrac{4K\log (\psi T \epsilon_{m_{i}}^{2})}{T\epsilon_{m_{i}}}\exp\bigg(-4\rho\log (\psi T\epsilon_{m_{i}}^{2})\bigg)\\
%& + \dfrac{8K\log (\psi T \epsilon_{g_{i}}^{2})}{T\epsilon_{g_{i}}}\exp\bigg(- \dfrac{3\rho_v}{2} \bigg(\dfrac{\sigma_{i}^{2}+\sqrt{\rho_{v}\epsilon_{g_{i}}}+1}{3\sigma_{i}^{2}+\sqrt{\rho_v \epsilon_{g_{i}}}}\bigg) \log(\psi T\epsilon_{g_{i}}^{2}) \bigg)
%\end{align*}
%\end{small}

Now, in the $\max\lbrace m_{i}, g_{i}\rbrace$-th round $\sqrt{\rho_{\mu}\epsilon_{m_{i}}}\leq \frac{\Delta_{i}}{2}$ or $\sqrt{\rho_v\epsilon_{g_{i}}}\leq \frac{\Delta_{i}}{2}$. Hence,

\begin{small}
\begin{align*}
&\delta_{\max\lbrace m_{i},g_{i}\rbrace} \leq \dfrac{4K\log (\psi T \frac{\Delta_{i}^{4}}{16\rho_{\mu}^{2}})}{\frac{\Delta_{i}^{2}}{4\rho_{\mu}}}\exp\bigg(-4\rho_{\mu}\log (\psi T\frac{\Delta_{i}^{4}}{16\rho_{\mu}^{2}})\bigg)\\
& + \dfrac{8K\log (\psi T \frac{\Delta_{i}^{4}}{16\rho_{v}^{2}})}{\frac{\Delta_{i}^{2}}{4\rho_{v}}}\exp\bigg(- \dfrac{3\rho_v}{2} \bigg(\dfrac{\sigma_{i}^{2}+\frac{\Delta_{i}}{2}+1}{3\sigma_{i}^{2}+\frac{\Delta_{i}}{2}}\bigg) \log(\psi T\frac{\Delta_{i}^{4}}{16\rho_{v}^{2}}) \bigg)\\
%%%%%%%%%%%%%%%%%%%%%%%%%%%%%%%%%%%%%%%
&\leq 16 C_1\exp\bigg(-4\rho_{\mu}\log (\psi T\frac{\Delta_{i}^{4}}{16\rho_{\mu}^{2}})\bigg)\\
& + 32C_2\exp\bigg(- \dfrac{3\rho_v}{2} \bigg(\dfrac{2\sigma_{i}^{2}+\Delta_{i}+2}{6\sigma_{i}^{2}+\Delta_{i}}\bigg) \log(\psi T\frac{\Delta_{i}^{4}}{16\rho_{v}^{2}}) \bigg)\\
&\text{where $C_1=\frac{K\rho_{\mu}\log (\psi T \frac{\Delta_{i}^{4}}{16\rho_{\mu}^{2}})}{\Delta_{i}^{2}}$ and $C_2= \frac{K\rho_v\log (\psi T \frac{\Delta_{i}^{4}}{16\rho_{v}^{2}})}{\Delta_{i}^{2}}$}\\
%%%%%%%%%%%%%%%%%%%%%%%%%%%%%%%%%%%%%%%
&\leq 16 C_1\exp\bigg(-4\rho_{\mu}\log (\psi T\frac{\Delta_{i}^{4}}{16\rho^{2}})\bigg)
 + 32C_2\exp\bigg(- \dfrac{3\rho_v}{2} \log(\psi T\frac{\Delta_{i}^{4}}{16\rho_{v}^{2}}) \bigg)
\end{align*}
\end{small}

%Summing over all rounds $m=0,1,..,M$,
Now, putting the values of $\psi$, $\rho_{\mu}$, $\rho_v$ and taking $\Delta_{i}\geq\min_{i\in A}\Delta=\sqrt{\frac{K\log K}{T}}\geq \sqrt{\frac{e}{T}},\forall i\in A$( see \cite{auer2010ucb}), 

\begin{small}
\begin{align*}
& \delta_{\max\lbrace m_{i}, g_{i}\rbrace}= \bigg\lbrace 16 C_1\exp\bigg(-4\rho_{\mu}\log (\psi T\frac{\Delta_{i}^{4}}{16\rho_{\mu}^{2}})\bigg)\\
& + 32C_2\exp\bigg(- \frac{3\rho_v}{2} \log(\psi T\frac{\Delta_{i}^{4}}{16\rho_{v}^{2}}) \bigg) \bigg\rbrace\\
%%%%%%%%%%%%%%%%%%%%
&\leq \bigg\lbrace  \frac{2K\log ( T^2 \frac{4\Delta_{i}^{4}}{\log K})}{\Delta_{i}^{2}}\exp\bigg(-\frac{1}{2}\log ( T^2\frac{4\Delta_{i}^{4}}{\log K})\bigg)\\
& + \frac{32K\log ( T^2 \frac{9\Delta_{i}^{4}}{\log K})}{3\Delta_{i}^{2}}\exp\bigg(- \frac{1}{2} \log( T^2 \frac{9\Delta_{i}^{4}}{\log K}) \bigg) \bigg\rbrace\\
%%%%%%%%%%%%%%%%%%%%
&\leq \bigg\lbrace  \frac{4K\log ( T \frac{2\Delta_{i}^{2}}{\sqrt{\log K}})}{\Delta_{i}^{2}}\exp\bigg(-\log ( T\frac{2\Delta_{i}^{2}}{\sqrt{\log K}})\bigg)\\
& + \frac{64K\log ( T \frac{3\Delta_{i}^{2}}{\sqrt{\log K}})}{3\Delta_{i}^{2}}\exp\bigg(- \log( T \frac{3\Delta_{i}^{2}}{\sqrt{\log K}}) \bigg) \bigg\rbrace\\
%%%%%%%%%%%%%%%%%%%%
&\leq \bigg\lbrace  \frac{4KT\log ( \frac{2 K\log K}{\sqrt{\log K}})}{K\log K}\exp\bigg(-\log ( \frac{2K\log K}{\sqrt{\log K}})\bigg)\\
& + \frac{64TK\log (\frac{3 K\log K}{\sqrt{\log K}})}{3 K\log K}\exp\bigg(- \log( \frac{3 K\log K}{\sqrt{\log K}}) \bigg) \bigg\rbrace\\
%%%%%%%%%%%%%%%%%%%
&\leq \bigg\lbrace  \frac{2T\log (2 K\sqrt{\log K})}{K (\log K)^2}
 + \frac{22T\log ( K\sqrt{\log K})}{ K(\log K)^2}\bigg) \bigg\rbrace\\
\end{align*}
\end{small}

%Now, applying the bound from Fact $2$, we can show that for all rounds $m=0,1,2,...,M$ the probability of the sub-optimal arm $i$ being pulled is bounded above by,
%
%\begin{small}
%\begin{align*}
%&P_{i} = \sum_{m=0}^{M} \delta_{m_{i}}.Q_{m_{i}} + \sum_{g=0}^{M} \delta_{g_{i}}.Q_{g_{i}}\\
%& \leq \sum_{m=0}^{M} \bigg\lbrace 16C_1\exp\bigg(-4\rho\log (\psi T\frac{\Delta_{i}^{4}}{16\rho^{2}})\bigg). 2\gamma \exp(-\dfrac{c_{0}\sqrt{T}}{2^{m_{i}}.4}) \\ &+  32C_2\exp\bigg(- \dfrac{3\rho_v}{2} \bigg(\dfrac{2\sigma_{i}^{2}+\Delta_{i}+2}{6\sigma_{i}^{2}+\Delta_{i}}\bigg) \log(\psi T\frac{\Delta_{i}^{4}}{16\rho_{v}^{2}}) \bigg).\\
%& 2\gamma \exp(-\dfrac{c_{0}\sqrt{T}}{2^{g_{i}}.4})\bigg\rbrace\\
%%%%%%%%%%%%%%%%%%%
%& \leq M \gamma \bigg\lbrace 32C_1\exp\bigg(-4\rho\log (\psi T\frac{\Delta_{i}^{4}}{16\rho^{2}})-\dfrac{c_{0}\sqrt{T}}{\frac{4\rho}{\Delta_{i}^{2}}.4}\bigg)\\
%& + 64C_2\exp\bigg(- \dfrac{3\rho_v}{2} \bigg(\dfrac{2\sigma_{i}^{2}+\Delta_{i}+2}{6\sigma_{i}^{2}+\Delta_{i}}\bigg)\log(\psi T\frac{\Delta_{i}^{4}}{16\rho_{v}^{2}})\\
%& -\dfrac{c_{0}\sqrt{T}}{\frac{4\rho_v}{\Delta_{i}^{2}}.4}  \bigg) \bigg\rbrace
%\text{, as $\frac{1}{2^{m_{i}}}=\epsilon_{m_{i}}$ or $\frac{1}{2^{g_{i}}}=\epsilon_{g_{i}}$ }  \\
%%%%%%%%%%%%%%%%%%%
%& \leq \gamma\log_{2}\dfrac{T}{e}\bigg\lbrace \bigg(16C_1\exp\bigg(-4\rho\log (\psi T\frac{\Delta_{i}^{4}}{16\rho^{2}})-\dfrac{c_{0}\sqrt{T}}{16\rho\Delta_{i}^{-2}}\bigg)\\
%&+  32C_2\exp\bigg(- \dfrac{3\rho_v}{2} \bigg(\dfrac{2\sigma_{i}^{2}+\Delta_{i}+2}{6\sigma_{i}^{2}+\Delta_{i}}\bigg)\log(\psi T\frac{\Delta_{i}^{4}}{16\rho_{v}^{2}})\\
%& -\dfrac{c_{0}\sqrt{T}}{16\rho_v\Delta_{i}^{-2}}  \bigg)\bigg\rbrace 
%\text{ for $M=\big \lfloor \dfrac{1}{2}\log_{2} \dfrac{T}{e}\big\rfloor$}\\
%%%%%%%%%%%%%%%%%%%
%& \leq \gamma\log_{2}\dfrac{T}{e}\bigg\lbrace 16C_1\exp\bigg(-4\rho\log (\psi T\frac{\Delta_{i}^{4}}{16\rho^{2}})-\dfrac{c_{0}\sqrt{T}}{16\rho i\max_{i}\Delta_{i}^{-2}}\bigg) \\
%& +  32C_2\exp\bigg(- \dfrac{3\rho_v}{2} \bigg(\dfrac{2\sigma_{i}^{2}+\Delta_{i}+2}{6\sigma_{i}^{2}+\Delta_{i}}\bigg)\log(\psi T\frac{\Delta_{i}^{4}}{16\rho_{v}^{2}})\\
%& -\dfrac{c_{0}\sqrt{T}}{16\rho_vi\max_{i}\Delta_{i}^{-2}}  \bigg) \bigg\rbrace\\
%%%%%%%%%%%%%%%%%%
%& \leq \gamma\log_{2}\dfrac{T}{e}\bigg\lbrace 16C_1\exp\bigg(-4\rho\log (\psi T\frac{\Delta_{i}^{4}}{16\rho^{2}})-\dfrac{c_{0}\sqrt{T}}{16\rho H_{2}}\bigg) \\
%& + 32C_2\exp\bigg(- \dfrac{3\rho_v}{2} \bigg(\dfrac{2\sigma_{i}^{2}+\Delta_{i}+2}{6\sigma_{i}^{2}+\Delta_{i}}\bigg)\log(\psi T\frac{\Delta_{i}^{4}}{16\rho_{v}^{2}})\\
%& -\dfrac{c_{0}\sqrt{T}}{16\rho_v H_{2}}  \bigg)\bigg\rbrace\\
%%%%%%%%%%%%%%%%%%
%& \leq \exp\bigg(-4\rho\log (\psi T\frac{\Delta_{i}^{4}}{16\rho^{2}})-\dfrac{c_{0}\sqrt{T}}{16\rho H_{2}} + \log \big( 16\gamma C_1\log_{2}\dfrac{T}{e}\big) \bigg)\\
%& + \exp\bigg(- \dfrac{3\rho_v}{2} \bigg(\dfrac{2\sigma_{i}^{2}+\Delta_{i}+2}{6\sigma_{i}^{2}+\Delta_{i}}\bigg)\log(\psi T\frac{\Delta_{i}^{4}}{16\rho_{v}^{2}})\\
%& -\dfrac{c_{0}\sqrt{T}}{16\rho_v H_{2}} + \log\big( 32\gamma C_2\log_{2}\dfrac{T}{e} \big)  \bigg)
%\end{align*}
%\end{small}


Now, for the $i$-th arm we can bound it probability of error for any round $m$ by applying Chernoff-Hoeffding and Bernstein inequality ,
\begin{small}
\begin{align*}
& \Pb\lbrace \xi_1\rbrace  + \Pb\lbrace \xi_2 \rbrace \leq \Pb\lbrace |\hat{r}_i -r_i| \leq 2c_i \rbrace + \Pb\lbrace |\hat{r}_i -r_i| \leq 2s_i \rbrace\\ 
&\leq \Pb\lbrace |\hat{r}_i - r_i| \leq \frac{\Delta_i}{2} \rbrace + \Pb\lbrace |\hat{r}_i - r_i| \leq \frac{\Delta_i}{2} \rbrace \\
&\leq 2\exp( -\frac{\Delta_{i}^{2}}{2}\delta_{m_{i}} ) + 2exp(- \frac{\Delta_{i}^{2}}{2\sigma_{i}^{2}+ \frac{2}{3}\Delta_i}\delta_{g_{i}})
\end{align*}
\end{small}

Now, summing over all arms $K$ and over all rounds $m=0,1,2,..,M$,

\begin{small}
\begin{align*}
&\E[\Ls(T)] \leq \sum_{i=1}^{K}\sum_{m=0}^{M}\bigg\lbrace 2\exp\bigg( -\frac{\Delta_{i}^{2}}{2}.\frac{2T\log (2 K\sqrt{\log K})}{K (\log K)^2}\bigg)\\
& + 2\exp\bigg(- \frac{\Delta_{i}^{2}}{2\sigma_{i}^{2}+ \frac{2}{3}\Delta_i}.\frac{22T\log ( K\sqrt{\log K})}{ K(\log K)^2} \bigg)\bigg\rbrace\\
%%%%%%%%%%%%%%%
&\E[\Ls(T)] \leq K\log_2\frac{T}{e}\bigg\lbrace\exp\bigg( -\frac{1}{i\max_{i}\Delta_{i}^{-2}}.\frac{T\log (2 K\sqrt{\log K})}{K (\log K)^2}\bigg)\\
& + \exp\bigg(- \frac{3}{i\max_i(3\sigma_{i}^{2}+ \Delta_i)\Delta_{i}^{-2}}.\frac{11T\log ( K\sqrt{\log K})}{ K(\log K)^2} \bigg)\bigg\rbrace\\
%%%%%%%%%%%%%%%
&\E[\Ls(T)] \leq K\log_2\frac{T}{e}\bigg\lbrace\exp\bigg( -\frac{T\log (2 K\sqrt{\log K})}{H_2 K (\log K)^2}\bigg)\\
& + \exp\bigg(- \frac{33T\log ( K\sqrt{\log K})}{H_{2}^{\sigma} K(\log K)^2} \bigg)\bigg\rbrace\\
\end{align*}
\end{small}


%Therefore we can say that with probability $1-P_{i}$, all arms $i$ above $\frac{\Delta_{i}}{2}$ are accepted and all arms $i$ below $\frac{\Delta_{i}}{2}$ are rejected. 
%
%Hence, the simple regret of AugUCB is upper bounded by,
%
%\begin{small}
%\begin{align*}
%& SR_{AugUCB} \leq \sum_{i=1}^{K} \Delta_{i}. P_{i}\\
%& \leq \sum_{i=1}^{K} \Delta_{i}\bigg\lbrace \exp\bigg(-4\rho\log (\psi T\frac{\Delta_{i}^{4}}{16\rho^{2}})-\dfrac{c_{0}\sqrt{T}}{16\rho H_{2}}\\
%& + \log \big( 16\gamma C_1\log_{2}\dfrac{T}{e} \big) \bigg) + \exp\bigg(- \dfrac{3\rho_v}{2} \bigg(\dfrac{2\sigma_{i}^{2}+\Delta_{i}+2}{6\sigma_{i}^{2}+\Delta_{i}}\bigg)\log(\psi T\frac{\Delta_{i}^{4}}{16\rho_{v}^{2}})\\
%& -\dfrac{c_{0}\sqrt{T}}{16\rho_v H_{2}} + \log\big ( 32\gamma C_2\log_{2}\dfrac{T}{e} \big)  \bigg)\bigg\rbrace
%\end{align*}
%\end{small}

\end{proof}

%	Next we specialize the result of Theorem \ref{Result:Theorem:1} in Corollary \ref{Result:Corollary:1}.
%
%\subsection{Corollary 2}
%
%
%\begin{corollary}
%\label{Result:Corollary:1}
%For $c_{0}=\sqrt{T}$, $\psi=\frac{T}{\log (K)}$, $\rho_{\mu}=\frac{1}{8}$ and $\rho_v=\frac{2}{3}$, the simple regret of AugUCB is given by,
%\begin{small}
%\begin{align*}
%& SR_{AugUCB} \leq \sum_{i=1}^{K} \Delta_{i}\bigg\lbrace\exp\bigg(-\log ( 2T\frac{\Delta_{i}^{2}}{\sqrt{\log K}})-\dfrac{T}{2 H_{2}}\\
%& + \log \big( \dfrac{4\gamma K\log ( 2T \frac{\Delta_{i}^{2}}{\sqrt{\log K}})}{T\Delta_{i}^{2}}\log_{2}\dfrac{T}{e} \big) \bigg)\\
%& +  \exp\bigg(- \bigg(\dfrac{2\sigma_{i}^{2}+\Delta_{i}+2}{6\sigma_{i}^{2}+\Delta_{i}}\bigg)\log( 3T\frac{\Delta_{i}^{2}}{8\sqrt{\log K}}) -\dfrac{3T}{32 H_{2}}\\
%& + \log\big ( \dfrac{64\gamma K\log ( 3T \frac{\Delta_{i}^{2}}{8\sqrt{\log K}})}{3T\Delta_{i}^{2}}\log_{2}\dfrac{T}{e} \big)  \bigg)\bigg\rbrace
%\end{align*}
%\end{small}
%\end{corollary}
%
%\begin{proof}
%Putting $c_{0}=\sqrt{T}$, $\psi=\frac{T}{\log (K)}$, $\rho_{\mu}=\frac{1}{8}$ and $\rho_v=\frac{2}{3}$ in the result obtained in Theorem \ref{Result:Theorem:1}, we get
%\begin{small}
%\begin{align*}
%& SR_{AugUCB} \leq \sum_{i=1}^{K} \Delta_{i}\bigg\lbrace \exp\bigg(-4\rho\log (\psi T\frac{\Delta_{i}^{4}}{16\rho^{2}})-\dfrac{c_{0}\sqrt{T}}{16\rho H_{2}}\\
%& + \log \big( 16\gamma C_1\log_{2}\dfrac{T}{e} \big) \bigg) + \exp\bigg(- \dfrac{3\rho_v}{2} \bigg(\dfrac{2\sigma_{i}^{2}+\Delta_{i}+2}{6\sigma_{i}^{2}+\Delta_{i}}\bigg)\log(\psi T\frac{\Delta_{i}^{4}}{16\rho_{v}^{2}})\\
%& -\dfrac{c_{0}\sqrt{T}}{16\rho_v H_{2}} + \log\big ( 32\gamma C_2\log_{2}\dfrac{T}{e} \big)  \bigg)\bigg\rbrace\\
%%%%%%%%%%%%%%%%%%
%&\leq \sum_{i=1}^{K} \Delta_{i}\bigg\lbrace\exp\bigg(-\dfrac{1}{2}\log ( T^{2}\frac{4\Delta_{i}^{4}}{\log K})-\dfrac{T}{2 H_{2}}\\
%& + \log \big( \dfrac{2\gamma K\log ( T^{2} \frac{4\Delta_{i}^{4}}{\log K})}{T\Delta_{i}^{2}}\log_{2}\dfrac{T}{e} \big) \bigg)\\
%& + \exp\bigg(-  \bigg(\dfrac{2\sigma_{i}^{2}+\Delta_{i}+2}{6\sigma_{i}^{2}+\Delta_{i}}\bigg)\log( T^{2}\frac{\Delta_{i}^{4}}{16.\frac{4}{9}\log K}) -\dfrac{c_{0}\sqrt{T}}{16.\frac{2}{3} H_{2}}\\
%& + \log\big ( \dfrac{32\gamma\rho_v K\log ( T^{2} \frac{\Delta_{i}^{4}}{16.\frac{2}{9}\log K})}{T\Delta_{i}^{2}}\log_{2}\dfrac{T}{e} \big)  \bigg)\bigg\rbrace\\
%%%%%%%%%%%%%%%%%%
%&\leq \sum_{i=1}^{K} \Delta_{i}\bigg\lbrace\exp\bigg(-\log ( 2T\frac{\Delta_{i}^{2}}{\sqrt{\log K}})-\dfrac{T}{2 H_{2}}\\
%& + \log \big( \dfrac{4\gamma K\log ( 2T \frac{\Delta_{i}^{2}}{\sqrt{\log K}})}{T\Delta_{i}^{2}}\log_{2}\dfrac{T}{e} \big) \bigg)\\
%& +  \exp\bigg(- \bigg(\dfrac{2\sigma_{i}^{2}+\Delta_{i}+2}{6\sigma_{i}^{2}+\Delta_{i}}\bigg)\log( 3T\frac{\Delta_{i}^{2}}{8\sqrt{\log K}}) -\dfrac{3T}{32 H_{2}}\\
%& + \log\big ( \dfrac{64\gamma K\log ( 3T \frac{\Delta_{i}^{2}}{8\sqrt{\log K}})}{3T\Delta_{i}^{2}}\log_{2}\dfrac{T}{e} \big)  \bigg)\bigg\rbrace
%\end{align*} 
%\end{small}
%\end{proof}
%
%
\section{Numerical Experiments}
\label{expt}


	In this section we compare the empirical performance of AugUCB against APT, Uniform Allocation, CSAR, UCBE and UCBEV algorithm. The threshold $\tau$ is set at $0.5$ for all experiments. Each algorithm is run independently $500$ times for $10000$ timesteps and the output set of arms suggested by the algorithms at every timestep is recorded. The output is considered erroneous if the correct set of arms is not $i=\lbrace 6,7,8,9,10 \rbrace$ (true for all the experiments). The error percentage over $500$ runs is plotted against $10000$ timesteps. For the uniform allocation algorithm, for each $t=1,2,..,T$ the arms are sampled uniformly. For UCBE algorithm (\cite{audibert2009exploration}) which was built for single best arm identification, we modify it according to \cite{locatelli2016optimal} to suit the goal of finding arms above the threshold $\tau$. So the exploration parameter $a$ in UCBE is set to $a=\frac{T-K}{H_1}$. Again, for UCBEV, following \cite{gabillon2011multi}, we modify it such that the exploration parameter $a = \frac{T-2K}{H_{1}^{\sigma}}$. Then for each timestep $t=1,2,..,T$ we pull the arm that minimizes $\lbrace |\hat{r}_{i} -\tau|-\sqrt{\frac{a}{n_{i}}} \rbrace$, where $n_{i}$ is the number of times the arm $i$ is pulled till $t-1$ timestep and $a$ is set as mentioned above for UCBE and UCBEV respectively. Also, APT is run with $\epsilon=0.05$, which denotes the precision with which the algorithm suggests the best set of arms and we modify CSAR as per \cite{locatelli2016optimal} such that it behaves as a Successive Reject algorithm whereby it rejects the arm farthest from $\tau$ after each phase. For AugUCB we take $\rho_{\mu}=\frac{1}{8}$ and $\rho_v=\frac{1}{3}$ as in Theorem \ref{Result:Theorem:1}. 
%Also we run AugUCBM with arm elimination just by mean estimation and AugUCBV with arm elimination just by variance estimation. For AugUCBM, at every timestep we pull arm that minimizes $i\in\argmin_{j\in B_{m}}\bigg\lbrace |\hat{r}_{j} - \tau | - 2c_{j}\bigg\rbrace$ while for AugUCBV we pull arm that minimizes $i\in\argmin_{j\in B_{m}}\bigg\lbrace |\hat{r}_{j} - \tau | - 2s_{j}\bigg\rbrace$.

	The first experiment is conducted on a testbed of $100$ arms involving Gaussian reward distribution with expected rewards of the arms $r_{1:4}=0.2+(0:3)*0.05$, $r_{5}=0.45$, $r_{6}=0.55$, $r_{7:10}=0.65+(0:3)*0.05$ and $r_{11:100}$=0.4. The means of first $10$ arms are set as arithmetic progression. Variance is set as $\sigma_{1:5}^{2}=0.5$ and $\sigma_{6:10}^{2}=0.6$. Then $\sigma_{11:100}^{2}$ is chosen uniform randomly between $0.38-0.42$. The means in the testbed are chosen in such a way that any algorithm has to spend a significant amount of budget to explore all the arms and variance is chosen in such a way that the arms above $\tau$ have high variance whereas arms below $\tau$ have lower variance. The result is shown in Figure \ref{Fig:budgetExpt1}. In this experiment we see that UCBEV which has access to the problem complexity and is a variance-aware algorithm beats all other algorithm including UCBE which has access to the problem complexity but does not take into account the variance of the arms. AugUCB with the said parameters outperforms UCBE, APT and the other non variance-aware algorithms that we have considered. 	
%AugUCBM with just mean estimation performs worse than AugUCB or AugUCBV, which have a matching performance in this setup.
	
	\begin{figure}[!h]
    \centering
    \begin{tabular}{cc}
    \subfigure[0.32\textwidth][Experiment $1$: Experiment with Arithmetic Progression]
    {
    		\pgfplotsset{
		tick label style={font=\Huge},
		label style={font=\Huge},
		legend style={font=\Large},
		}
        \begin{tikzpicture}[scale=0.4]
      	\begin{axis}[
		xlabel={timestep},
		ylabel={Error Percentage},
		grid=major,
        %clip mode=individual,grid,grid style={gray!30},
        clip=true,
        %clip mode=individual,grid,grid style={gray!30},
  		legend style={at={(0.5,1.2)},anchor=north, legend columns=3} ]
      	% UCB
		\addplot table{results/budgetTestAP/APT12_comp_subsampled.txt};
		\addplot table{results/budgetTestAP/AugUCBV1_comp_subsampled.txt};
		%\addplot table{results/budgetTestAP/AugUCBV_1_13_comp_subsampled.txt};
		\addplot table{results/budgetTestAP/UCBEM1_comp_subsampled.txt};
		\addplot table{results/budgetTestAP/UCBEMV1_comp_subsampled.txt};
		\addplot table{results/budgetTestAP/SR1_comp_subsampled.txt};
		\addplot table{results/budgetTestAP/UA1_comp_subsampled.txt};
		%\addplot table{results/budgetTestAP/AugUCBM12_comp_subsampled.txt};
		%\addplot table{results/budgetTestAP/AugUCBV1_comp_subsampled.txt};
      	%\legend{APT,AugUCB,UCBE,UCBEV,CSAR,Unif Alloc,AugUCBM,AugUCBV}
      	\legend{APT,AugUCB,UCBE,UCBEV,CSAR,Unif Alloc}
      	\end{axis}
      	\end{tikzpicture}
  		\label{Fig:budgetExpt1}
    }
    &
    \subfigure[0.32\textwidth][Experiment $2$: Experiment with Geometric Progression ]
    {
    	\pgfplotsset{
		tick label style={font=\Huge},
		label style={font=\Huge},
		legend style={font=\Large},
		}
        \begin{tikzpicture}[scale=0.4]
        \begin{axis}[
		xlabel={timestep},
		ylabel={Error Percentage},
        %clip mode=individual,grid,grid style={gray!30},
		grid=major,
		clip=true,
  		legend style={at={(0.5,1.2)},anchor=north, legend columns=3} ]
        % UCB
		\addplot table{results/budgetTestGP/APT12_comp_subsampled.txt};
		\addplot table{results/budgetTestGP/AugUCBV1_comp_subsampled.txt};
		%\addplot table{results/budgetTestGP/AugUCBV_1_13_comp_subsampled.txt};
		\addplot table{results/budgetTestGP/UCBEM1_comp_subsampled.txt};
		\addplot table{results/budgetTestGP/UCBEMV1_comp_subsampled.txt};
		\addplot table{results/budgetTestGP/SR1_comp_subsampled.txt};
		\addplot table{results/budgetTestGP/UA1_comp_subsampled.txt};
		%\addplot table{results/budgetTestGP/AugUCBM12_comp_subsampled.txt};
		%\addplot table{results/budgetTestGP/AugUCBV1_comp_subsampled.txt};
        %\legend{APT,AugUCB,UCBE,UCBEV,CSAR,Unif Alloc,AugUCBM,AugUCBV}
        \legend{APT,AugUCB,UCBE,UCBEV,CSAR,Unif Alloc}
      	\end{axis}
      	\label{Fig:budgetExpt2}
        \end{tikzpicture}
    }
    %%%%%%%%%%
    % New row
    %%%%%%%%%%
    \\
    \subfigure[0.32\textwidth][Experiment $3$: Experiment with three Group Setting ]
    {
    		\pgfplotsset{
		tick label style={font=\Huge},
		label style={font=\Huge},
		legend style={font=\Large},
		}
        \begin{tikzpicture}[scale=0.4]
        \begin{axis}[
		xlabel={timestep},
		ylabel={Error Percentage},
        %clip mode=individual,grid,grid style={gray!30},
       	grid=major,
       	clip=true,
  		legend style={at={(0.5,1.2)},anchor=north, legend columns=3} ]
      	% UCB
		\addplot table{results/budgetTestGR1/APT1_comp_subsampled.txt};
		\addplot table{results/budgetTestGR1/AugUCB1_comp_subsampled.txt};
		\addplot table{results/budgetTestGR1/UCBEM1_comp_subsampled.txt};
		\addplot table{results/budgetTestGR1/UCBEMV1_comp_subsampled.txt};
		\addplot table{results/budgetTestGR1/SR1_comp_subsampled.txt};
		\addplot table{results/budgetTestGR1/UA1_comp_subsampled.txt};
        \legend{APT,AugUCB,UCBE,UCBEV,CSAR,Unif Alloc}
      	\end{axis}
      	\end{tikzpicture}
   		\label{Fig:budgetExpt3} 
    }
    &
    \subfigure[0.32\textwidth][Experiment $4$: Experiment with Geometric Progression (Bernoulli) ]
    {
    	\pgfplotsset{
		tick label style={font=\Huge},
		label style={font=\Huge},
		legend style={font=\Large},
		}
        \begin{tikzpicture}[scale=0.4]
        \begin{axis}[
		xlabel={timestep},
		ylabel={Error Percentage},
        %clip mode=individual,grid,grid style={gray!30},
		grid=major,
		clip=true,
  		legend style={at={(0.5,1.2)},anchor=north, legend columns=3} ]
        % UCB
		\addplot table{results/budgetTestGR2/APT1_comp_subsampled.txt};
		\addplot table{results/budgetTestGR2/AugUCBV1_comp_subsampled.txt};
		%\addplot table{results/budgetTestGP/AugUCBV_1_13_comp_subsampled.txt};
		\addplot table{results/budgetTestGR2/UCBEM1_comp_subsampled.txt};
		\addplot table{results/budgetTestGR2/UCBEMV1_comp_subsampled.txt};
		\addplot table{results/budgetTestGR2/SR1_comp_subsampled.txt};
		\addplot table{results/budgetTestGR2/UA1_comp_subsampled.txt};
		%\addplot table{results/budgetTestGP/AugUCBM12_comp_subsampled.txt};
		%\addplot table{results/budgetTestGP/AugUCBV1_comp_subsampled.txt};
        %\legend{APT,AugUCB,UCBE,UCBEV,CSAR,Unif Alloc,AugUCBM,AugUCBV}
        \legend{APT,AUgUCB,UCBE,UCBEV,CSAR,Unif Alloc}
        %\legend{APT,AugUCB,UCBE,UCBEV,CSAR,Unif Alloc}
      	\end{axis}
      	\label{Fig:budgetExpt4}
        \end{tikzpicture}
    }
%    \subfigure[0.32\textwidth][Experiment $4$: Experiment with Geometric Progression (Bernoulli) ]
%    {
%    	\pgfplotsset{
%		tick label style={font=\Huge},
%		label style={font=\Huge},
%		legend style={font=\Large},
%		}
%        \begin{tikzpicture}[scale=0.4]
%        \begin{axis}[
%		xlabel={timestep},
%		ylabel={Error Percentage},
%        %clip mode=individual,grid,grid style={gray!30},
%		grid=major,
%		clip=true,
%  		legend style={at={(0.5,1.2)},anchor=north, legend columns=3} ]
%        % UCB
%		\addplot table{results/budgetTestGPBern/APT1_comp_subsampled.txt};
%		\addplot table{results/budgetTestGP/AugUCBV1_comp_subsampled.txt};
%		%\addplot table{results/budgetTestGP/AugUCBV_1_13_comp_subsampled.txt};
%		\addplot table{results/budgetTestGPBern/UCBEM1_comp_subsampled.txt};
%		\addplot table{results/budgetTestGPBern/UCBEMV1_comp_subsampled.txt};
%		\addplot table{results/budgetTestGPBern/SR1_comp_subsampled.txt};
%		\addplot table{results/budgetTestGPBern/UA1_comp_subsampled.txt};
%		%\addplot table{results/budgetTestGP/AugUCBM12_comp_subsampled.txt};
%		%\addplot table{results/budgetTestGP/AugUCBV1_comp_subsampled.txt};
%        %\legend{APT,AugUCB,UCBE,UCBEV,CSAR,Unif Alloc,AugUCBM,AugUCBV}
%        \legend{APT,AUgUCB,UCBE,UCBEV,CSAR,Unif Alloc}
%        %\legend{APT,AugUCB,UCBE,UCBEV,CSAR,Unif Alloc}
%      	\end{axis}
%      	\label{Fig:budgetExpt4}
%        \end{tikzpicture}
%    }
    \end{tabular}
    \caption{Experiments with thresholding bandit}
    \label{fig:budgetExpt}
\end{figure}

	
	The second experiment is conducted on a testbed of $100$ arms with the means of first $10$ arms set as Geometric Progression. The testbed involves Gaussian reward distribution with expected rewards of the arms as $r_{1:4}=0.4-(0.2)^{1:4}$, $r_{5}=0.45$, $r_{6}=0.55$ and $r_{7:10}=0.6+(0.2)^{5-(1:4)}$. The variances of all the arms and $r_{11:100}$ are set in the same way as in experiment $1$. AugUCB, APT, CSAR, Uniform Allocation, UCBE and UCBEV with the same settings as experiment $1$ are run on this testbed. The result is shown in Figure \ref{Fig:budgetExpt2}. Here, again we see that AugUCB beats APT, UCBE and all the non-variance aware algorithms with only UCBEV beating AugUCB. 
	
	The third experiment is conducted on a testbed of $100$ arms with the means of first $10$ arms set  in three groups. The testbed involves Gaussian reward distribution with expected rewards of the arms as $r_{1:3}=0.1$, $r_{4:7}=\lbrace 0.35, 0.45, 0.55, 0.65\rbrace$ and $r_{8:10}=0.9$. The variances of all the arms and $r_{11:100}$ are set in the same way as in experiment $1$. AugUCB, APT, CSAR, Uniform Allocation, UCBE and UCBEV with the same settings as experiment $1$ are run on this testbed. The result is shown in Figure \ref{Fig:budgetExpt3}. Here, also we see that AugUCB beats APT, UCBE and all the non-variance aware algorithms with only UCBEV beating AugUCB. 
	
	The fourth experiment is the replication of the second experiment on a testbed of $100$ arms involving Bernoulli reward distribution with expected rewards and variances of the arms set in same way as in Experiment $2$. The result is shown in Figure \ref{Fig:budgetExpt4}. AugUCB is only beaten by UCBEV in this setup as well.

\vspace{-2mm}
\section{Conclusion}
\label{conclusion}
From a theoretical viewpoint we conclude the expected loss AugUCB is more than UCBEV (which has access to problem complexity). From the numerical experiments on settings with large number of arms with different mean and variance, we observed that AugUCB outperforms all the non-variance aware algorithms. It would be interesting future research to come up with an anytime version of AugUCB algorithm. This is also the first paper to apply elimination by variance estimation in the TBP problem by modifying UCB-Improved and CCB algrithms. 

% Acknowledgments---Will not appear in anonymized version
%\acks{We thank a bunch of people.}

\clearpage
\newpage
\bibliographystyle{named}
\bibliography{ijcai17}

%\clearpage
%\newpage
%\section{Appendix(We will comment this out later)}
%\label{appendix}
%\appendix
\begin{align*}
& H_{1}^{\sigma}=\sum_{i=1}^{K}\frac{\sigma_{i}+\sqrt{\sigma_{i}^{2}+(16/3)\Delta_{i}}}{\Delta_{i}^{2}}\\
& H_{2}^{\sigma}=\min_{i\in \mathcal{A}} i\tilde{\Delta}_{(i)}^{-2} \text{, where } \tilde{\Delta}_{i}^{-2}=\frac{\sigma_{i}+\sqrt{\sigma_{i}^{2}+(16/3)\Delta_{i}}}{\Delta_{i}^{2}}
%& H_{2}^{\sigma}=\min_{i\in \mathcal{A}} i\frac{12\sigma_{(i)}^{2} + \Delta_{(i)}}{12\Delta_{(i)}^{2}}
\end{align*}

We know that $\sigma_{i}\in [0,1], \forall i\in \mathcal{A}$ and $\Delta_{i}\in [0,1], \forall i\in \mathcal{A}$ and so $\sigma_{i}^{2} \leq \sigma_{i}$ and $\sqrt{\Delta_{i}} \geq \Delta_{i}$.

\begin{align*}
(3\Delta_{i}^{2}).\left(\frac{4\sigma_{i}^{2}+\Delta_{i}+4}{12\sigma_{i}^{2}+\Delta_{i}}\right) & >  \left(\frac{12\Delta_{i}^{2}}{12\sigma_{i}^{2}+\Delta_{i}}\right)\\
& > \left(\frac{12\Delta_{i}^{2}}{12\sigma_{i}^{2}+12\Delta_{i}}\right)\\
& > \left(\frac{\Delta_{i}^{2}}{\sigma_{i}+\Delta_{i}}\right)\\
%& > \left(\frac{\Delta_{i}^{2}}{\sigma_{i}+(\sigma_{i}^{2} + (16/3)\Delta_{i})}\right)\\
& > \left(\frac{\Delta_{i}^{2}}{\sigma_{i}+\sqrt{\sigma_{i}^{2} + (16/3)\Delta_{i}}}\right)\\
& > \left(\frac{1}{\min_{i}i\tilde{\Delta}_{i}^{2}}\right)\\
\end{align*}

Now, from \cite{audibert2010best} we know that,
\begin{align*}
\sum_{i=1}^{K}\tilde{\Delta}_{i}^{-2} = \tilde{\Delta}_{(2)}^{-2} + \sum_{i=2}^{K}\frac{1}{i}i\tilde{\Delta}_{(i)}^{-2} &\leq \bar{\log K}\min_{i}i\tilde{\Delta}_{(i)}^{-2}\\
& \leq \log(2K) H_{2}^{\sigma}, \text{ as $\bar{\log K} \leq \log(2K)$}
\end{align*}

So, $H_{2}^{\sigma} \leq H_{1}^{\sigma} \leq \log(2K) H_{2}^{\sigma}$



\end{document}

