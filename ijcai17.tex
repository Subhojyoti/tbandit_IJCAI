%%%% ijcai17.tex

\typeout{IJCAI-17 Instructions for Authors}

% These are the instructions for authors for IJCAI-17.
% They are the same as the ones for IJCAI-11 with superficical wording
%   changes only.

\documentclass{article}
% The file ijcai17.sty is the style file for IJCAI-17 (same as ijcai07.sty).
\usepackage{ijcai17}

% Use the postscript times font!
\usepackage{times}

\usepackage{macros}

\usepackage{latexsym} 


\title{Thresholding Bandits with Augmented UCB}
\author{Author names withheld}

\begin{document}

\maketitle

\begin{abstract}
In this paper we propose the Augmented-UCB (AugUCB) algorithm for a fixed-budget version of the thresholding bandit problem (TBP), where the objective is to identify a set of arms whose quality is above a threshold. A key feature of AugUCB is that it uses both mean and variance estimates to eliminate arms that have been sufficiently explored; to the best of our knowledge this is the first algorithm to employ such an approach for the considered TBP.  Theoretically, we obtain an upper bound on the loss (probability of mis-classification) incurred by AugUCB. Although UCBEV in literature provides a better guarantee, it is important to emphasize that UCBEV has access to problem complexity (whose computation requires arms' mean and variances), and hence is not realistic in practice; this is in contrast to AugUCB whose implementation does not require any such complexity inputs. We conduct extensive simulation experiments to validate the performance of AugUCB. Through our simulation work, we establish that AugUCB, owing to its utilization of variance estimates, performs significantly better than the state-of-the-art APT, CSAR and other non variance-based algorithms.


%%%%%%%
%In this paper we propose the Augmented-UCB (AugUCB) algorithm for the fixed-budget setting of a specific combinatorial, pure-exploration, stochastic multi-armed bandit setup called the thresholding bandit problem. Our algorithm is based on arm elimination, employing both mean and variance estimates and to our knowledge this is the first algorithm to employ such an approach in this setting. Through  simulation experiments we establish that our algorithm, owing to its utilization of variance estimates in arm elimination, performs significantly better than the state-of-the-art APT and CSAR algorithms, particularly when a large number of arms with different means and variances are involved. Theoretically, our algorithm is not comparable with APT or CSAR which use just mean estimation. AugUCB provides a weaker guarantee (in terms of an upper bound on the expected loss) than UCBEV, a variant of GapE-V \cite{gabillon2011multi} algorithm, modified for thresholding bandit problem. However, UCBEV requires access to the problem complexity (which is not realistic), while AugUCB requires no such complexity parameters as input. 
%%%%%%%%%%%



%is an anytime arm elimination variance-aware algorithm, and is the first of its kind which employs arm elimination 
%In earlier works, it was seen that algorithms using variance estimates outperform other competing algorithms.

%We propose the  Augmented-UCB (AugUCB) algorithm for the thresholding bandit problem, which is an instance of the combinatorial fixed-budget pure-exploration stochastic multi-armed bandit setup.
%the latter ones since it is a variance-aware algorithm. In the considered test cases comprising large number of arms, our algorithm has consistently performed much better than the state-of-the art APT and CSAR algorithms.  
\end{abstract}

%\begin{keywords}
%Multi-Armed Bandit, Regret, Exploration-exploitation, UCB
%\end{keywords}

\section{Introduction}
\label{intro}
In this paper we study the fixed-budget setting of a specific combinatorial pure-exploration problem, called the thresholding bandit problem (TBP), in the context of stochastic multi-armed bandit (MAB) setting. MAB problems are instances of the classic sequential decision-making scenario; specifically an MAB problem comprises of a learner and a collection of actions (or arms), denoted $\mathcal{A}$. In each trial the learner plays (or pulls) an arm $i\in\mathcal{A}$ which yields independent and identically distributed (i.i.d.) reward samples from a distribution (corresponding to arm $i$), whose expectation is denoted by $r_i$. 
%whose rewards are  samples from the distribution specific to the arm $i\in A$ and whose expected reward is denoted by $r_{i},\forall i\in A$. 
The learner's objective is to identify an arm corresponding to the maximum expected reward, denoted $r^{*}$. Thus, at each time-step the learner 
%selects an arm $i$ and hence
is faced with the \emph{exploration vs.\ exploitation dilemma}, where it can pull an arm which has yielded the highest mean reward (denoted $\hat{r}_{i}$) thus far (\emph{exploitation}) or continue to explore other arms with the prospect of finding a better arm 
%superior performance 
whose performance is yet not observed sufficiently (\emph{exploration}).

%In the stochastic multi-armed bandit setting a learning agent is required to choose from a set of decisions or arms at every round. The agent is then presented with a reward for that round, which is an independent draw from a stationary distribution specific to the arm selected. The agent, however, does not know the mean of the distributions associated with each arm, denoted by $r_{i}$, including the optimal arm which will give it the best reward, denoted by $r^{*}$. The agent attempts to make arm choices that will maximize some performance measure by keeping track of the reward that has been gathered from previous selections of the arm, for each arm. This is called the estimated mean reward of an arm denoted by $\hat{r}_{i}$. The bandit problem can be conceptualized as a sequential decision making process where the agent is at each round presented with an \emph{exploration-exploitation dilemma}. The agent could pull the arm which has the highest observed mean reward till now (exploitation) or to explore other arms, with the prospect of finding superior performance which was previously unobserved (exploration). 

%
%	Formally, let $r_i$, $i=1,\ldots,K$ denote the mean rewards of the $K$ arms and $r^* = \max_i r_i$ the optimal mean reward. The objective in some of the stochastic bandit problem is to minimize the cumulative regret, which is defined as follows:
%\begin{align*}
%R_{T}=r^{*}T - \sum_{i\in A} r_{i}N_{i}(T),
%\end{align*}
%where $T$ is the number of rounds, $N_{i}(T)=\sum_{m=1}^T I(I_m=i)$ is the number of times the algorithm chose arm $i$ up to round $T$.
%The expected regret of an algorithm after $T$ rounds can be written as,
%
%\begin{align*}
%\E[R_{T}]= \sum_{i=1}^K \E[N_i(T)] \Delta_i,
%\end{align*}
%where $\Delta_{i}=r^{*}-r_{i}$ denotes the gap between the means of the optimal arm and of the $i$-th arm. 

Pure-exploration problems are unlike their traditional (exploration vs.\ exploitation) counterparts where the  objective is to minimize the cumulative regret, which is the total loss incurred by the learner for not playing the optimal arm throughout the time horizon $T$. Instead, in the pure exploration setup the learning algorithm is provided with a threshold $\tau$, and the objective, after exploring for $T$ rounds, is to  output all arms $i$ whose $r_{i}$ is above $\tau$. Thus, the learning algorithm, until  time $T$, can invest entirely on exploring the arms  without being concerned about the loss incurred while exploring. The thresholding bandit problem is different from the threshold bandit setup mentioned in \cite{abernethy2016threshold} where the reward on each timestep depends on a threshold value and the learner receives the reward only if the reward is above the threshold.


%This is a specific instance of combinatorial pure exploration where the learning algorithm can explore as much as possible given a fixed horizon $T$ and not be concerned with the usual exploration-exploitation dilemma. 

Formally, the problem we consider is the following. First, we define the set $S_{\tau}=\lbrace i\in \mathcal{A}: r_{i}\geq \tau \rbrace$. Note that, $S_\tau$ is the set of all arms whose reward mean is greater than $\tau$. Let 
$S_\tau^c$ % =\mathcal{A}\backslash S_\tau$
 denote the complement of $S_\tau$, i.e.,  $S_{\tau}^{c}=\lbrace i\in \mathcal{A}: r_{i} < \tau \rbrace$. Next, let $\hat{S}_{\tau}=\hat{S}_{\tau}(T)\subseteq \mathcal{A}$ denote the recommendation of the learning algorithm after $T$ time units of exploration, while $\hat{S}_{\tau}^c$ denotes its complement.
%  Also we define $\hat{S}_{\tau}=\hat{S}_{\tau}(T)\subset \mathcal{A}$ and its complementary set $\hat{S}_{\tau}^{C}$ as the recommendation of the learning algorithm after $T$ rounds. 
% Given such sets exists, 
The performance of the learning agent is measured by the accuracy with which it can classify the arms into $S_{\tau}$ and $S_{\tau}^{c}$ after time horizon $T$. Equivalently, using $\mathbb{I}(E)$ to denote the indicator of an event $E$, the \emph{loss} $\mathcal{L}(T)$ is defined as
\begin{align*}
\Ls (T) = \mathbb{I}\big(\lbrace S_{\tau}\cap \hat{S}_{\tau}^{c}\neq \emptyset\rbrace    \cup    \lbrace\hat{S}_{\tau}\cap S_{\tau}^{c}\neq \emptyset\rbrace\big).
\end{align*}			
Finally, the goal of the learning agent is to minimize the expected loss:
% So, the expected loss after $T$ rounds is,
\begin{align*}
\E[\Ls(T)] = \Pb\big(\lbrace S_{\tau}\cap \hat{S}_{\tau}^{c} \neq \emptyset \rbrace  \cup   \lbrace \hat{S}_{\tau}\cap S_{\tau}^{c} \neq \emptyset\rbrace\big).
\end{align*}
Note that the expected loss is simply the \emph{probability of error}, that occurs either if a good arm is rejected or a bad arm is accepted as a good one.
% (represented by the events $\lbrace S_{\tau}\cap \hat{S}_{\tau}^{c} \neq \emptyset \rbrace$ and $\lbrace \hat{S}_{\tau}\cap S_{\tau}^{c} \neq \emptyset\rbrace$, respectively).

%which we can say is the probability of making mistake, that is whether the learning agent at the end of round $T$ rejects arms from $S_{\tau}$ or accepts arms from $S_{\tau}^{C}$ in its final recommendation. 

%Also, we are looking at an anytime algorithm, so the knowledge of $T$ may not be known to the learner.





%\subsection{Motivation}
%\label{motivation}
The thresholding bandit problem (TBP) has several relevant industrial applications. In some cases the TBP problem is more relevant than the variants of TopM problem (identifying the best $M$ arms from $K$ given arms).

%\begin{enumerate}
%\item \emph{Online Shop Domain (\cite{ghavamzadeh2015bayesian}):} In the online shop domain, a retailer aims to maximize profit by sequentially suggesting set of products to online shopping customers. In this scenario, at every timestep,  the retailer displays an item to a customer from a pool of items which has the highest probability of being selected by the customer. The episode ends when the customer selects or does not select a product (which will be considered as a loss to the retailer) and the process is again repeated till a pre-specified number of times with the retailer gathering valuable information regarding the customer from this behaviour and modifying its policy to display the next item.
%\item \emph{Medical Treatment Design (\cite{thompson1933likelihood}):} Here at every timestep, the agent chooses to administer one out of several treatments sequentially on a patient. Here, the episode ends when the patient responds well or does not respond well to the treatment whereby the agent modifies its policy for the next suggestion.
%\item \emph{Financial Portfolio Management:} In financial portfolio management MAB model can be used. Here, the agent is faced with the choice of selecting the most profitable stock option out of several stock options. The simplest strategy where we can employ a bandit model is this; at the start of every trading session the agent suggests a stock to purchase worth Re $1$, while at the closing of the trading session it sells off the stock to witness its value after a day's trading. The  profit recorded is treated as the reward revealed by the environment and the agent modifies its policy for the next day.

	1. \emph{Product Selection:} A company wants to introduce a new product in market and there is a clear separation of the test phase from the commercialization phase. In this case the company tries to minimize the loss it might incur in the commercialization phase by testing as much as possible in the test phase. So from the several variants of the product that are in the test phase the learning agent must suggest the product variant(s) that are above a particular threshold $\tau$ at the end of the test phase that have the highest probability of minimizing loss in the commercialization phase. A similar problem has been discussed for single best product variant identification without threshold in \cite{bubeck2011pure}. 

	2. \emph{Mobile Phone Channel Allocation:} Another similar problem as above concerns channel allocation for mobile phone communications (\cite{audibert2009exploration}). Here there is a clear separation between the allocation phase and communication phase whereby in the allocation phase a learning algorithm has to explore as many channels as possible to suggest the best possible set of channel(s) that are above a particular threshold $\tau$. The threshold depends on the subscription level of the customer. With higher subscription the customer is allowed better channel(s) with the $\tau$ set high. Each evaluation of a channel is noisy and the learning algorithm must come up with the best possible suggestion within a very small  number of attempts.

	3. \emph{Anomaly Detection and Classification:} Thresholding bandit can also be used for anomaly detection and classification where we define a cutoff level $\tau$ and for any samples above this cutoff gets classified as an anomaly. For further reading we point the reader to section 3 of \cite{locatelli2016optimal}.
%\end{enumerate}

%In all the above examples the MAB model performs well mainly because all of them suffer from \textit{exploration-exploitation dilemma}. This is characterized by action-selection choice faced by the agent where it must decide whether to stay with the action yielding highest reward till now or to explore newer actions which might be more profitable in the long run. MAB's are suited for such scenarios because 
%\begin{enumerate}
%\item They are easy to implement.
%\item The switch between exploration and exploitation is more well defined theoretically.
%\item They perform well empirically.
%\end{enumerate}
\subsection{Related Work}
\label{prevRes}
A significant amount of work has been done on the stochastic MAB setting regarding minimizing cumulative regret with a single optimal arm. For a survey of such works we refer the reader to \cite{bubeck2012regret}. Starting from the early work of \cite{thompson1933likelihood}, \cite{robbins1952some} to \cite{lai1985asymptotically} which gives us an asymptotic lower bound on the cumulative regret we come to the UCB1 algorithm in  \cite{auer2002finite}. Subsequent works such as \cite{audibert2009minimax} and \cite{auer2010ucb} have shown better upper bounds on the cumulative regret. In \cite{auer2010ucb} they propose the UCB-Improved algorithm which is round-based algorithm\footnote{An algorithm is \textit{round-based} if it pulls all the arms equal number of times in each round and then proceeds to eliminate one or more arms that it identifies to be sub-optimal.}. Of special mention is the \cite{audibert2009exploration} where they introduce variance-aware algorithm UCB-V and show that algorithms that take into account variance estimation along with mean estimation tends to perform better than algorithms than solely focuses on mean estimation such as UCB1.


%An early work involving a bandit setup is \cite{thompson1933likelihood}, where the author deals with the problem of choosing between two treatments to administer on patients who come in sequentially. Following the seminal work of  \cite{robbins1952some}, bandit algorithms have been extensively studied in a variety of applications. From a theoretical standpoint, an asymptotic lower bound for the regret was established in \cite{lai1985asymptotically}. Several other works such as \cite{auer2002finite},  \cite{audibert2009minimax} and \cite{auer2010ucb} have shown results for minimizing cumulative regret in stochastic bandit setup whereas works such as \cite{auer2002nonstochastic} have concentrated on adversarial bandit setup.
	
	In the pure exploration setup, a significant amount of research has been done on finding the best arm(s) from a set of arms. The pure exploration setup has been explored in mainly two settings:-
	
	\emph{1. Fixed Budget setting:} In this setting the learning algorithm has to suggest the best arm(s) within a fixed number of attempts that is given as an input. The objective here is to maximize the probability of returning the best arm(s).  We study this setting in our paper. In \cite{audibert2010best} the authors come up with the algorithm UCBE and Successive Reject(SR) with simple regret guarantees to find the single best arm. The relationship between cumulative regret and simple regret is proved in \cite{bubeck2011pure} where the authors prove that minimizing the simple regret necessarily results in maximizing the cumulative regret. In the combinatorial fixed budget setup \cite{gabillon2011multi} come up with Gap-E and Gap-EV algorithm which suggests the best $m$ (given as input) arms at the end of the budget with high probability. Similarly, \cite{bubeck2013multiple} comes up with the algorithm Successive Accept Reject(SAR) which is an extension of the SR algorithm. SAR is a round based algorithm whereby at the end of a round an arm is either accepted or rejected based on certain conditions till the required top $m$ arms are suggested at the end of the budget with high probability. A similar combinatorial setup was also explored in \cite{chen2014combinatorial} where the authors come up with an algorithm, called Combinatorial Successive Accept Reject (CSAR) which is similar to SAR but with a more general setup. 

	\emph{2. Fixed Confidence setting:} In this setting the the learning algorithm has to suggest the best arm(s) with a fixed confidence (given as input) with as less number of attempts as possible. The single best arm identification has been handled in \cite{even2006action} while in the combinatorial setup \cite{kalyanakrishnan2012pac} have suggested the LUCB algorithm which on termination returns $m$ arms which are atleast $\epsilon$ close to the true top $m$ arms with $1-\delta$ probability. For a survey of this setup we refer the reader to \cite{jamieson2014best}. 

	Apart from these two settings some unified approach has also been suggested in \cite{gabillon2012best} which proposes the algorithms UGapEb and UGapEc which can work in both the above two settings. The thresholding bandit problem is a specific instance of the pure exploration setup of \cite{chen2014combinatorial}. In the latest work in \cite{locatelli2016optimal} the algorithm Anytime Parameter-Free Thresholding (APT) algorithm comes up with a better anytime guarantee than CSAR for the thresholding bandit problem.	
	
	
%	\emph{1. Fixed Budget setting:} In this setting the learning algorithm has to suggest the best arm(s) within a fixed number of attempts that is given as an input. The objective here is to maximize the probability of returning the best arm(s). One of the foremost papers to deal with single best arm identification is \cite{audibert2009exploration} where the authors come up with the algorithm UCBE and Successive Reject(SR) with simple regret guarantees. The relationship between cumulative regret and simple regret is proved in \cite{bubeck2011pure} where the authors prove that minimizing the simple regret necessarily results in maximizing the cumulative regret. In the combinatorial fixed budget setup \cite{gabillon2011multi} come up with Gap-E and Gap-EV algorithm which suggests the best $m$ (given as input) arms at the end of the budget with high probability. Similarly, \cite{bubeck2013multiple} comes up with the algorithm Successive Accept Reject(SAR) which is an extension of the SR algorithm. SAR is a round based algorithm whereby at the end of round an arm is either accepted or rejected based on certain conditions till the required top $m$ arms are suggested at the end of the budget with high probability. 
%
%	\emph{2 Fixed Confidence setting:} In this setting the the learning algorithm has to suggest the best arm(s) with a fixed (given as input) confidence with as less number of attempts as possible. The single best arm identification has been handled in \cite{even2006action} where they come up with an algorithm called Successive Elimination (SE) which comes up with an arm that is $\epsilon$ close to the optimal arm. In the combinatorial setup recently \cite{kalyanakrishnan2012pac} have suggested the LUCB algorithm which on termination returns $m$ arms which are atleast $\epsilon$ close to the true top $m$ arms with $1-\delta$ probability.
%
%	Apart from these two settings some unified approach has also been suggested in \cite{gabillon2012best} which proposes the algorithms UGapEb and UGapEc which can work in both the above two settings. A similar combinatorial setup was also explored in \cite{chen2014combinatorial} where the authors come up with more similarities and dissimilarities between these two settings in a more general setup. In their work, the learning algorithm, called Combinatorial Successive Accept Reject (CSAR) is similar to SAR with a more general setup. The thresholding bandit problem is a specific instance of the pure exploration setup of \cite{chen2014combinatorial}. In the latest work in \cite{locatelli2016optimal} the algorithm Anytime Parameter-Free Thresholding (APT) algorithm comes up with a better anytime guarantee than CSAR for the thresholding bandit problem.

\subsection{Our Contribution}
\label{contribution}
In this paper we propose the Algorithm AugUCB which is an anytime action elimination algorithm suited for the TBP problem. It combines the approach of UCB-Improved, CCB (\cite{liu2016modification}) and APT algorithm. Our algorithm is also a variance-aware algorithm which takes into account the empirical variance of the arms. We also address an open problem raised in \cite{auer2010ucb} of coming up with an algorithm that can eliminate arms based on variance. Both CSAR and APT are not variance-aware algorithms. Theoretically our result is more closer to CSAR and is weaker than APT. But empirically we show that for a large action set when the variance of the arms lying above $\tau$ are high, our algorithm performs better than all other algorithms, except the algorithm UCBEV which has access to the underlying problem complexity and also is a variance aware algorithm. Irrespective of this case AugUCB also employs elimination of arms based on mean estimation only and is the first such algorithm which uses elimination by both mean and variance estimation simultaneously. AugUCB requires three input parameters and the exact choices for these parameters are derived in Theorem \ref{Result:Theorem:1}. Also, unlike SAR or CSAR, AugUCB does not have explicit accept or reject set rather the arm elimination conditions simply removes arm(s) if it is sufficiently sure that the mean of the arms are very high or very low about the threshold based on mean and variance estimation thereby re-allocating the remaining budget among the surviving arms. This although is a tactic similar to SAR or CSAR, but here at any round, an arbitrary number of arms can be accepted or rejected thereby improving upon SAR and CSAR which accepts/rejects one arm in every round. At the end of the budget $T$ the algorithm outputs all the arms whose $\hat{r}_{i}$ is above the threshold $\tau$ thereby making this an anytime algorithm whereby we need not finish every round. 

\begin{table}
\caption{Expected Loss for different bandit algorithms}
\label{tab:regret-bds}
\begin{center}
\begin{tabular}{|p{1.3cm}|p{6.33cm}|}
\toprule
Algorithm  & Upper Bound on Expected Loss \\
\midrule
APT         &$\exp(-\frac{T}{64H_1}+2\log((\log(T)+1)K))$ \\\midrule
CSAR		&$K^2\exp(-\frac{T-K}{72\log(K)H_2})$ \\\midrule
AugUCB      &
$\exp\left( -\frac{T\log ( 2K\log K)}{ 16 H_2 K\log K} + \log\left(K\left(\log_2\frac{T}{e}\right)\right)\right)$\newline
 $+$\newline
 $ \exp\left(- \frac{T\log ( \frac{3}{4} K\log K)}{64H_{2}^{\sigma} K(\log K)}
 + 2\log\left(K\left(\log_2\frac{T}{e}\right)\right) \right) $
%$\exp\big( -\frac{T\log (2 K\sqrt{\log K})}{2 H_2 K (\log K)^{3/2}} + \log(K(\log_2\frac{T}{e}+1))\big)
%+ \exp \big(- \frac{5T\log ( K\sqrt{\log K})}{H_{2}^{\sigma} K(\log K)^{3/2}} $
%\newline$ + \log(K\big(\log_2\frac{T}{e}+1))\big)$ 
\\\bottomrule
\end{tabular}
\end{center}
\end{table}

  
%
%\section{Notation Used and Assumptions}
%\label{notation}
%\textbf{Notations and assumptions:} $\mathcal{A}$ denotes the set of arms, and $|\mathcal{A}|=K$ is the number of arms in $\mathcal{A}$. 
%Arms generic arm is indexed by $i,j\in\mathcal{A}$. 
For arm $i\in\mathcal{A}$, we use $r_{i}$ to denote the true mean of the distribution from which the rewards are sampled, while $\hat{r}_{i}(t)$ denotes the estimated mean at time $t$. Formally, using $n_i(t)$ to denote the number of times arm $i$ has been pulled until time $t$, we have $\hat{r}_{i}(t)=\frac{1}{n_{i}(t)}\sum_{z=1}^{n_i(t)} X_{i,z}$, where $X_{i,z}$ is the reward sample received when arm $i$ is pulled for the $z$-th time. %
Similarly, we use $\sigma_{i}^{2}$ to denote the true variance of the reward distribution corresponding to arm $i$, while $\hat{v}_{i}(t)$ is the estimated variance, i.e., $\hat{v}_{i}(t)=\frac{1}{n_i(t)}\sum_{z=1}^{n_{i}(t)}(X_{i,z}-\hat{r}_{i})^{2}$. Whenever there is no ambiguity about the underlaying  time index $t$, for simplicity we neglect $t$ from the notations and simply use  $\hat{r}_i, \hat{v}_i,$ and $n_i, $ to denote the respective quantities.  Let  $\Delta_{i}=|\tau-r_{i}|$ denote the distance of the true mean from the threshold $\tau$.



%The average estimated payoff for any arm is denoted by  whereas the true mean of the distribution  is denoted by . The optimal arm is denoted by $*$. The '*' superscript is used to denote anything related to optimal arm. 

 % $n_{i}$ denotes the number of times the arm $i$ has been pulled. $\psi $ denotes the exploration regulatory factor and $\rho_\mu ,\rho_v$ as arm elimination parameters. $\hat{V}_{i}=\frac{1}{n_i}\sum_{t=1}^{n_{i}}(x_{i,t}-r_{i})^{2}$ denotes the empirical variance and $x_{i,t}$ is the reward obtained at timestep t for arm $i$. Also   denotes the true variance of the arm $i$. 
 
Finally, we assume that all the reward distributions 
%from which rewards are sampled are identical and independent 
are $1$-sub-Gaussian (note that,  $1$-sub-Gaussian includes Gaussian distributions with variance less than $1$, distributions supported on an interval of length less than 2, etc). Further, the rewards are assumed to take values in the interval $[0,1]$.
%. In our case, we  assume that all rewards are .

%Also we define $\Delta_{i}=r^{*} - r_{i}$ and $\hat{\Delta}_{i}=\hat{r}^{*} - \hat{r}_{i}$. In all cases $\min_{i\in A}{\Delta_{i}}$ is denoted by $\Delta$.
%and the optimal arm is denoted by $*$. The '*' superscript is used to denote anything related to optimal arm
%\paragraph*{}It is assumed that the distribution from which rewards are sampled are identical and independent sub-Gaussian distributions. Throughout the paper, we assume that the distributions $v_{i}$ are sub-Gaussian that is $\int e^{\lambda(x - r)} v_{i} (dx) ≤ e^{\lambda /2}, \forall \lambda \in \mathbb{R}$. Note that these include Gaussian distributions with variance less than 1 and distributions supported on an interval of length less than 2. All the experiments are also conducted with sub-Gaussians having variance as 1. Together with a Chernoff-Hoeffding bound, the sub-Gaussian assumption implies the following concentration inequality, valid for any
%$u > 0$,
%\newline
%\hspace*{8em}$\mathbb{P}\lbrace \hat{r}_{i} - r^{*} > u\rbrace \leq exp(-\dfrac{su^{2}}{2}) $
%\newline
%where s is the number of pulls of $a_{i}$. T is the horizon over which this entire algorithm runs.  $A^{'}$ at any round $m$ denotes the arms still not eliminated.
%\paragraph*{}The paper is organized as follows. We first present the algorithm in section 6. We then provide the proofs of Phase1 which includes regret-bound calculation and arm deletion conditions in section 7. In section 8, we provide proofs for Phase2 along with early stopping conditions. Section 9 deals with regret bound and then we provide error probability and error bounds in section 10. Experimental results are provided in section 11, and we conclude in section 12.
%
%
\vspace*{-1em}
\section{Augmented-UCB Algorithm}
\label{algorithm}
%The algorithm is presented below:-

%%%%%%%%%%%%%%%% alg-custom-block %%%%%%%%%%%%
\algblock{ArmElim}{EndArmElim}
\algnewcommand\algorithmicArmElim{\textbf{\em Arm Elimination by Mean Estimation}}
 \algnewcommand\algorithmicendArmElim{}
\algrenewtext{ArmElim}[1]{\algorithmicArmElim\ #1}
\algrenewtext{EndArmElim}{\algorithmicendArmElim}
\algtext*{EndArmElim}

\algblock{ArmElimV}{EndArmElimV}
\algnewcommand\algorithmicArmElimV{\textbf{\em Arm Elimination by Mean and Variance Estimation}}
 \algnewcommand\algorithmicendArmElimV{}
\algrenewtext{ArmElimV}[1]{\algorithmicArmElimV\ #1}
\algrenewtext{EndArmElimV}{\algorithmicendArmElimV}
\algtext*{EndArmElimV}

\algblock{ResetParam}{EndResetParam}
\algnewcommand\algorithmicResetParam{\textbf{\em Reset Parameters}}
 \algnewcommand\algorithmicendResetParam{}
\algrenewtext{ResetParam}[1]{\algorithmicResetParam\ #1}
\algrenewtext{EndResetParam}{\algorithmicendResetParam}
\algtext*{EndResetParam}

%%%%%%%%%%%%%%%%%%%%%%%%%%%%%%%%%%%%%%%%%%%%%%%%%%%%%%%%%%%%%%%%%%%%%%%%
%Old Algorithm
%%%%%%%%%%%%%%%%%%%%%%%%%%%%%%%%%%%%%%%%%%%%%%%%%%%%%%%%%%%%%%%%%%%%%%%%

%\begin{algorithm}[th!]
%\caption{AugmentedUCB}
%\label{alg:augucb}
%\begin{algorithmic}
%\State {\bf Input:} Time horizon $T$, exploration parameters $\rho_{\mu}$, $\rho_v$ and $\psi$, threshold $\tau$.
%\State {\bf Initialization:} Set $B_{0}:=A$, $M=\left\lfloor \frac{1}{2}\log_{2} \frac{T}{e}\right\rfloor $, $m:=0$, $\epsilon_{0}:=1$, $\ell_{0}=\left\lceil \frac{2\psi\log( T\epsilon_{0}^{2})}{\epsilon_{0}} \right\rceil$ and $N_{0}=K\ell_{0} $.
%\State Pull each arm once
%\State \For{$t=K+1,..,T$}
%\State Pull arm $i\in\argmin_{j\in B_{m}}\bigg\lbrace |\hat{r}_{j} - \tau | - 2s_{j}\bigg\rbrace$
%\State $t:=t+1$ 
%\ArmElim
%\State For each arm $i \in B_{m}$, remove arm ${i}$ from $B_{m}$ if
%\begin{align*}
%\hat{r}_{i} + c_i  < \tau - c_i \mbox{ or } \hat{r}_{i} - c_i  > \tau + c_i \\
%\text{where $c_i=\sqrt{\frac{\rho_{\mu}\psi\log{( T\epsilon_{m}^{2})}}{2 n_{i}}}$}
%\end{align*}
%\EndArmElim
%\ArmElimV
%\State For each arm $i \in B_{m}$, remove arm ${i}$ from $B_{m}$ if
%\begin{align*}
%\hat{r}_{i} + s_i  < \tau - s_i \mbox{ or } \hat{r}_{i} - s_i  > \tau + s_i \\
%\text{where $s_i=\sqrt{\frac{\rho_v\psi\hat{V}_{i}\log{( T\epsilon_{m}^{2})}}{4 n_{i}} + \frac{\rho_v\psi \log{(T\epsilon_{m}^{2})}}{4 n_{i}}}$}
%\end{align*}
%\EndArmElimV
%\State \If{$t\geq N_{m}$ and $m \leq M$}
%\ResetParam
%\State $\epsilon_{m+1}:=\frac{\epsilon_{m}}{2}$
%\State $B_{m+1} := B_{m}$
%\State $\ell_{m+1}:=\left\lceil \frac{2\psi\log( T\epsilon_{m+1}^{2})}{\epsilon_{m+1}} \right\rceil$
%\State $N_{m+1} := t + |B_{m+1}|\ell_{m+1}$
%\State $m := m+1$
%\EndResetParam
%\EndIf
%\EndFor
%\State Output $\hat{S}_{\tau}=\lbrace i: \hat{r}_{i}\geq \tau \rbrace$.
%\end{algorithmic}
%\end{algorithm}

%%%%%%%%%%%%%%%%%%%%%%%%%%%%%%%%%%%%%%%%%%%%%%%%%%%%%%%%%%%%%%%%%%%%%%%%%%%%%%%%%%%%%

%%%%%%%%%%%%%%%%%%%%%%%%%%%%%%%%%%%%%%%%%%%%%%%%%%%%%%%%%%%%%%%%%%%%%%%%%%%%%%%%%
%New 	Algorithm
%%%%%%%%%%%%%%%%%%%%%%%%%%%%%%%%%%%%%%%%%%%%%%%%%%%%%%%%%%%%%%%%%%%%%%%%%%%%%%%%%

\begin{algorithm}[th!]
\caption{AugmentedUCB}
\label{alg:augucb}
\begin{algorithmic}
\State {\bf Input:} Time horizon $T$, exploration parameters $\rho_{\mu}$, $\rho_v$ and threshold $\tau$.
\State {\bf Initialization:} Set $B_{0}:=\mathcal{A}$, $M=\left\lfloor \frac{1}{2}\log_{2} \frac{T}{e}\right\rfloor $, $m:=0$, $\epsilon_{0}:=1$, $\psi_{0}=\frac{T\epsilon_{0}}{8K\log K}$, $\ell_{0}=\left\lceil \frac{2\psi\log( T\epsilon_{0})}{\epsilon_{0}} \right\rceil$ and $N_{0}=K\ell_{0} $.
\State Pull each arm once
\State \For{$t=K+1,..,T$}
\State Pull arm $i\in\argmin_{j\in B_{m}}\bigg\lbrace |\hat{r}_{j} - \tau | - 2s_{j}\bigg\rbrace$
\State $t:=t+1$ 
\ArmElim
\State For each arm $i \in B_{m}$, remove arm ${i}$ from $B_{m}$ if
\begin{align*}
\hat{r}_{i} + c_i  < \tau - c_i \mbox{ or } \hat{r}_{i} - c_i  > \tau + c_i \\
\text{where $c_i=\sqrt{\frac{\rho_{\mu}\psi_{m}\log{( T\epsilon_{m})}}{2 n_{i}}}$}
\end{align*}
\EndArmElim
\ArmElimV
\State For each arm $i \in B_{m}$, remove arm ${i}$ from $B_{m}$ if
\begin{align*}
\hat{r}_{i} + s_i  < \tau - s_i \mbox{ or } \hat{r}_{i} - s_i  > \tau + s_i \\
\text{where $s_i=\sqrt{\frac{\rho_v\psi_{m}\hat{V}_{i}\log{( T\epsilon_{m})}}{4 n_{i}} + \frac{\rho_v\psi_{m} \log{(T\epsilon_{m})}}{4 n_{i}}}$}
\end{align*}
\EndArmElimV
\State \If{$t\geq N_{m}$ and $m \leq M$}
\ResetParam
\State $\epsilon_{m+1}:=\frac{\epsilon_{m}}{2}$
\State $B_{m+1} := B_{m}$
\State $\psi_{m+1}=\frac{T\epsilon_{m+1}}{8 K\log K}$
\State $\ell_{m+1}:=\left\lceil \frac{2\psi_{m+1}\log( T\epsilon_{m+1})}{\epsilon_{m+1}} \right\rceil$
\State $N_{m+1} := t + |B_{m+1}|\ell_{m+1}$
\State $m := m+1$
\EndResetParam
\EndIf
\EndFor
\State Output $\hat{S}_{\tau}=\lbrace i: \hat{r}_{i}\geq \tau \rbrace$.
\end{algorithmic}
\end{algorithm}

%%%%%%%%%%%%%%%%%%%%%%
%Notations moved here
%%%%%%%%%%%%%%%%%%%%%%
\label{notation}
\textbf{Notations and assumptions:} $\mathcal{A}$ denotes the set of arms, and $|\mathcal{A}|=K$ is the number of arms in $\mathcal{A}$. 
%Arms generic arm is indexed by $i,j\in\mathcal{A}$. 
For arm $i\in\mathcal{A}$, we use $r_{i}$ to denote the true mean of the distribution from which the rewards are sampled, while $\hat{r}_{i}(t)$ denotes the estimated mean at time $t$. Formally, using $n_i(t)$ to denote the number of times arm $i$ has been pulled until time $t$, we have $\hat{r}_{i}(t)=\frac{1}{n_{i}(t)}\sum_{z=1}^{n_i(t)} X_{i,z}$, where $X_{i,z}$ is the reward sample received when arm $i$ is pulled for the $z$-th time. %
Similarly, we use $\sigma_{i}^{2}$ to denote the true variance of the reward distribution corresponding to arm $i$, while $\hat{v}_{i}(t)$ is the estimated variance, i.e., $\hat{v}_{i}(t)=\frac{1}{n_i(t)}\sum_{z=1}^{n_{i}(t)}(X_{i,z}-\hat{r}_{i})^{2}$. Whenever there is no ambiguity about the underlaying  time index $t$, for simplicity we neglect $t$ from the notations and simply use  $\hat{r}_i, \hat{v}_i,$ and $n_i, $ to denote the respective quantities.  Let  $\Delta_{i}=|\tau-r_{i}|$ denote the distance of the true mean from the threshold $\tau$.



%The average estimated payoff for any arm is denoted by  whereas the true mean of the distribution  is denoted by . The optimal arm is denoted by $*$. The '*' superscript is used to denote anything related to optimal arm. 

 % $n_{i}$ denotes the number of times the arm $i$ has been pulled. $\psi $ denotes the exploration regulatory factor and $\rho_\mu ,\rho_v$ as arm elimination parameters. $\hat{V}_{i}=\frac{1}{n_i}\sum_{t=1}^{n_{i}}(x_{i,t}-r_{i})^{2}$ denotes the empirical variance and $x_{i,t}$ is the reward obtained at timestep t for arm $i$. Also   denotes the true variance of the arm $i$. 
 
Finally, we assume that all the reward distributions 
%from which rewards are sampled are identical and independent 
are $1$-sub-Gaussian (note that,  $1$-sub-Gaussian includes Gaussian distributions with variance less than $1$, distributions supported on an interval of length less than 2, etc). Further, the rewards are assumed to take values in the interval $[0,1]$.
%. In our case, we  assume that all rewards are .

%Also we define $\Delta_{i}=r^{*} - r_{i}$ and $\hat{\Delta}_{i}=\hat{r}^{*} - \hat{r}_{i}$. In all cases $\min_{i\in A}{\Delta_{i}}$ is denoted by $\Delta$.
%and the optimal arm is denoted by $*$. The '*' superscript is used to denote anything related to optimal arm
%\paragraph*{}It is assumed that the distribution from which rewards are sampled are identical and independent sub-Gaussian distributions. Throughout the paper, we assume that the distributions $v_{i}$ are sub-Gaussian that is $\int e^{\lambda(x - r)} v_{i} (dx) ≤ e^{\lambda /2}, \forall \lambda \in \mathbb{R}$. Note that these include Gaussian distributions with variance less than 1 and distributions supported on an interval of length less than 2. All the experiments are also conducted with sub-Gaussians having variance as 1. Together with a Chernoff-Hoeffding bound, the sub-Gaussian assumption implies the following concentration inequality, valid for any
%$u > 0$,
%\newline
%\hspace*{8em}$\mathbb{P}\lbrace \hat{r}_{i} - r^{*} > u\rbrace \leq exp(-\dfrac{su^{2}}{2}) $
%\newline
%where s is the number of pulls of $a_{i}$. T is the horizon over which this entire algorithm runs.  $A^{'}$ at any round $m$ denotes the arms still not eliminated.
%\paragraph*{}The paper is organized as follows. We first present the algorithm in section 6. We then provide the proofs of Phase1 which includes regret-bound calculation and arm deletion conditions in section 7. In section 8, we provide proofs for Phase2 along with early stopping conditions. Section 9 deals with regret bound and then we provide error probability and error bounds in section 10. Experimental results are provided in section 11, and we conclude in section 12.
%%%%%%%%%%%%%%%%%%%%%%
\textbf{Algorithm:} In algorithm \ref{alg:augucb}, hence referred to as AugUCB, we have two exploration parameters, $\rho_{\mu}$ and $\rho_v$ which are the arm elimination parameters. $\psi_{m}$ is the exploration regulatory factor. The main approach is based on UCB-Improved with modifications suited for the thresholding bandit problem. The active set $B_{0}$ is initialized with all the arms from $\mathcal{A}$. We divide the entire budget $T$ into rounds/phases as like UCB-Improved, CCB, SAR and CSAR. After the end of each such round $m$ we eliminate arm(s) from active set $B_{m}$ and update parameters. As suggested by \cite{liu2016modification} to make AugUCB an anytime algorithm and to overcome too much early exploration, we no longer pull all the arms equal number of times in each round but pull the arm that minimizes,  
$\min_{i\in B_{m}}\big\lbrace |\hat{r}_{i} - \tau | - 2\sqrt{\frac{\rho_v\psi_m \hat{V}_{i} \log ( T \epsilon_{m})}{4 n_{i}} + \frac{\rho_v\psi_m \log{( T\epsilon_{m})}}{4 n_{i}}} \big\rbrace $
in the active set $B_{m}$. This condition makes it possible to pull the arms closer to the threshold $\tau$ and with suitable choice of $\rho_{\mu}$ and $\rho_v$ we can fine tune the exploration. Also because of the said condition, like \cite{liu2016modification} we also claim that AugUCB is an anytime algorithm. The choice of exploration factor $\psi_m=\frac{T\epsilon_m}{(\log(\frac{3}{16} K\log K))^{2}}$ comes directly from \cite{audibert2010best} and \cite{bubeck2011pure} which states that in pure exploration setup, the exploring factor must be linear in $T$ to give us an exponentially small probability of error rather than logarithmic in $T$ which is suited for minimizing cumulative regret.

%
\vspace{-2mm}
\section{Theoretical Results}
\label{results}
% \subsection{Problem Complexity}

Let us begin by recalling the following definitions of the  \emph{problem complexity} as introduced in \cite{locatelli2016optimal}:
\begin{align*}
H_{1} = \sum_{i=1}^{K}\dfrac{1}{\Delta_{i}^{2}} \hspace{1mm}\text{     and }  \hspace{1mm}
H_{2} =\min_{i\in \mathcal{A}}\dfrac{i}{{\Delta_{(i)}^{2}}} 
\end{align*}
where $(\Delta_{(i)}: i\in\mathcal{A})$ is obtained by arranging $(\Delta_i:i\in\mathcal{A})$ in an increasing order. Also, from \cite{chen2014combinatorial} we have
\begin{align*}
H_{CSAR,2}=\max_{i\in\mathcal{A}}\frac{i}{\Delta_{(i)}^2}.
\end{align*}
$H_{CSAR,2}$ is the complexity term appearing in the bound for the CSAR algorithm. The relation between the above complexity terms are as follows (see \cite{locatelli2016optimal}):
%
%$H_1$ and $H_2$ is same as the problem complexity defined in \cite{locatelli2016optimal} for the thresholding bandit problem while $H_{CSAR,2}=\max_{i}\frac{i}{\Delta_{(i)}^2}$ is defined in \cite{chen2014combinatorial}. Also we know from \cite{locatelli2016optimal} that,
\begin{align*}
H_{2}\leq H_{1}\leq \log(2K)H_{2} \hspace{1mm} \text{ and } \hspace{1mm} H_1 \leq \log(K)H_{CSAR,2}
\end{align*}

As ours is a variance-aware algorithm, we require $H_{1}^{\sigma}$ (as defined in \cite{gabillon2011multi}) that incorporates reward variances into its expression as given below:
\begin{align*}
 H_{1}^{\sigma}=\sum_{i=1}^{K}\frac{\sigma_{i}+\sqrt{\sigma_{i}^{2}+(16/3)\Delta_{i}}}{\Delta_{i}^{2}}.
\end{align*}
Finally, analogous to $H_2$, in this paper we introduce the complexity term $H_2^\sigma$, which is given by
%and $H_{2}^{\sigma}$ (introduced in this paper) as,
\begin{align*}
%& H_{1}^{\sigma}=\sum_{i=1}^{K}\frac{\sigma_{i}+\sqrt{\sigma_{i}^{2}+(16/3)\Delta_{i}}}{\Delta_{i}^{2}}\\
H_{2}^{\sigma}=\min_{i\in \mathcal{A}} \frac{i}{\tilde{\Delta}_{(i)}^{2}}%& H_{2}^{\sigma}=\min_{i\in \mathcal{A}} i\frac{12\sigma_{(i)}^{2} + \Delta_{(i)}}{12\Delta_{(i)}^{2}}
\end{align*}
where $\tilde{\Delta}_{i}^{2}=\frac{\Delta_{i}^{2}}{\sigma_{i}+\sqrt{\sigma_{i}^{2}+(16/3)\Delta_{i}}}$, and $(\tilde{\Delta}_{(i)})$ is an increasing ordering of $(\tilde{\Delta}_{i})$.
Similar to the relation between $H_1$ and $H_2$, it can be shown that
%which also gives us that 
$H_{2}^{\sigma} \leq H_{1}^{\sigma} \leq \log(2K) H_{2}^{\sigma}$.

%\subsection{Theorem 1}
Our main result is summarized in the following theorem where we prove an  upper bound on the expected loss. 
\begin{theorem}
\label{Result:Theorem:1}
For $K\geq 4$ and
%with $\rho_{\mu}=\frac{1}{8}$ and 
$\rho={1}/{3}$,
the expected loss of the AugUCB algorithm is given by,
%\begin{small}
\begin{align*}
\E[\Ls(T)]
%\exp\bigg( -\frac{T\log (2 K\sqrt{\log K})}{2H_2 K (\log K)^{3/2}} + \log\bigg(K\big(\log_2\frac{T}{e}+1\big)\bigg)\bigg)\\
%& + \exp\bigg(- \frac{5T\log ( K\sqrt{\log K})}{H_{2}^{\sigma} K(\log K)^{3/2}}  + \log\bigg(K\big(\log_2\frac{T}{e}+1\big)\bigg)\bigg).
& \leq 2K\left(\log_2\frac{T}{e}+1\right)
% \bigg\lbrace\exp\bigg( -\frac{T}{ 64 H_2 a}\bigg)
% + 2
 \exp\bigg(- \frac{T}{4096 H_{2}^{\sigma} a} \bigg)
 %\bigg\rbrace
\end{align*}
where $a=\log(\frac{3}{16} K\log K)$.
%\end{small}
%For every $0<\eta <1$ and $\gamma > 1$, there exists time $t$ such that for all $T>t$ the simple regret of AugUCB is upper bounded by,
%\begin{small}
%\begin{align*}
%& SR_{AugUCB} \leq \sum_{i=1}^{K} \Delta_{i}\bigg\lbrace \exp\bigg(-4\rho\log (\psi T\frac{\Delta_{i}^{4}}{16\rho^{2}})-\dfrac{c_{0}\sqrt{T}}{16\rho H_{2}}\\
%& + \log \big( 16\gamma C_1\log_{2}\dfrac{T}{e} \big) \bigg) + \exp\bigg(- \dfrac{3\rho_v}{2} \bigg(\dfrac{2\sigma_{i}^{2}+\Delta_{i}+2}{6\sigma_{i}^{2}+\Delta_{i}}\bigg)\log(\psi T\frac{\Delta_{i}^{4}}{16\rho_{v}^{2}})\\
%& -\dfrac{c_{0}\sqrt{T}}{16\rho_v H_{2}} + \log\big ( 32\gamma C_2\log_{2}\dfrac{T}{e} \big)  \bigg)\bigg\rbrace
%\end{align*}
%\end{small}
%with probability at least $1-\eta$, where $c_{0}>0$ is a constant and $C_1=\dfrac{K\rho\log (\psi T \frac{\Delta_{i}^{4}}{16\rho^{2}})}{T\Delta_{i}^{2}}$ and $C_2= \dfrac{K\rho_v\log (\psi T \frac{\Delta_{i}^{4}}{16\rho_{v}^{2}})}{T\Delta_{i}^{2}}$.
\end{theorem}

\begin{proof}
The proof comprises three modules. In the first module we investigate the necessary conditions for arm elimination within a specified number of rounds, which is motivated by the technique in \cite{}. Bounds on the arm-elimination probability is then obtained; however, since we use variance estimates, we invoke the Bernstein inequality (as in \cite{}) rather that the Chernoff-Hoeffding bounds (which is appropriate for the UCB-Improved \cite{}). In the second module, as in \cite{}, we define a favourable event that will yield a bound for the expected loss. In the final module we conclude by combining the results for the first two modules. The details are as follows. 


\textbf{Arm Elimination:} Recall the notations used in the algorithm, Also, for each arm $i\in\mathcal{A}$, define $m_{i}=\min\left\lbrace m| \sqrt{\rho\epsilon_{m}}<\frac{\Delta_{i}}{2}\right\rbrace$. In the $m_i$-th round, whenever $n_i=\ell_{m_i}\ge\frac{2\psi_{m_i}\log{(T\epsilon_{m_{i}})}}{\epsilon_{m_{i}}}$, we obtain (as $\hat{v}_i\in[0,1]$)
%
%\begin{align*}
%s_{i}&=\sqrt{\dfrac{\rho \psi_{m_i} \hat{v}_{i} \epsilon_{m_{i}}\log ( T\epsilon_{m_{i}})}{4 n_{i}} + \dfrac{\rho \psi_{m_i}\log{( T\epsilon_{m_{i}})}}{4 n_{i}}} \\
%&\leq \sqrt{\dfrac{\rho\psi_{m_i} \epsilon_{m_{i}}\log ( T\epsilon_{m_{i}})}{4*2 \log(\psi_{m_i} T\epsilon_{m_{i}})} + \dfrac{\rho\psi_{m_i}\epsilon_{m_{i}} \log{( T\epsilon_{m_{i}})}}{4*2\psi_{m_i} \log( T\epsilon_{m_{i}})} } \text{, as }\hat{V}_{i}\in [0,1].\\
%& \leq \sqrt{\dfrac{\rho_v \epsilon_{g_{i}}}{8} + \dfrac{\rho_v \epsilon_{g_{i}}}{8} } \leq \dfrac{\sqrt{\rho_v \epsilon_{g_{i}}}}{2}< \dfrac{\Delta_{i}}{4} \text{, as }\rho_v\in (0,1].
%%& \leq \sqrt{\rho_v \epsilon_{g_{i}+1}} < \dfrac{\Delta_{i}}{4} \text{, as }\rho_v\in (0,1].
%\end{align*}
%
\begin{align}
\label{si_bound_equn}
s_i 
&\le \sqrt{\frac{\rho(\hat{v}_i+1)\epsilon_{m_i}}{8}}
% +\frac{\rho\epsilon_{m_i}}{8}}
  \le \frac{\sqrt{\rho\epsilon_{m_i}}}{2} < \frac{\Delta_i}{4}.
\end{align}

First, let us consider a bad arm $i\in\mathcal{A}$ (i.e., $r_i<\tau$). We note that, in the $m_i$-th round  whenever 
$\hat{r}_i \le r_i +2s_i$, then arm $i$ is eliminated as a bad arm. This is easy to verify as follows: using (\ref{si_bound_equn}) we obtain,
\begin{align*}
\hat{r}_{i}&\leq r_{i} + 2s_{i} \\
&= r_{i} + 4s_{i} - 2s_{i} \\
&< r_{i} - \Delta_{i} - 2s_{i} \\
&= \tau - 2s_{i} % \geq \tau + s_{i}
\end{align*}
which is precisely one of the elimination conditions in Algorithm~\ref{alg:augucb}. Thus, the probability that a bad arm is not eliminated correctly in the $m_i$-th round (or before) is given by

%%%%%%%%%%%%%%%%% Favorable event is defined here
%We note that in the $g_i$-th round arm $i$ can be pulled no more than $\ell_{g_i}$ number of times. 
%
%
%According to the algorithm, the number of rounds is $m=\lbrace 0,1,2,.. M\rbrace $ where $M=\bigg\lfloor \frac{1}{2}\log_{2} \frac{T}{e}\bigg\rfloor$. So, $\epsilon_{m}\geq 2^{-M}\geq \sqrt{\frac{e}{T}}$. Also each round $m$ consists of $|B_{m}|\ell_{m}$ timesteps where $\ell_{m} = \left\lceil\frac{2\psi_{m}\log( T \epsilon_{m})}{\epsilon_{m}}\right\rceil$, $B_{m}$ is the set of all surviving arms and let $a=(\log(\frac{3}{16} K\log K))$.
%
%
%Let $c_{i} = \sqrt{\frac{\rho_{\mu}\psi_{m} \log{(T\epsilon_{m})}}{2 n_{i}}}$ denote the confidence interval, where $n_{i}$ is the number of times an arm $i$ is pulled. Let $\mathcal{A}^{'}=\lbrace i\in \mathcal{A}|\Delta_{i}\geq b\rbrace$, for $b\geq \sqrt{\frac{e}{T}}$. Define $m_{i}=\min\lbrace m| \sqrt{\rho_{\mu}\epsilon_{m}}<\frac{\Delta_{i}}{2}\rbrace$.
%% Let $m_{i}$ be the minimum round such that an arm $i$ gets eliminated such that. 
%
%% Let $s_{i}=\sqrt{\frac{\rho_v\psi_{g} \hat{V_{i}} \log{( T\epsilon_{g})}}{4 n_{i}} + \frac{\rho_v\psi_{g} \log{( T\epsilon_{g})}}{4 n_{i}}}$ and 
%% $g_{i}$ be the minimum round that an arm $i$ gets eliminated such that $g_{i}=min\lbrace g| \sqrt{\rho_{v}\epsilon_{g}}<\frac{\Delta_{i}}{2}\rbrace$. 
%%In this proof sub-optimal arms refer to the arms whose $r_{i}$ is lower than the threshold $\tau$.
%
%%At the end of any round $\max\lbrace m_{i},g_{i}\rbrace$, for any arm $i$, two cases are possible.
%
%Let $\xi_{1}$ and $\xi_{2}$ be the favorable event such that,
%\begin{align*}
%\xi_{1}&=\bigg\lbrace \forall i\in \mathcal{A}, \forall m=0,1,2,..,M: |\hat{r_i} - r_i| \leq 2c_i\bigg\rbrace\\
%\xi_{2}&=\bigg\lbrace \forall i\in \mathcal{A}, \forall m=0,1,2,..,M: |\hat{r_i} - r_i| \leq  2s_i\bigg\rbrace
%\end{align*}
%
%So, $\xi_{1}$ and $\xi_{2}$ signifies the event any arm $i$ will get eliminated from $B_m$.
%%%%%%%%%%%%%%%%%%%%%%







%%%%%%%%%%%%%%%%%
%\subsubsection{\textit{Arm i is not eliminated on or before round $\max\lbrace m_{i},g_{i}\rbrace$}}
%
%For any arm $i$, if it is eliminated from active set $B_{m_{i}}$ then one of the below two events has to occur,
%%\begin{small}
%\begin{align}
%\hat{r}_{i} + c_{i} < \tau - c_{i}, \label{eq:armelim-casea}\\
%\hat{r}_{i} - c_{i} > \tau + c_{i}, \label{eq:armelim-caseb}
%\end{align}
%%\end{small}
%For (\ref{eq:armelim-casea}) we can see that it eliminates arms that have performed poorly and removes them  from $B_{m_{i}}$. Similarly, (\ref{eq:armelim-caseb}) eliminates arms from $B_{m_{i}}$ that have performed very well compared to threshold $\tau$.
%
%%Each round consists of $|B_{m_{i}}|\ell_{m_{i}}$ timesteps. 
%In the $m_{i}$-th round an arm $i$ can be pulled no more than $\ell_{m_{i}}$ times. So when $n_{i}=\ell_{m_{i}}$, putting the value of $\ell_{m_{i}}\ge\frac{2\psi_{m_i}\log{( T\epsilon_{m_{i}})}}{\epsilon_{m_{i}}}$ in $c_{i}$ we get, 
%%\begin{small}
%\begin{align*}
%c_{i}
%&=\sqrt{\frac{\rho_{\mu}\psi_{m_i}\epsilon_{m_{i}}\log ( T\epsilon_{m_{i}})}{2 n_{i}}}
%\le\sqrt{\frac{\rho_{\mu}\psi_{m_i}\epsilon_{i}\log ( T\epsilon_{m_{i}})}{2*2 \psi_{m_i} \log( T\epsilon_{m_{i}})}}\\
%& \le\frac{\sqrt{\rho_{\mu}\epsilon_{m_{i}}}}{2}
%% % \leq \sqrt{\rho_{\mu}\epsilon_{m_{i}+1}} 
%< \frac{\Delta_{i}}{4} \text{, as }\rho_{\mu}\in (0,1].
%\end{align*}
%%\end{small}
%Again, for ${i} \in \mathcal{A}^{'}$ for the  elimination condition in (\ref{eq:armelim-casea}), 
%%\begin{small}
%%\begin{align*}
%%\hat{r}_{i} + c_{i}&\leq r_{i} + 2c_{i} = r_{i} + 4c_{i} - 2c_{i} \\
%%&< r_{i} + \Delta_{i} - 2c_{i} = \tau -2c_{i} \leq \tau - c_{i}
%%\end{align*}
%%\end{small}
%%\begin{small}
%\begin{align*}
%\hat{r}_{i} &\leq r_{i} + 2c_{i} = r_{i} + 4c_{i} - 2c_{i} \\
%&< r_{i} + \Delta_{i} - 2c_{i} = \tau -2c_{i}.
%\end{align*}
%%\end{small}
%Similarly, for ${i} \in \mathcal{A}^{'}$ for the  elimination condition in (\ref{eq:armelim-caseb}), 
%%\begin{small}
%\begin{align*}
%\hat{r}_{i} &\geq r_{i} - 2c_{i} = r_{i} - 4c_{i} + 2c_{i} \\
%&> r_{i} - \Delta_{i} + 2c_{i}= \tau + 2c_{i}.
%\end{align*}
%%\end{small}
%
%
%%Now, arm elimination condition is being checked at every timestep, in the $m_{i}$-th round as soon as $n_{i}=\ell_{m_{i}}$, arm $i$ gets eliminated. 
%Applying Chernoff-Hoeffding bound and considering independence of complementary of the event in (\ref{eq:armelim-casea}),
%%\begin{small}
%\begin{align*}
%%\mathbb{P}\lbrace\hat{r}_{i}\geq r_{i} - 2c_{i}\rbrace &\leq exp(-2(\tau + 2c_{i})^{2}n_{i})\\
%&\mathbb{P}\lbrace\hat{r}_{i}> r_{i} + 2c_{i}\rbrace \leq \exp(-4 c_{i}^{2}n_{i})\\
%&\leq \exp(-8 * \dfrac{\rho_{\mu}\psi_{m_i}\log ( T\epsilon_{m_{i}})}{2 n_{i}} *n_{i})\\
%&\leq \exp\big(-4\rho_{\mu}\psi_{m_i}\log ( T\epsilon_{m_{i}})\big)\\
%&\leq \exp\left(-\rho_{\mu}\frac{T\epsilon_{m_{i}}}{32 a^2}\log ( T\epsilon_{m_{i}})\right),\\
%&\text{putting the value of $\psi_{m_i}=\frac{T\epsilon_{m_i}}{128(\log(\frac{3}{16} K\log K))^{2}}$}
%\end{align*}
%%\end{small}
%Similarly for the condition in (\ref{eq:armelim-caseb}), $\mathbb{P}\lbrace\hat{r}_{i}< r_{i} - 2c_{i}\rbrace\leq \exp\left(-\frac{T\rho_{\mu}\epsilon_{m_{i}}}{32 a^2 }\log ( T\epsilon_{m_{i}})\right)$.
%
%Summing the above two expressions, the probability that arm ${i}$ is not eliminated on or before $m_{i}$-th is $\left(2\exp\left(-\frac{T\rho_{\mu}\epsilon_{m_{i}}}{32 a^2 }\log ( T\epsilon_{m_{i}})\right)\right)$. 
%%%%%%%%%%%%%%%%%%%%%

%%%%%%%%%%%%
%Again for any arm $i$, if it is eliminated from active set $B_{g_{i}}$ then the below two events have to come true,
%%\begin{small}
%\begin{align}
%\hat{r}_{i} + s_{i} < \tau - s_{i}, \label{eq:armelim-var-casea}\\
%\hat{r}_{i} - s_{i} > \tau + s_{i}, \label{eq:armelim-var-caseb}
%\end{align}
%%\end{small}
%%
%% For \ref{eq:armelim-var-casea} we can see that it eliminates arms that have performed poorly and removes them them from $B_{g_{i}}$. Similarly, \ref{eq:armelim-var-caseb} eliminates arms from $B_{g_{i}}$ that have performed very well compared to threshold $\tau$.
%%But, we know that $\epsilon_{m_{i}}=\epsilon_{g_{i}}$ and round consist of $|B_{g_{i}}|\ell_{g_{i}}$ timesteps. 
%In the $g_{i}$-th round an arm $i$ can be pulled no more than $\ell_{g_{i}}$ times. So when $n_{i}=\ell_{g_{i}}$, putting the value of $\ell_{g_{i}}\ge\frac{2\psi_{m_i}\log{( T\epsilon_{g_{i}})}}{\epsilon_{g_{i}}}$ in $s_{i}$ we get, 
%%\begin{small}
%\begin{align*}
%s_{i}&=\sqrt{\dfrac{\rho_v \psi_{g_i} \hat{V}_{i} \epsilon_{g_{i}}\log ( T\epsilon_{g_{i}})}{4 n_{i}} + \dfrac{\rho_v \psi_{g_i}\log{( T\epsilon_{g_{i}})}}{4 n_{i}}} \\
%&\leq \sqrt{\dfrac{\rho_v\psi_{g_i} \epsilon_{g_{i}}\log ( T\epsilon_{g_{i}})}{4*2 \log(\psi_{g_i} T\epsilon_{g_{i}})} + \dfrac{\rho_v \psi_{g_i}\epsilon_{g_{i}} \log{( T\epsilon_{g_{i}})}}{4*2\psi_{g_i} \log( T\epsilon_{g_{i}})} } \text{, as }\hat{V}_{i}\in [0,1].\\
%& \leq \sqrt{\dfrac{\rho_v \epsilon_{g_{i}}}{8} + \dfrac{\rho_v \epsilon_{g_{i}}}{8} } \leq \dfrac{\sqrt{\rho_v \epsilon_{g_{i}}}}{2}< \dfrac{\Delta_{i}}{4} \text{, as }\rho_v\in (0,1].
%%& \leq \sqrt{\rho_v \epsilon_{g_{i}+1}} < \dfrac{\Delta_{i}}{4} \text{, as }\rho_v\in (0,1].
%\end{align*}
%%\end{small}
%
%Again, for ${i} \in \mathcal{A}^{'}$ for the elimination condition in (\ref{eq:armelim-var-casea}),
%%\begin{small}
%\begin{align*}
%\hat{r}_{i} &\leq r_{i} + 2s_{i} = r_{i} + 4s_{i} - 2s_{i} \\
%&< r_{i} + \Delta_{i} - 2s_{i} = \tau -2s_{i} % \leq \tau - s_{i}
%\end{align*}
%%\end{small} 
%
%
%Also, for ${i} \in \mathcal{A}^{'}$ for the elimination condition in (\ref{eq:armelim-var-caseb}), 
%%\begin{small}
%\begin{align*}
%\hat{r}_{i}&\geq r_{i} - 2s_{i} = r_{i} - 4s_{i} + 2s_{i} \\
%&> r_{i} - \Delta_{i} + 2s_{i}\geq \tau + 2s_{i} % \geq \tau + s_{i}
%\end{align*}
%%\end{small}
%%%%%%%%%%%%%%%%


%Since, arm elimination condition is being checked at every timestep, in the $g_{i}$-th round as soon as $n_{i}=\ell_{g_{i}}$, arm $i$ gets eliminated. 
% Applying Bernstein inequality and considering independence of complementary of the event in (\ref{eq:armelim-var-casea}),
%\begin{small}
\noindent
\begin{align}
\mathbb{P}(\hat{r}_{i}> r_{i} + 2s_{i})
% &= \mathbb{P}\bigg( \hat{r}_{i} > r_{i}+ 2\sqrt{\dfrac{\rho\psi_{m_i} \hat{v}_{i}\log( T\epsilon_{m_{i}}) + \rho\psi_{m_i} \log{( T\epsilon_{m_{i}})}}{4n_{i}} } \bigg)\nonumber\\
&\leq \mathbb{P}\left( \hat{r}_{i} > r_{i}+ 2\bar{s}_i\right)  % \label{eq:prob_eq1}\\ 
+ \mathbb{P}\left( \hat{v}_{i}\geq \sigma_{i}^{2}+\sqrt{\rho_{v}\epsilon_{g_{i}}}\right)\label{eq:prob_eq2}
\end{align}
where 
\begin{align*}
\bar{s}_i=\sqrt{\dfrac{\rho\psi_{m_i} (\sigma_{i}^{2}+\sqrt{\rho\epsilon_{m_{i}}} + 1)\log( T\epsilon_{m_{i}})}{4n_{i}}}
\end{align*}
%\end{small}
Note that, substituting $n_i=\ell_{m_i}\ge \frac{2\psi_{m_i}\log{(T\epsilon_{m_{i}})}}{\epsilon_{m_{i}}}$, $\bar{s}_i$ can be simplified to obtain,
\begin{align}
2\bar{s}_i
% &\le 2\sqrt{\dfrac{\rho_v\psi_{g_i} (\sigma_{i}^{2}+\sqrt{\rho_{v}\epsilon_{g_{i}}})\log( T\epsilon_{g_{i}})}{\frac{8\psi_{g_i}\log( T \epsilon_{g_{i}})}{\epsilon_{g_{i}}}} }
%+ \dfrac{\rho_v\psi_{g_i} \log{( T\epsilon_{g_{i}})}}{\frac{8\psi_{g_i}\log( T \epsilon_{g_{i}})}{\epsilon_{g_{i}}}}}
\leq \dfrac{\sqrt{\rho\epsilon_{m_{i}}(\sigma_{i}^{2}+\sqrt{\rho\epsilon_{m_{i}}} + 1)}}{2}\leq \sqrt{\rho \epsilon_{m_{i}}}.
\label{si_bar_equn}
\end{align}

%Now, we know that in the $g_{i}$-th round,
%%\begin{small}
%\begin{align*}
%& 2\sqrt{\dfrac{\rho_v\psi_{g_i} [\sigma_{i}^{2}+\sqrt{\rho_{v}\epsilon_{g_{i}}}]\log( T\epsilon_{g_{i}})}{4n_{i}} + \dfrac{\rho_v\psi_{g_i}  \log{(T\epsilon_{g_{i}})}}{4 n_{i}}}\\ &\leq  2\sqrt{\dfrac{\rho_v\psi_{g_i} [\sigma_{i}^{2}+\sqrt{\rho_{v}\epsilon_{g_{i}}}]\log( T\epsilon_{g_{i}})}{\frac{8\psi_{g_i}\log( T \epsilon_{g_{i}})}{\epsilon_{g_{i}}}} + \dfrac{\rho_v\psi_{g_i} \log{( T\epsilon_{g_{i}})}}{\frac{8\psi_{g_i}\log( T \epsilon_{g_{i}})}{\epsilon_{g_{i}}}}}\\
%& \leq \dfrac{\sqrt{\rho_v \epsilon_{g_{i}}[\sigma_{i}^{2}+\sqrt{\rho_{v}\epsilon_{g_{i}}} + 1]}}{2}\leq \sqrt{\rho_v \epsilon_{g_{i}}}
%\end{align*}
%%\end{small}
%--------------------

The first term in the LHS of (\ref{eq:prob_eq2}) can be bounded using the Bernstein inequality as below:
\begin{align}
&\mathbb{P}\left( \hat{r}_{i} > r_{i}+ 2\bar{s}_i\right)\nonumber \\
&\le \exp\left(- \dfrac{(2\bar{s}_i)^2 n_i}{2\sigma_i^2+\frac{4}{3}\bar{s}_i}\right)\nonumber \\
& \le \exp\left(- \dfrac{\rho\psi_{m_i} (\sigma_{i}^{2}+\sqrt{\rho\epsilon_{m_{i}}} + 1)\log( T\epsilon_{m_{i}})}{2\sigma_i^2+\frac{2}{3}\sqrt{\rho \epsilon_{m_{i}}}}\right)\nonumber \\
& \overset{(a)}{\leq} \exp\left(- \dfrac{3\rho T\epsilon_{m_i}}{256 a^2} \left(\dfrac{\sigma_{i}^{2}+\sqrt{\rho\epsilon_{m_{i}}}+1}{3\sigma_{i}^{2}+\sqrt{\rho \epsilon_{m_{i}}}}\right) \log( T\epsilon_{m_{i}}) \right) \nonumber \\
&:= \exp(-Z_i) 
\label{lhs1_equn}
\end{align}
where, for simplicity, we have used $\alpha_i$ to denoted the exponent in the inequality $(a)$.
Also, note that $(a)$ is obtained by using  $\psi_{m_i}=\frac{T\epsilon_{m_i}}{128a^{2}}$, where $a=(\log(\frac{3}{16} K\log K))$.
%For the term in (\ref{eq:prob_eq1}), by applying Bernstein inequality, we can write as,
%\begin{small}
%\begin{align*}
%&\mathbb{P}\bigg( \hat{r}_{i}> r_{i} + \bigg(2\sqrt{\frac{\rho_v\psi_{g_i} [\sigma_{i}^{2}+\sqrt{\rho_{v}\epsilon_{g_{i}}} + 1]\log( T\epsilon_{g_{i}})}{4n_{i}}  } \bigg)\bigg)\\
%%%%%%%%%%%%%%%%%%%%%%%%
% &\leq \exp\bigg(- \dfrac{\bigg(2\sqrt{\frac{\rho_v\psi_{g_i} [\sigma_{i}^{2}+\sqrt{\rho_{v}\epsilon_{g_{i}}} +1]\log( T\epsilon_{g_{i}})}{4n_{i}}}\bigg)^{2}n_{i}}{2\sigma_{i}^{2}+\frac{4}{3}\sqrt{\frac{\rho_v\psi_{g_i} [\sigma_{i}^{2}+\sqrt{\rho_{v}\epsilon_{g_{i}}}+1]\log( T\epsilon_{g_{i}})}{4n_{i}}}}\bigg) \\
%%%%%%%%%%%%%%%%%%%%%%%
%&\leq \exp\bigg(- \dfrac{\bigg(\rho_v\psi_{g_i} [\sigma_{i}^{2}+\sqrt{\rho_{v}\epsilon_{g_{i}}} + 1]\log( T\epsilon_{g_{i}})\bigg)}{2\sigma_{i}^{2}+\frac{2}{3}\sqrt{\rho_v \epsilon_{g_{i}}}} \bigg)\\
% &\leq \exp\bigg(- \dfrac{3\rho_v\psi_{g_i}}{2} \bigg(\dfrac{\sigma_{i}^{2}+\sqrt{\rho_{v}\epsilon_{g_{i}}}+1}{3\sigma_{i}^{2}+\sqrt{\rho_v \epsilon_{g_{i}}}}\bigg) \log( T\epsilon_{g_{i}}) \bigg)\\
%%%%%%%%%%%%%%%%%%%%%%%
% &\leq \exp\left(- \dfrac{3\rho_v T\epsilon_{g_i}}{256 a^2} \left(\dfrac{\sigma_{i}^{2}+\sqrt{\rho_{v}\epsilon_{g_{i}}}+1}{3\sigma_{i}^{2}+\sqrt{\rho_v \epsilon_{g_{i}}}}\right) \log( T\epsilon_{g_{i}}) \right),
% &\text{ putting the value of $\psi_{g_i}=\frac{T\epsilon_{g_i}}{128(\log(\frac{3}{16} K\log K))^{2}}$}
%%%%%%%%%%%%%%%%%%%%%%%%%%%%%%%%%%%%%%%%%%%%%%%%%%%%%%%%%%%%%%%%%%%%%%%%%%%%%%%%%%
%\begin{align*}
%&\mathbb{P}\bigg\lbrace \hat{r}_{i}> r_{i} + \bigg(2\sqrt{\frac{\rho_v\psi_{g_i} [\sigma_{i}^{2}+\sqrt{\rho_{v}\epsilon_{g_{i}}} + 1]\log( T\epsilon_{g_{i}})}{4n_{i}}  } \bigg)\bigg\rbrace\\
%&\leq \exp\bigg(- \dfrac{\bigg(2\sqrt{\frac{\rho_v\psi_{g_i} [\sigma_{i}^{2}+\sqrt{\rho_{v}\epsilon_{g_{i}}}]\log( T\epsilon_{g_{i}})}{4n_{i}} + \frac{\rho_v\psi_{g_i} \log{( T\epsilon_{g_{i}})}}{4 n_{i}}}\bigg)^{2}n_{i}}{2\sigma_{i}^{2}+\frac{4}{3}\sqrt{\frac{\rho_v\psi_{g_i} [\sigma_{i}^{2}+\sqrt{\rho_{v}\epsilon_{g_{i}}}]\log( T\epsilon_{g_{i}})}{4n_{i}}+\frac{\rho_v\psi_{g_i} \log{( T\epsilon_{g_{i}})}}{4 n_{i}}}}\bigg) \\
%&\leq \exp\bigg(- \dfrac{\bigg(\rho_v\psi_{g_i} [\sigma_{i}^{2}+\sqrt{\rho_{v}\epsilon_{g_{i}}} + 1]\log( T\epsilon_{g_{i}})\bigg)}{2\sigma_{i}^{2}+\frac{2}{3}\sqrt{\rho_v \epsilon_{g_{i}}}} \bigg)\\
%&\leq \exp\bigg(- \dfrac{3\rho_v\psi_{g_i}}{2} \bigg(\dfrac{\sigma_{i}^{2}+\sqrt{\rho_{v}\epsilon_{g_{i}}}+1}{3\sigma_{i}^{2}+\sqrt{\rho_v \epsilon_{g_{i}}}}\bigg) \log( T\epsilon_{g_{i}}) \bigg)\\
%&\leq \exp\left(- \dfrac{3\rho_v T\epsilon_{g_i}}{16 K\log K} \left(\dfrac{\sigma_{i}^{2}+\sqrt{\rho_{v}\epsilon_{g_{i}}}+1}{3\sigma_{i}^{2}+\sqrt{\rho_v \epsilon_{g_{i}}}}\right) \log( T\epsilon_{g_{i}}) \right),\\
%&\text{ putting the value of $\psi_{g_i}=\frac{T\epsilon_{g_i}}{128(\log(\frac{3}{16} K\log K))^{2}}$}
%%%%%%%%%%%%%%%%%%%%%%%%%%%%%%%%%%%%%%%%%%%%%%%%%%%%%%%%%%%%%%%%%%%%%%%%%%%%%%%%%%
%\end{align*}
%\end{small}
% where  the last inequality is obtained using 
% \begin{align*}
% \psi_{m_i}=\frac{T\epsilon_{m_i}}{128(\log(\frac{3}{16} K\log K))^{2}}.
% \end{align*}
 
 The second term in the LHS of (\ref{eq:prob_eq2}) can be simplified as follows:
% For the term in , by applying Bernstein inequality, we can write as,
%\begin{small}
\begin{align}
&\mathbb{P}\bigg\lbrace \hat{v}_{i}\geq \sigma_{i}^{2}+\sqrt{\rho_{v}\epsilon_{g_{i}}}\bigg\rbrace\nonumber\\
&\leq \mathbb{P}\bigg\lbrace \dfrac{1}{n_{i}}\sum_{t=1}^{n_{i}}(X_{i,t}-r_{i})^{2}-(\hat{r}_{i}-r_{i})^{2}\geq \sigma_{i}^{2}+\sqrt{\rho\epsilon_{m_{i}}}\bigg\rbrace\nonumber\\
&\leq \mathbb{P}\bigg\lbrace \dfrac{\sum_{t=1}^{n_{i}}(X_{i,t}-r_{i})^{2}}{n_{i}}\geq \sigma_{i}^{2}+\sqrt{\rho\epsilon_{m_{i}}} \bigg\rbrace\nonumber\\
&\overset{(a)}{\leq} \mathbb{P}\bigg\lbrace \dfrac{\sum_{t=1}^{n_{i}}(X_{i,t}-r_{i})^{2}}{n_{i}}\geq \sigma_{i}^{2} + 2\bar{s}_i\bigg\rbrace \nonumber\\
% &\bigg(2\sqrt{\dfrac{\rho_v\psi_{g_i} [\sigma_{i}^{2}+\sqrt{\rho_{v}\epsilon_{g_{i}}}]\log( T\epsilon_{g_{i}})}{4n_{i}}+\frac{\rho_v\psi_{g_i}  \log{(T\epsilon_{g_{i}})}}{4 n_{i}}}\bigg)\bigg\rbrace\\
&\overset{(b)}{\leq} \exp\bigg(- \dfrac{3\rho\psi_{m_i}}{2} \bigg(\dfrac{\sigma_{i}^{2}+\sqrt{\rho\epsilon_{m_{i}}}+1}{3\sigma_{i}^{2}+\sqrt{\rho \epsilon_{m_{i}}}}\bigg) \log( T\epsilon_{m_{i}}) \bigg)\nonumber \\
%&\leq \exp\bigg(- \dfrac{3\rho_vT\epsilon_{g_i}}{256 a^2 } \bigg(\dfrac{\sigma_{i}^{2}+\sqrt{\rho_{v}\epsilon_{g_{i}}}+1}{3\sigma_{i}^{2}+\sqrt{\rho_v \epsilon_{g_{i}}}}\bigg) \log( T\epsilon_{g_{i}}) \bigg)
& = \exp(-Z_i)
%&\text{ putting the value of $\psi_{g_i}=\frac{T\epsilon_{g_i}}{128(\log(\frac{3}{16} K\log K))^{2}}$}
\label{lhs2_equn}
\end{align}
%\end{small}
where inequality $(a)$ is obtained using (\ref{si_bar_equn}), while $(b)$ follows from the Bernstein inequality. 
  
Thus, using (\ref{lhs1_equn}) and (\ref{lhs2_equn}) in (\ref{eq:prob_eq2}) we obtain $\mathbb{P}(\hat{r}_{i}> r_{i} + 2s_{i})\le 2\exp(-Z_i)$.
% \begin{small}
% \begin{align*}
%& \mathbb{P}(\hat{r}_{i}> r_{i} + 2s_{i}) \le\\ 
%&2\exp\left(- \frac{3T\rho_v\epsilon_{g_{i}}}{256 a^2 } \left(\frac{\sigma_{i}^{2}+\sqrt{\rho_{v}\epsilon_{g_{i}}}+1}{3\sigma_{i}^{2}+\sqrt{\rho_v \epsilon_{g_{i}}}}\right) \log( T\epsilon_{g_{i}}) \right)
% \end{align*}
% \end{small}
 %
Proceeding similarly, for a good arm $i\in\mathcal{A}$, the probability that it is not correctly eliminated in the $m_i$-th round (or before) is also bounded by $\mathbb{P}(\hat{r}_{i}< r_{i} - 2s_{i})\le 2\exp(-Z_i)$. In general, for any $i\in\mathcal{A}$ we have
\begin{align}
\Pb(|\hat{r}_i-r_i|>2s_i) 
&\le4\exp(-Z_i).
\label{final_bound_equn}
\end{align}
  
  
%Similarly, the condition for the complementary event for the elimination case \ref{eq:armelim-var-caseb} holds such that $\mathbb{P}\lbrace\hat{r}_{i}< r_{i} - 2s_{i}\rbrace \leq 2\exp\left(- \frac{3T\rho_v\epsilon_{g_{i}}}{256 a^2 } \left(\frac{\sigma_{i}^{2}+\sqrt{\rho_{v}\epsilon_{g_{i}}}+1}{3\sigma_{i}^{2}+\sqrt{\rho_v \epsilon_{g_{i}}}}\right) \log( T\epsilon_{g_{i}}) \right)$.


\textbf{Favourable Event:} Following the notation in \cite{locatelli2016optimal} we define the event
\begin{align*}
\xi&=\bigg\lbrace \forall i\in \mathcal{A}, \forall t=1,2,..,T: |\hat{r_i} - r_i| \leq  2s_i\bigg\rbrace.
\end{align*}
Note that, on $\xi$ each arm $i\in \mathcal{A}$  is eliminated correctly in the $m_i$-th round (or before). Thus, it follows that $\mathbb{E}[\mathcal{L}(T)]\le P(\xi^c)$. Since $\xi^c$ can be expressed as an union of the events $(|\hat{r}_i-r_i|>2s_i)$ for all $i\in\mathcal{A}$ and all $t=1,2,\cdots,T$, using union bound we can write
\begin{align*}
&\mathbb{E}[\mathcal{L}(T)] \\
&\le \sum_{i\in\mathcal{A}}\sum_{t=1}^T \Pb(|\hat{r}_i-r_i|>2s_i) \\
&\le \sum_{i\in\mathcal{A}}\sum_{t=1}^T 4 \exp(-Z_i) \\
&\le 4T\sum_{i\in\mathcal{A}} \exp\left(- \dfrac{3\rho T\epsilon_{m_i}}{256 a^2} \left(\dfrac{\sigma_{i}^{2}+\sqrt{\rho\epsilon_{m_{i}}}+1}{3\sigma_{i}^{2}+\sqrt{\rho \epsilon_{m_{i}}}}\right) \log( T\epsilon_{m_{i}}) \right) \\
&\overset{(a)}{\le} 4T \sum_{i\in\mathcal{A}} \exp\left(- \frac{3T\Delta_{i}^{2}}{4096 a^2} \left(\frac{4\sigma_{i}^{2}+\Delta_{i}+4}{12\sigma_{i}^{2}+\Delta_{i}}\right) \log( \frac{3}{16} T\Delta_{i}^{2}) \right) \\
&\overset{(b)}{\le} 4T \sum_{i\in\mathcal{A}}\exp\bigg(- \frac{12T\Delta_{i}^{2}}{(12\sigma_{i}+ 12\Delta_{i})}\frac{\log (\frac{3}{16} K\log K)}{4096 a^2 } \bigg) \\
&\overset{(c)}{\le} 4T \sum_{i\in\mathcal{A}} \exp\bigg(- \frac{T\Delta_{i}^{2}\log ( \frac{3}{16} K\log K)}{4096 (\sigma_{i} + \sqrt{\sigma_{i}^{2} + (16/3)\Delta_{i}}) a^2} \bigg) \\
& \overset{(d)}{\le}4T \sum_{i\in\mathcal{A}} \exp\bigg(- \frac{T\log ( \frac{3}{16} K\log K)}{4096 \max_{i}(i\tilde{\Delta}_{(i)}^{-2}) (\log(\frac{3}{16} K\log K))^{2}} \bigg) \\
& \overset{(e)}{\le}4KT \exp\bigg(- \frac{T}{4096 H_{2}^{\sigma} (\log(\frac{3}{16} K\log K))}\bigg).
\end{align*}
The following are used to achieve the above simplification:
\begin{itemize}
\item $(a)$ is obtained by noting that in round $m_i$ we have $\frac{\Delta_i}{4}\leq\sqrt{\epsilon_{m_{i}}\rho}<\frac{\Delta_i}{2}$.
\item For $(b)$, we note that the function $x\mapsto x\exp(-Cx^2)$, where $x\in[0,1]$, is  decreasing on $[1/\sqrt{2C},1]$ for any $C>0$ (see \cite{bubeck2011pure,auer2010ucb}). Thus, using $C=\lfloor \sqrt{e/T}\rfloor$ and putting $\min_{i\in \mathcal{A}}\Delta_i =\Delta =\sqrt{\frac{K\log K}{T}} > \sqrt{\frac{e}{T}},\forall i\in \mathcal{A}$ we obtain (b).
\item To obtain $(c)$ we have used the inequality $\Delta_i\le \sqrt{\sigma_{i}^{2} + (16/3)\Delta_{i}}$ (which holds because $\Delta_i\in[0,1]$).
\item To obtain $(d)$ and $(e)$, recall that $\tilde{\Delta}_i=\frac{\Delta_{i}^{2}}{\sigma_{i}+\sqrt{\sigma_{i}^{2}+(16/3)\Delta_{i}}}$ and $a=\log(\frac{3}{16} K\log K)$. Finally, note that
\begin{align*}
\tilde{\Delta}_i^{-2}\le \max_{j\in\mathcal{A}}\Delta_{(j)}^{-2}\le  \max_{j\in\mathcal{A}}j \Delta_{(j)}^{-2}=H_2^\sigma.
\end{align*}
\end{itemize}

%Again  summing the above expressions, the probability that an arm ${i}$ is not eliminated on or before $g_{i}$-th round based on the (\ref{eq:armelim-var-casea}) and (\ref{eq:armelim-var-caseb}) elimination condition is  $4\exp\left(- \frac{3T\rho_v\epsilon_{g_{i}}}{256 a^2 } \left(\frac{\sigma_{i}^{2}+\sqrt{\rho_{v}\epsilon_{g_{i}}}+1}{3\sigma_{i}^{2}+\sqrt{\rho_v \epsilon_{g_{i}}}}\right) \log( T\epsilon_{g_{i}}) \right)$. 
  
%%%%%%%%%%%%%%%%%%%%%%%%%%%%%%%%%%%%%%%%%%%%%%%%%%%%%%%%%%%%%%%%%%%%%%%%%%%%%%%%%%%%%%
%Not Required for probability of error for AugUCB
%%%%%%%%%%%%%%%%%%%%%%%%%%%%%%%%%%%%%%%%%%%%%%%%%%%%%%%%%%%%%%%%%%%%%%%%%%%%%%%%%%%%%%

%We start with an upper bound on the number of plays $\delta_{\max\lbrace m_{i}, g_{i}\rbrace}$ in the $\max\lbrace m_{i}, g_{i}\rbrace$-th round. We know that the total number of arms surviving in the $\max\lbrace m_{i}, g_{i}\rbrace$-th arm is, 
%
%\begin{small}
%\begin{align*}
%&|B_{\max\lbrace m_{i}, g_{i}\rbrace}|=2K\exp\bigg(-4\rho_{\mu}\log (\psi T\epsilon_{m_{i}}^{2})\bigg)\\ 
%& + 4K\exp\bigg(- \frac{3\rho_v}{2} \big(\frac{\sigma_{i}^{2}+\sqrt{\rho_{v}\epsilon_{g_{i}}}+1}{3\sigma_{i}^{2}+\sqrt{\rho_v \epsilon_{g_{i}}}}\big) \log(\psi T\epsilon_{g_{i}}^{2}) \bigg)
%\end{align*}     
%\end{small}
%
%
%Again for AugUCB, we know that the number of pulls allocated for each surviving arm $i$ in the $m_{i}$-th round is $\ell_{m_{i}}=\frac{2\log (\psi T \epsilon_{m_{i}}^{2})}{\epsilon_{m_{i}}}$ or for the $g_{i}$-th round is $\ell_{g_{i}}=\frac{2\log (\psi T \epsilon_{g_{i}}^{2})}{\epsilon_{g_{i}}}$. Therefore, the proportion of plays $\delta_{\max\lbrace m_{i}, g_{i}\rbrace}$ in the $\max\lbrace m_{i}, g_{i}\rbrace$-th round can be written as,
%
%\begin{small}
%\begin{align*}
%&\delta_{\max\lbrace m_{i}, g_{i}\rbrace}=(|B_{m_{i}}|.\ell_{m_{i}}) + (|B_{g_{i}}|.\ell_{g_{i}})\\
%&\leq 2K\exp\bigg(-4\rho_{\mu}\log (\psi T\epsilon_{m_{i}}^{2})\bigg).\dfrac{2\log (\psi T \epsilon_{m_{i}}^{2})}{\epsilon_{m_{i}}}\\
% & + 4K\exp\bigg(- \dfrac{3\rho_v}{2} \bigg(\dfrac{\sigma_{i}^{2}+\sqrt{\rho_{v}\epsilon_{g_{i}}}+1}{3\sigma_{i}^{2}+\sqrt{\rho_v \epsilon_{g_{i}}}}\bigg) \log(\psi T\epsilon_{g_{i}}^{2})\bigg).\dfrac{2\log (\psi T \epsilon_{g_{i}}^{2})}{\epsilon_{g_{i}}} \\
%& \leq \dfrac{4K\log (\psi T \epsilon_{m_{i}}^{2})}{\epsilon_{m_{i}}}\exp\bigg(-4\rho_{\mu}\log (\psi T\epsilon_{m_{i}}^{2})\bigg)\\
%& + \dfrac{8K\log (\psi T \epsilon_{g_{i}}^{2})}{\epsilon_{g_{i}}}\exp\bigg(- \dfrac{3\rho_v}{2} \bigg(\dfrac{\sigma_{i}^{2}+\sqrt{\rho_{v}\epsilon_{g_{i}}}+1}{3\sigma_{i}^{2}+\sqrt{\rho_v \epsilon_{g_{i}}}}\bigg) \log(\psi T\epsilon_{g_{i}}^{2}) \bigg)
%\end{align*}
%\end{small}

%Now, in the $\max\lbrace m_{i}, g_{i}\rbrace$-th round $\sqrt{\rho_{\mu}\epsilon_{m_{i}}}\leq \frac{\Delta_{i}}{2}$ or $\sqrt{\rho_v\epsilon_{g_{i}}}\leq \frac{\Delta_{i}}{2}$. Hence,
%
%\begin{small}
%\begin{align*}
%&\delta_{\max\lbrace m_{i},g_{i}\rbrace} \leq \dfrac{4K\log (\psi T \frac{\Delta_{i}^{4}}{16\rho_{\mu}^{2}})}{\frac{\Delta_{i}^{2}}{4\rho_{\mu}}}\exp\bigg(-4\rho_{\mu}\log (\psi T\frac{\Delta_{i}^{4}}{16\rho_{\mu}^{2}})\bigg)\\
%& + \dfrac{8K\log (\psi T \frac{\Delta_{i}^{4}}{16\rho_{v}^{2}})}{\frac{\Delta_{i}^{2}}{4\rho_{v}}}\exp\bigg(- \dfrac{3\rho_v}{2} \bigg(\dfrac{\sigma_{i}^{2}+\frac{\Delta_{i}}{2}+1}{3\sigma_{i}^{2}+\frac{\Delta_{i}}{2}}\bigg) \log(\psi T\frac{\Delta_{i}^{4}}{16\rho_{v}^{2}}) \bigg)\\
%%%%%%%%%%%%%%%%%%%%%%%%%%%%%%%%%%%%%%%%
%&\leq 16 C_1\exp\bigg(-4\rho_{\mu}\log (\psi T\frac{\Delta_{i}^{4}}{16\rho_{\mu}^{2}})\bigg)\\
%& + 32C_2\exp\bigg(- \dfrac{3\rho_v}{2} \bigg(\dfrac{2\sigma_{i}^{2}+\Delta_{i}+2}{6\sigma_{i}^{2}+\Delta_{i}}\bigg) \log(\psi T\frac{\Delta_{i}^{4}}{16\rho_{v}^{2}}) \bigg)\\
%&\text{where $C_1=\frac{K\rho_{\mu}\log (\psi T \frac{\Delta_{i}^{4}}{16\rho_{\mu}^{2}})}{\Delta_{i}^{2}}$ and $C_2= \frac{K\rho_v\log (\psi T \frac{\Delta_{i}^{4}}{16\rho_{v}^{2}})}{\Delta_{i}^{2}}$}\\
%%%%%%%%%%%%%%%%%%%%%%%%%%%%%%%%%%%%%%%%
%&\leq 16 C_1\exp\bigg(-4\rho_{\mu}\log (\psi T\frac{\Delta_{i}^{4}}{16\rho^{2}})\bigg)
% + 32C_2\exp\bigg(- \dfrac{3\rho_v}{2} \log(\psi T\frac{\Delta_{i}^{4}}{16\rho_{v}^{2}}) \bigg)
%\end{align*}
%\end{small}
%
%%Summing over all rounds $m=0,1,..,M$,
%Now, putting the values of $\psi$, $\rho_{\mu}$, $\rho_v$ and taking $\Delta_{i}\geq\min_{i\in A}\Delta=\sqrt{\frac{K\log K}{T}}\geq \sqrt{\frac{e}{T}},\forall i\in A$( see \cite{auer2010ucb}), 
%
%\begin{small}
%\begin{align*}
%& \delta_{\max\lbrace m_{i}, g_{i}\rbrace}= \bigg\lbrace 16 C_1\exp\bigg(-4\rho_{\mu}\log (\psi T\frac{\Delta_{i}^{4}}{16\rho_{\mu}^{2}})\bigg)\\
%& + 32C_2\exp\bigg(- \frac{3\rho_v}{2} \log(\psi T\frac{\Delta_{i}^{4}}{16\rho_{v}^{2}}) \bigg) \bigg\rbrace\\
%%%%%%%%%%%%%%%%%%%%%
%&\leq \bigg\lbrace  \frac{2K\log ( T^2 \frac{4\Delta_{i}^{4}}{\log K})}{\Delta_{i}^{2}}\exp\bigg(-\frac{1}{2}\log ( T^2\frac{4\Delta_{i}^{4}}{\log K})\bigg)\\
%& + \frac{32K\log ( T^2 \frac{9\Delta_{i}^{4}}{\log K})}{3\Delta_{i}^{2}}\exp\bigg(- \frac{1}{2} \log( T^2 \frac{9\Delta_{i}^{4}}{\log K}) \bigg) \bigg\rbrace\\
%%%%%%%%%%%%%%%%%%%%%
%&\leq \bigg\lbrace  \frac{4K\log ( T \frac{2\Delta_{i}^{2}}{\sqrt{\log K}})}{\Delta_{i}^{2}}\exp\bigg(-\log ( T\frac{2\Delta_{i}^{2}}{\sqrt{\log K}})\bigg)\\
%& + \frac{64K\log ( T \frac{3\Delta_{i}^{2}}{\sqrt{\log K}})}{3\Delta_{i}^{2}}\exp\bigg(- \log( T \frac{3\Delta_{i}^{2}}{\sqrt{\log K}}) \bigg) \bigg\rbrace\\
%%%%%%%%%%%%%%%%%%%%%
%&\leq \bigg\lbrace  \frac{4KT\log ( \frac{2 K\log K}{\sqrt{\log K}})}{K\log K}\exp\bigg(-\log ( \frac{2K\log K}{\sqrt{\log K}})\bigg)\\
%& + \frac{64TK\log (\frac{3 K\log K}{\sqrt{\log K}})}{3 K\log K}\exp\bigg(- \log( \frac{3 K\log K}{\sqrt{\log K}}) \bigg) \bigg\rbrace\\
%%%%%%%%%%%%%%%%%%%%
%&\leq \bigg\lbrace  \frac{2T\log (2 K\sqrt{\log K})}{K (\log K)^{3/2}}
% + \frac{22T\log ( K\sqrt{\log K})}{ K(\log K)^{3/2}}\bigg) \bigg\rbrace\\
%\end{align*}
%\end{small}
%Now we know that till $m_i$-th round $2c_i > \frac{\Delta_i}{2}$  or till $g_i$ th round $2s_i > \frac{\Delta_i}{2}$. Hence, for the $i$-th arm we can bound the probability of error for any round $m$ by applying Chernoff-Hoeffding and Bernstein inequality,
%\begin{small}
%\begin{align*}
% \Pb\lbrace \xi_1\rbrace  + \Pb\lbrace \xi_2 \rbrace &\geq 1-(\Pb\lbrace |\hat{r}_i -r_i| > 2c_i \rbrace + \Pb\lbrace |\hat{r}_i -r_i| > 2s_i \rbrace)\\ 
%&\geq 1-\left(\Pb\lbrace |\hat{r}_i - r_i| > \frac{\Delta_i}{2} \rbrace + \Pb\lbrace |\hat{r}_i - r_i| > \frac{\Delta_i}{2} \rbrace\right) \\
%&\geq 1-\big(2\exp( -\frac{\Delta_{i}^{2}}{4}n_i ) + 2\exp(- \frac{\Delta_{i}^{2}}{8\sigma_{i}^{2}+ \frac{4}{3}\Delta_i}n_i)\big)\\
%&\geq 1-\bigg(2\exp( -\frac{\Delta_{i}^{2}}{4}\delta_{m_{i}} ) + 2\exp(- \frac{\Delta_{i}^{2}}{8\sigma_{i}^{2}+ \frac{4}{3}\Delta_i}\delta_{g_{i}})\bigg)
%\end{align*}
%\end{small}
%Now, we know that $\E[\Ls(T)]\le1- (\Pb\lbrace \xi_1\rbrace  + \Pb\lbrace \xi_2 \rbrace) $. Summing over all arms $K$ and over all rounds $m=0,1,2,..,M$ we get that,
%\begin{small}
%\begin{align*}
%&\E[\Ls(T)] \leq \sum_{i=1}^{K}\sum_{m=0}^{M}\bigg\lbrace 2\exp\bigg( -\frac{\Delta_{i}^{2}}{4}.\frac{2T\log (2 K\sqrt{\log K})}{K (\log K)^{3/2}}\bigg)\\
%& + 2\exp\bigg(- \frac{\Delta_{i}^{2}}{8\sigma_{i}^{2}+ \frac{4}{3}\Delta_i}.\frac{22T\log ( K\sqrt{\log K})}{ K(\log K)^{3/2}} \bigg)\bigg\rbrace\\
%%%%%%%%%%%%%%%%
%&\E[\Ls(T)] \leq K\left\lceil\log_2\frac{T}{e}\right\rceil\bigg\lbrace\exp\bigg( -\frac{1}{i\max_{i}\Delta_{i}^{-2}}.\frac{T\log (2 K\sqrt{\log K})}{2K (\log K)^{3/2}}\bigg)\\
%& + \exp\bigg(- \frac{3}{i\max_i(6\sigma_{i}^{2}+ \Delta_i)\Delta_{i}^{-2}}.\frac{5T\log ( K\sqrt{\log K})}{ K(\log K)^{3/2}} \bigg)\bigg\rbrace\\
%%%%%%%%%%%%%%%%
%&\E[\Ls(T)] \leq K\left(\log_2\frac{T}{e}+1\right)\bigg\lbrace\exp\bigg( -\frac{T\log (2 K\sqrt{\log K})}{2 H_2 K (\log K)^{3/2}}\bigg)\\
%& + \exp\bigg(- \frac{5T\log ( K\sqrt{\log K})}{H_{2}^{\sigma} K(\log K)^{3/2}} \bigg)\bigg\rbrace\\
%\end{align*}
%\end{small}
%%%%%%%%%%%%%%%%%%%%%%%%%%%%%%%%%%%%%%%%%%%%%%%%%%%%%%%%%%%%%%%%%%%%%%%%%%%%%%%%%%%%%%
%Not Required for probability of error for AugUCB
%%%%%%%%%%%%%%%%%%%%%%%%%%%%%%%%%%%%%%%%%%%%%%%%%%%%%%%%%%%%%%%%%%%%%%%%%%%%%%%%%%%%%%

%Hence, for the $i$-th arm we can bound the probability of it getting eliminated till $\max\lbrace m_i , g_i  \rbrace$-th round by,
%%\begin{small}
%\begin{align*}
% & \Pb\lbrace \text{$i\in \mathcal{A}^{'}$ getting eliminated on or before round $\max\lbrace m_i, g_i\rbrace$} \rbrace \\
%&\geq 1-(\Pb\lbrace |\hat{r}_i -r_i| > 2c_i \rbrace + \Pb\lbrace |\hat{r}_i -r_i| > 2s_i \rbrace)\\
%&\geq 1- \bigg( \left(2\exp\left(-\frac{T\rho_{\mu}\epsilon_{m_{i}}}{32 a^2}\log ( T\epsilon_{m_{i}})\right)\right)\\
%& + 4\exp\left(- \frac{3T\rho_v\epsilon_{g_{i}}}{256 a^2 } \left(\frac{\sigma_{i}^{2}+\sqrt{\rho_{v}\epsilon_{g_{i}}}+1}{3\sigma_{i}^{2}+\sqrt{\rho_v \epsilon_{g_{i}}}}\right) \log( T\epsilon_{g_{i}}) \right)\bigg)
%\end{align*}
%%\end{small}
%Now, in the $m_i$-th round or in the $g_i$-th round we know that $\frac{\Delta_i}{4}\leq\sqrt{\epsilon_{m_{i}}\rho_{\mu}}<\frac{\Delta_i}{2}$ or  $\frac{\Delta_i}{4}\leq\sqrt{\epsilon_{g_{i}}\rho_{v}}<\frac{\Delta_i}{2}$.
%%\begin{small}
%\begin{align*}
%&\Pb\lbrace \text{$i\in \mathcal{A}^{'}$ getting eliminated on or before round $\max\lbrace m_i, g_i\rbrace$} \rbrace\\
%%%%%%%%%%%%%%%%%%%%%%%%%%%%%%%%%%%%%%%%%%%%%%%%%%%%%%%
%& \geq 1- \bigg( 2\exp\left(-\frac{T\rho_{\mu}\frac{\Delta_{i}^{2}}{16\rho_{\mu}}}{32 a^2 }\log ( T\frac{\Delta_{i}^{2}}{16\rho_{\mu}})\right)\\
%& + 4\exp\left(- \frac{3T\rho_v\frac{\Delta_{i}^{2}}{16\rho_{v}}}{256 a^2} \left(\frac{\sigma_{i}^{2}+\frac{\Delta_{i}}{4}+1}{3\sigma_{i}^{2}+\frac{\Delta_{i}}{4}}\right) \log( T\frac{\Delta_{i}^{2}}{16\rho_{v}}) \right)\bigg)\\
%%%%%%%%%%%%%%%%%%%%%%%%%%%%%%%%%%%%%%%%%%%%%%%%%%%%%%%	
%&\geq 1-\bigg( 2\exp\left(-\frac{T\Delta_{i}^{2}}{64a}\log( \frac{T\Delta_{i}^{2}}{2})\right) \\
%& + 4\exp\left(- \frac{3T\Delta_{i}^{2}}{4096 a^2} \left(\frac{4\sigma_{i}^{2}+\Delta_{i}+4}{12\sigma_{i}^{2}+\Delta_{i}}\right) \log( \frac{3}{16} T\Delta_{i}^{2}) \right)\bigg),\\
%&\text{putting the values of $\rho_{\mu}$ and $\rho_{v}$.}
%\end{align*}
%%\end{small}
%Again, $\Pb\lbrace \xi_1 \cup \xi_2 \rbrace\geq 1- \sum_{i=1}^{K}\sum_{m=0}^{\max\lbrace m_{i} ,g_{i}\rbrace}\Pb\lbrace i\in \mathcal{A}^{'}$ getting eliminated on or before round $\max\lbrace m_i, g_i\rbrace \rbrace $.
%Also, $\E[\Ls(T)]\le 1- \Pb\lbrace \xi_1 \cup \xi_2 \rbrace $. We know from \cite{bubeck2011pure} and \cite{auer2010ucb} that the function $x\in [0,1]\mapsto x\exp(-Cx^2)$ is  decreasing on $[1/\sqrt{2C},1]$ for any $C>0$. So, taking $C=\lfloor \sqrt{e/T}\rfloor$ and putting $\min_{i\in \mathcal{A}}\Delta_i =\Delta =\sqrt{\frac{K\log K}{T}} > \sqrt{\frac{e}{T}},\forall i\in \mathcal{A}$ we get that,
%%and summing over all arms $K$ and over all rounds $m=0,1,2,..,\max\lbrace m_{i} ,g_{i}\rbrace$
%%\begin{small}
%\begin{align*}
%&\E[\Ls(T)] \leq \sum_{i=1}^{K}\sum_{m=0}^{\max\lbrace m_{i} ,g_{i}\rbrace}\bigg\lbrace \bigg( 2\exp\left(-\frac{T\Delta_{i}^{2} \log(\frac{T\Delta_{i}^{2}}{2})}{64 a^2 }\right) \\
%& + 4\exp\left(- \frac{3T\Delta_{i}^{2}}{4096 a^2 } \left(\frac{4\sigma_{i}^{2}+\Delta_{i}+4}{12\sigma_{i}^{2}+\Delta_{i}}\right) \log( \frac{3}{16} T\Delta_{i}^{2}) \right)\bigg\rbrace\\
%%%%%%%%%%%%%%%%%
%& \leq K\sum_{m=0}^{M}\bigg\lbrace 2\exp\bigg( -\frac{T}{\min_{i}i\Delta_{(i)}^{-2}}.\frac{\log (\frac{1}{2} K\log K)}{64 a^2 }\bigg)\\
%& + 4\exp\bigg(- \frac{12T\Delta_{i}^{2}}{(12\sigma_{i}+ 12\Delta_{i})}.\frac{\log (\frac{3}{16} K\log K)}{4096 a^2 } \bigg)\bigg\rbrace\\
%%%%%%%%%%%%%%%%
%&\leq K\left(\log_2\frac{T}{e}+1\right)\bigg\lbrace\exp\bigg( -\frac{T\log ( \frac{1}{2} K\log K)}{ 64 H_2 a^2}\bigg)\\
%& + 2\exp\bigg(- \frac{T\Delta_{i}^{2}\log ( \frac{3}{16} K\log K)}{4096 (\sigma_{i} + \sqrt{\sigma_{i}^{2} + (16/3)\Delta_{i}}) a^2} \bigg)\bigg\rbrace\\
%%%%%%%%%%%%%%%%
%&\leq K\left(\log_2\frac{T}{e}+1\right)\bigg\lbrace\exp\bigg( -\frac{T\log ( \frac{1}{2} K\log K)}{ 64 H_2 (\log(\frac{3}{16} K\log K))^{2}}\bigg)\\
%& + 2\exp\bigg(- \frac{T\log ( \frac{3}{16} K\log K)}{4096 \min_{i}i\tilde{\Delta}_{(i)}^{-2} (\log(\frac{3}{16} K\log K))^{2}} \bigg)\bigg\rbrace\\
%%%%%%%%%%%%%%%%
%&\leq K\left(\log_2\frac{T}{e}+1\right)\bigg\lbrace\exp\bigg( -\frac{T}{ 64 H_2 (\log(\frac{3}{16} K\log K))}\bigg)\\
%& + 2\exp\bigg(- \frac{T}{4096 H_{2}^{\sigma} (\log(\frac{3}{16} K\log K))} \bigg)\bigg\rbrace\\
%\end{align*}
%\end{small}
\end{proof}

%	Next we specialize the result of Theorem \ref{Result:Theorem:1} in Corollary \ref{Result:Corollary:1}.
%
%\subsection{Corollary 2}
%
%
%\begin{corollary}
%\label{Result:Corollary:1}
%For $c_{0}=\sqrt{T}$, $\psi=\frac{T}{\log (K)}$, $\rho_{\mu}=\frac{1}{8}$ and $\rho_v=\frac{2}{3}$, the simple regret of AugUCB is given by,
%\begin{small}
%\begin{align*}
%& SR_{AugUCB} \leq \sum_{i=1}^{K} \Delta_{i}\bigg\lbrace\exp\bigg(-\log ( 2T\frac{\Delta_{i}^{2}}{\sqrt{\log K}})-\dfrac{T}{2 H_{2}}\\
%& + \log \big( \dfrac{4\gamma K\log ( 2T \frac{\Delta_{i}^{2}}{\sqrt{\log K}})}{T\Delta_{i}^{2}}\log_{2}\dfrac{T}{e} \big) \bigg)\\
%& +  \exp\bigg(- \bigg(\dfrac{2\sigma_{i}^{2}+\Delta_{i}+2}{6\sigma_{i}^{2}+\Delta_{i}}\bigg)\log( 3T\frac{\Delta_{i}^{2}}{8\sqrt{\log K}}) -\dfrac{3T}{32 H_{2}}\\
%& + \log\big ( \dfrac{64\gamma K\log ( 3T \frac{\Delta_{i}^{2}}{8\sqrt{\log K}})}{3T\Delta_{i}^{2}}\log_{2}\dfrac{T}{e} \big)  \bigg)\bigg\rbrace
%\end{align*}
%\end{small}
%\end{corollary}
%
%\begin{proof}
%Putting $c_{0}=\sqrt{T}$, $\psi=\frac{T}{\log (K)}$, $\rho_{\mu}=\frac{1}{8}$ and $\rho_v=\frac{2}{3}$ in the result obtained in Theorem \ref{Result:Theorem:1}, we get
%\begin{small}
%\begin{align*}
%& SR_{AugUCB} \leq \sum_{i=1}^{K} \Delta_{i}\bigg\lbrace \exp\bigg(-4\rho\log (\psi T\frac{\Delta_{i}^{4}}{16\rho^{2}})-\dfrac{c_{0}\sqrt{T}}{16\rho H_{2}}\\
%& + \log \big( 16\gamma C_1\log_{2}\dfrac{T}{e} \big) \bigg) + \exp\bigg(- \dfrac{3\rho_v}{2} \bigg(\dfrac{2\sigma_{i}^{2}+\Delta_{i}+2}{6\sigma_{i}^{2}+\Delta_{i}}\bigg)\log(\psi T\frac{\Delta_{i}^{4}}{16\rho_{v}^{2}})\\
%& -\dfrac{c_{0}\sqrt{T}}{16\rho_v H_{2}} + \log\big ( 32\gamma C_2\log_{2}\dfrac{T}{e} \big)  \bigg)\bigg\rbrace\\
%%%%%%%%%%%%%%%%%%
%&\leq \sum_{i=1}^{K} \Delta_{i}\bigg\lbrace\exp\bigg(-\dfrac{1}{2}\log ( T^{2}\frac{4\Delta_{i}^{4}}{\log K})-\dfrac{T}{2 H_{2}}\\
%& + \log \big( \dfrac{2\gamma K\log ( T^{2} \frac{4\Delta_{i}^{4}}{\log K})}{T\Delta_{i}^{2}}\log_{2}\dfrac{T}{e} \big) \bigg)\\
%& + \exp\bigg(-  \bigg(\dfrac{2\sigma_{i}^{2}+\Delta_{i}+2}{6\sigma_{i}^{2}+\Delta_{i}}\bigg)\log( T^{2}\frac{\Delta_{i}^{4}}{16.\frac{4}{9}\log K}) -\dfrac{c_{0}\sqrt{T}}{16.\frac{2}{3} H_{2}}\\
%& + \log\big ( \dfrac{32\gamma\rho_v K\log ( T^{2} \frac{\Delta_{i}^{4}}{16.\frac{2}{9}\log K})}{T\Delta_{i}^{2}}\log_{2}\dfrac{T}{e} \big)  \bigg)\bigg\rbrace\\
%%%%%%%%%%%%%%%%%%
%&\leq \sum_{i=1}^{K} \Delta_{i}\bigg\lbrace\exp\bigg(-\log ( 2T\frac{\Delta_{i}^{2}}{\sqrt{\log K}})-\dfrac{T}{2 H_{2}}\\
%& + \log \big( \dfrac{4\gamma K\log ( 2T \frac{\Delta_{i}^{2}}{\sqrt{\log K}})}{T\Delta_{i}^{2}}\log_{2}\dfrac{T}{e} \big) \bigg)\\
%& +  \exp\bigg(- \bigg(\dfrac{2\sigma_{i}^{2}+\Delta_{i}+2}{6\sigma_{i}^{2}+\Delta_{i}}\bigg)\log( 3T\frac{\Delta_{i}^{2}}{8\sqrt{\log K}}) -\dfrac{3T}{32 H_{2}}\\
%& + \log\big ( \dfrac{64\gamma K\log ( 3T \frac{\Delta_{i}^{2}}{8\sqrt{\log K}})}{3T\Delta_{i}^{2}}\log_{2}\dfrac{T}{e} \big)  \bigg)\bigg\rbrace
%\end{align*} 
%\end{small}
%\end{proof}
%
%
%\vspace{-1em}
\vspace{-5mm}
\section{Numerical Experiments}
\label{expt}
\begin{figure}
    \centering
    \begin{tabular}{cc}
    \subfigure[0.32\textwidth][Experiment $1$: Experiment with Arithmetic Progression]
    {
    		\pgfplotsset{
		tick label style={font=\Huge},
		label style={font=\Huge},
		legend style={font=\Large},
		}
        \begin{tikzpicture}[scale=0.4]
      	\begin{axis}[
		xlabel={timestep},
		ylabel={Error Percentage},
		grid=major,
        %clip mode=individual,grid,grid style={gray!30},
        clip=true,
        %clip mode=individual,grid,grid style={gray!30},
  		legend style={at={(0.5,1.2)},anchor=north, legend columns=3} ]
      	% UCB
		\addplot table{results/budgetTestAP1/APT1_comp_subsampled.txt};
		\addplot table{results/budgetTestAP1/UA1_comp_subsampled.txt};
		\addplot table{results/budgetTestAP1/UCBEM1_comp_subsampled.txt};
		\addplot table{results/budgetTestAP1/UCBEMV1_comp_subsampled.txt};
		\addplot table{results/budgetTestAP1/AugUCB1_comp_subsampled.txt};
		%\addplot table{results/budgetTestAP/UA1.txt};
      	\legend{APT,Unif Alloc,UCBE($1$,UCBEV($1$),AugUCB}
      	\end{axis}
      	\end{tikzpicture}
  		\label{Fig:budgetExpt1}
    }
    %\\
    &
    \subfigure[0.32\textwidth][Experiment $2$: Experiment with $4$ Group Setting ]
    {
    		\pgfplotsset{
		tick label style={font=\Huge},
		label style={font=\Huge},
		legend style={font=\Large},
		}
        \begin{tikzpicture}[scale=0.4]
        \begin{axis}[
		xlabel={timestep},
		ylabel={Error Percentage},
        %clip mode=individual,grid,grid style={gray!30},
       	grid=major,
       	clip=true,
  		legend style={at={(0.5,1.2)},anchor=north, legend columns=3} ]
      	% UCB
		\addplot table{results/budgetTestGR1/APT1_comp_subsampled.txt};
		\addplot table{results/budgetTestGR1/UA1_comp_subsampled.txt};
		\addplot table{results/budgetTestGR1/UCBEM1_comp_subsampled.txt};
		\addplot table{results/budgetTestGR1/UCBEMV1_comp_subsampled.txt};
		\addplot table{results/budgetTestGR1/AugUCB1_comp_subsampled.txt};
		%\addplot table{results/budgetTestGP/UA1.txt};
        \legend{APT,Unif Alloc,UCBE($1$),UCBEV1$(1)$,AugUCB}
      	\end{axis}
      	\end{tikzpicture}
   		\label{Fig:budgetExpt2} 
    }
%    &
%    \subfigure[Experiment $3$: Threshold Bandit with $3$ Group Setting ]
%    {
%    		\pgfplotsset{
%		tick label style={font=\Large},
%		label style={font=\Large},
%		%legend style={font=\footnotesize}
%		}
%        \begin{tikzpicture}[scale=0.5]
%        \begin{axis}[
%		xlabel={timestep},
%		ylabel={Error Percentage},
%        %clip mode=individual,grid,grid style={gray!30},
%		grid=major,
%		clip=true,
%  		legend style={at={(0.5,-0.2)},anchor=north, legend columns=3} ]
%        % UCB
%		\addplot table{results/budgetTestGR/APT1.txt};
%		\addplot table{results/budgetTestGR/AugUCB1.txt};
%		\addplot table{results/budgetTestGR/UCBE_1_41.txt};
%		\addplot table{results/budgetTestGR/UCBE_11.txt};
%		\addplot table{results/budgetTestGR/UCBE_2561.txt};
%		\addplot table{results/budgetTestGR/UA1.txt};
%        \legend{APT,AugUCBE,UCBE($0.25$),UCBE(1),UCBE(256),Unif Alloc}
%      	\end{axis}
%      	\label{Fig:budgetExpt3}
%        \end{tikzpicture}
%    }
    \end{tabular}
    \caption{Experiments with thresholding bandit}
    \label{fig:budgetExpt}
\end{figure}


	In this section we compare the empirical performance of AugUCB against APT, Uniform Allocation, UCBE and UCBEV algorithm. The threshold $\tau$ is set at $0.5$ for all experiments. Each algorithm is run independently $500$ times for $15000$ timesteps and the output set of arms suggested by the algorithms at every timestep is recorded. The output is considered erroneous if the correct set of arms is not $i=\lbrace 6,7,8,9,10 \rbrace$ (true for all the experiments). The error percentage over $500$ runs is plotted against $15000$ timesteps. For the uniform allocation algorithm, for each $t=1,2,..,T$ the arms are sampled uniformly. For UCBE algorithm  (\cite{audibert2009exploration}) which was built for single best arm identification, we modify it according to \cite{locatelli2016optimal} to suit the goal of finding arms above the threshold $\tau$. So the exploration parameter $a$ in UCBE is set to $a_{i}=\frac{T-K}{H_1}$. Then for each timestep $t=1,2,..,T$ we pull the arm that maximizes $\lbrace |\hat{r}_{i} -\tau|-\sqrt{\frac{a_{i}}{n_{i}}} \rbrace$, where $n_{i}$ is the number of times the arm $i$ is pulled till $t-1$ timestep. Also, APT is run with $\epsilon=0$, which denotes the precision with which the algorithm suggests the best set of arms. So when $\epsilon$ is  set to $0$ APT has to suggest the exact set of arms above the threshold. For AugUCB we take $\psi=\frac{T}{\log K}$, $\rho=\frac{1}{8}$ and $\rho_v=\frac{2}{3}$ as in Corollary \ref{Result:Corollary:1}.
	
	The first experiment is conducted on a testbed of $100$ arms involving Gaussian reward distribution with expected rewards of the arms $r_{1:4}=0.2+(0:3)*0.05$, $r_{5}=0.45$, $r_{6}=0.55$, $r_{7:10}=0.65+(0:3)*0.05$ and $r_{11:100}$=0.4. The means of first $10$ arms are set as arithmetic progression. Variance is set as $\sigma_{i=1:6 \text{ and } 11:100}^{2}=0.5$ and $\sigma_{i=7:10}=0.6$. The means in the testbed are chosen in such a way that any algorithm has to spend a significant amount of budget to explore all the arms and variance is chosen in such a way that the arms above $\tau$ have high variance. In this experiment we see that AugUCB performs better than all the other algorithms mentioned. Only UCBE($1$) and UCBEV($1$) beats AugUCB and that is because they have access to the exact problem complexity. The result is shown in Figure \ref{Fig:budgetExpt1}.
	
	The second experiment is conducted on a testbed of $100$ arms with the means divided into $4$ groups. Again the testbed involves Gaussian reward distributions with expected rewards of the arms as $r_{1:3}=0.1$, $r_{4:7}=\lbrace 0.35,0.45,0.55,0.65\rbrace$, $r_{8:10}=0.9$ and  $r_{11:100}=0.4$. Also $\sigma_{i=1:7 \text{ and } 11:100}^{2}=0.5$ and $\sigma^{2}_{i=8:10}=0.6$. AugUCB, APT, Uniform Allocation, UCBE and UCBEV with the same settings as experiment $1$ are run on this testbed. The result is shown in Figure \ref{Fig:budgetExpt2}. Here, also we see that AugUCB beats APT.  
	
%	The second experiment is conducted on a testbed of $10$ arms with the means set as Geometric Progression. The testbed involves Bernoulli reward distribution with expected rewards of the arms as $r_{1:4}=0.4-(0.2)^{1:4}$, $r_{5}=0.45$, $r_{6}=0.55$ and $r_{7:10}=0.6+(0.2)^{5-(1:4)}$. AugUCB, APT, Uniform Allocation and UCBE with the same settings  as experiment $1$ are run on this testbed. The result is shown in Figure \ref{Fig:budgetExpt2}. Here, we see that AugUCB beats APT with only UCBE($1$) performing at par with AugUCB. 

%\begin{figure}
%\centering
  %\begin{tabular}{c}
  %&
  %%%%%%Expt4
  %\begin{subfigure}{0.45\textwidth}
 %\tabl{c}{\scalebox{0.7}{
% \begin{tikzpicture}
%      \begin{axis}[
%	xlabel={timestep},
%	ylabel={Error Percentage},
%       clip mode=individual,grid,grid style={gray!30},
%  legend style={at={(0.5,-0.2)},anchor=north, legend columns=4} ]
%      % UCB
%\addplot table[x index=0,y index=1,col sep=tab,each nth point={10}] {results/budgetTestGR/APT.txt};
%\addplot table[x index=0,y index=1,col sep=tab,each nth point={10}] {results/budgetTestGR/AugUCB.txt};
%\addplot table[x index=0,y index=1,col sep=tab,each nth point={10}] {results/budgetTestGR/UA.txt};
%\addplot table[x index=0,y index=1,col sep=tab,each nth point={10}] {results/budgetTestGR/UCBE_1.txt};
%\addplot table[x index=0,y index=1,col sep=tab,each nth point={10}] {results/budgetTestGR/UCBE_256.txt};
%\addplot table[x index=0,y index=1,col sep=tab,each nth point={10}] {results/budgetTestGR/UCBE_1_4.txt};
%      \legend{APT,AugUCBE,Unif Alloc,UCBE(1),UCBE(256),UCBE($\frac{1}{4}$)}
%      %\legend{ClusUCB (NC, p=1),ClusUCB (C, p=4),ClusUCB(C, p=10) ,MOSS, ClusUCB(C, p=5, NAE), ClusUCB(C, p=10, NAE)}
%      %\legend{ClusUCB(1,A),ClusUCB(4,B),ClusUCB(10,B), MOSS,ClusUCB(5,S), ClusUCB(10,A)}
%      \end{axis}
%      \end{tikzpicture}
%      %}\\}
%			\caption{Experiment $3$: Threshold Bandit with $3$ Group Setting }
%  \label{Fig:budgetExpt3}
  %\end{subfigure}
  %\end{tabular}
%\end{figure}

%	The third experiment is conducted on a testbed of $10$ arms with the means divided into $3$ groups. Again the testbed involves Bernoulli reward distribution with expected rewards of the arms as $r_{1:3}=0.1$, $r_{4:7}=\lbrace 0.35,0.45,0.55,0.65\rbrace$ and $r_{8:10}=0.9$. AugUCB, APT, Uniform Allocation and UCBE with the same settings as experiment $1$ are run on this testbed. The result is shown in Figure \ref{Fig:budgetExpt3}. Here, also we see that AugUCB beats APT.  

%
%\vspace{-1.2em}
\vspace{-1mm}
\section{Conclusion}
\label{conclusion}
To be written.

% Acknowledgments---Will not appear in anonymized version
%\acks{We thank a bunch of people.}

\clearpage
\newpage
\bibliographystyle{named}
\bibliography{ijcai17}

\clearpage
\newpage
\section{Appendix}
\label{appendix}
\appendix



\end{document}

