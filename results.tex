\subsection{Problem Complexity}

We define problem complexity as,
\begin{align*}
H_{1} = \sum_{i=1}^{K}\dfrac{1}{\Delta_{i}^{2}} \text{ ,   } H_{2}=\max_{i\in A}\dfrac{i}{{\Delta_{i}^{2}}} \text{, where } \Delta_{i}=|r_{i}-\tau|
\end{align*}
This is same as the problem complexity defined in \cite{locatelli2016optimal} for the thresholding bandit problem and is similar to the problem complexity defined in \cite{audibert2010best} for single best arm identification. Also we know that,
\begin{align*}
H_{2}\leq H_{1}\leq \log(2K)H_{2}
\end{align*}

Also, we define $H_{1}^{\sigma}$ (\cite{gabillon2011multi}) and $H_{2}^{\sigma}$ as,
\begin{align*}
& H_{1}^{\sigma}=\sum_{i=1}^{K}\frac{\sigma_{i}+\sqrt{\sigma_{i}^{2}+(16/3)\Delta_{i}}}{\Delta_{i}^{2}}\\
& H_{2}^{\sigma}=\max_{i\in A} i\frac{6\sigma_{i}^{2} + \Delta_i}{6\Delta_i^{2}}
\end{align*}
which also gives us that $H_{2}^{\sigma} < H_{1}^{\sigma}$.


\subsection{Theorem 1}

\begin{theorem}
\label{Result:Theorem:1}
With $\psi=\frac{T\epsilon_{m}}{8 K\log K}$, $\rho_{\mu}=\frac{1}{8}$ and $\rho_v=\frac{1}{3}$,
the expected loss of the AugUCB algorithm is given by,
%\begin{small}
\begin{align*}
&\E[\Ls(T)] \leq 
%\exp\bigg( -\frac{T\log (2 K\sqrt{\log K})}{2H_2 K (\log K)^{3/2}} + \log\bigg(K\big(\log_2\frac{T}{e}+1\big)\bigg)\bigg)\\
%& + \exp\bigg(- \frac{5T\log ( K\sqrt{\log K})}{H_{2}^{\sigma} K(\log K)^{3/2}}  + \log\bigg(K\big(\log_2\frac{T}{e}+1\big)\bigg)\bigg).
 \exp\bigg( -\frac{T\log ( 2K\log K)}{ 16 H_2 K\log K} + \log\bigg(K\big(\log_2\frac{T}{e}+1\big)\bigg)\bigg)\\
& + \exp\bigg(- \frac{T\log ( \frac{3}{4} K\log K)}{64H_{2}^{\sigma} K(\log K)}
 + 2\log\bigg(K\big(\log_2\frac{T}{e}+1\big)\bigg) \bigg)  .
\end{align*}
%\end{small}
%For every $0<\eta <1$ and $\gamma > 1$, there exists time $t$ such that for all $T>t$ the simple regret of AugUCB is upper bounded by,
%\begin{small}
%\begin{align*}
%& SR_{AugUCB} \leq \sum_{i=1}^{K} \Delta_{i}\bigg\lbrace \exp\bigg(-4\rho\log (\psi T\frac{\Delta_{i}^{4}}{16\rho^{2}})-\dfrac{c_{0}\sqrt{T}}{16\rho H_{2}}\\
%& + \log \big( 16\gamma C_1\log_{2}\dfrac{T}{e} \big) \bigg) + \exp\bigg(- \dfrac{3\rho_v}{2} \bigg(\dfrac{2\sigma_{i}^{2}+\Delta_{i}+2}{6\sigma_{i}^{2}+\Delta_{i}}\bigg)\log(\psi T\frac{\Delta_{i}^{4}}{16\rho_{v}^{2}})\\
%& -\dfrac{c_{0}\sqrt{T}}{16\rho_v H_{2}} + \log\big ( 32\gamma C_2\log_{2}\dfrac{T}{e} \big)  \bigg)\bigg\rbrace
%\end{align*}
%\end{small}
%with probability at least $1-\eta$, where $c_{0}>0$ is a constant and $C_1=\dfrac{K\rho\log (\psi T \frac{\Delta_{i}^{4}}{16\rho^{2}})}{T\Delta_{i}^{2}}$ and $C_2= \dfrac{K\rho_v\log (\psi T \frac{\Delta_{i}^{4}}{16\rho_{v}^{2}})}{T\Delta_{i}^{2}}$.
\end{theorem}

\begin{proof}
According to the algorithm, the number of rounds is $m=\lbrace 0,1,2,.. M\rbrace $ where $M=\bigg\lfloor \frac{1}{2}\log_{2} \frac{T}{e}\bigg\rfloor$. So, $\epsilon_{m}\geq 2^{-M}\geq \sqrt{\frac{e}{T}}$. Also each round $m$ consists of $|B_{m}|\ell_{m}$ timesteps where $\ell_{m} = \left\lceil\frac{2\psi\log( T \epsilon_{m})}{\epsilon_{m}}\right\rceil$ and $B_{m}$ is the set of all surviving arms. 

Let $c_{i} = \sqrt{\frac{\rho_{\mu}\psi \log{(T\epsilon_{m})}}{2 n_{i}}}$ denote the confidence interval, where $n_{i}$ is the number of times an arm $i$ is pulled. Let $A^{'}=\lbrace i\in A|\Delta_{i}\geq b\rbrace$, for $b\geq \sqrt{\frac{e}{T}}$. Define $m_{i}=\min\lbrace m| \sqrt{\rho_{\mu}\epsilon_{m}}<\frac{\Delta_{i}}{2}\rbrace$.
% Let $m_{i}$ be the minimum round such that an arm $i$ gets eliminated such that. 

Let $s_{i}=\sqrt{\frac{\rho_v\psi \hat{V_{i}} \log{( T\epsilon_{g})}}{4 n_{i}} + \frac{\rho_v\psi \log{( T\epsilon_{g})}}{4 n_{i}}}$ and 
% $g_{i}$ be the minimum round that an arm $i$ gets eliminated such that 
$g_{i}=min\lbrace g| \sqrt{\rho_{v}\epsilon_{g}}<\frac{\Delta_{i}}{2}\rbrace$. 
%In this proof sub-optimal arms refer to the arms whose $r_{i}$ is lower than the threshold $\tau$.

%At the end of any round $\max\lbrace m_{i},g_{i}\rbrace$, for any arm $i$, two cases are possible.

Let $\xi_{1}$ and $\xi_{2}$ be the favorable event such that,
\begin{align*}
\xi_{1}&=\bigg\lbrace \forall i\in A, \forall m=0,1,2,..,M: |\hat{r_i} - r_i| \leq 2c_i\bigg\rbrace\\
\xi_{2}&=\bigg\lbrace \forall i\in A, \forall m=0,1,2,..,M: |\hat{r_i} - r_i| \leq  2s_i\bigg\rbrace
\end{align*}

So, $\xi_{1}$ and $\xi_{2}$ signifies the event till any arm $i$ will get eliminated from $B_m$.

\subsubsection{\textit{Arm i is not eliminated on or before round $\max\lbrace m_{i},g_{i}\rbrace$}}

For any arm $i$, if it is eliminated from active set $B_{m_{i}}$ then one of the below two events has to occur,
%\begin{small}
\begin{align}
\hat{r}_{i} + c_{i} < \tau - c_{i}, \label{eq:armelim-casea}\\
\hat{r}_{i} - c_{i} > \tau + c_{i}, \label{eq:armelim-caseb}
\end{align}
%\end{small}
For (\ref{eq:armelim-casea}) we can see that it eliminates arms that have performed poorly and removes them  from $B_{m_{i}}$. Similarly, (\ref{eq:armelim-caseb}) eliminates arms from $B_{m_{i}}$ that have performed very well compared to threshold $\tau$.

%Each round consists of $|B_{m_{i}}|\ell_{m_{i}}$ timesteps. 
In the $m_{i}$-th round an arm $i$ can be pulled no more than $\ell_{m_{i}}$ times. So when $n_{i}=\ell_{m_{i}}$, putting the value of $\ell_{m_{i}}\ge\frac{2\psi\log{( T\epsilon_{m_{i}})}}{\epsilon_{m_{i}}}$ in $c_{i}$ we get, 
%\begin{small}
\begin{align*}
c_{i}
&=\sqrt{\frac{\rho_{\mu}\psi\epsilon_{m_{i}}\log ( T\epsilon_{m_{i}})}{2 n_{i}}}
\le\sqrt{\frac{\rho_{\mu}\psi\epsilon_{i}\log ( T\epsilon_{m_{i}})}{2*2 \psi \log( T\epsilon_{m_{i}})}}\\
& \le\frac{\sqrt{\rho_{\mu}\epsilon_{m_{i}}}}{2}
% % \leq \sqrt{\rho_{\mu}\epsilon_{m_{i}+1}} 
< \frac{\Delta_{i}}{4} \text{, as }\rho_{\mu}\in (0,1].
\end{align*}
%\end{small}
Again, for ${i} \in A^{'}$ for the  elimination condition in (\ref{eq:armelim-casea}), 
%\begin{small}
%\begin{align*}
%\hat{r}_{i} + c_{i}&\leq r_{i} + 2c_{i} = r_{i} + 4c_{i} - 2c_{i} \\
%&< r_{i} + \Delta_{i} - 2c_{i} = \tau -2c_{i} \leq \tau - c_{i}
%\end{align*}
%\end{small}
%\begin{small}
\begin{align*}
\hat{r}_{i} &\leq r_{i} + 2c_{i} = r_{i} + 4c_{i} - 2c_{i} \\
&< r_{i} + \Delta_{i} - 2c_{i} = \tau -2c_{i}.
\end{align*}
%\end{small}
Similarly, for ${i} \in A^{'}$ for the  elimination condition in (\ref{eq:armelim-caseb}), 
%\begin{small}
\begin{align*}
\hat{r}_{i} &\geq r_{i} - 2c_{i} = r_{i} - 4c_{i} + 2c_{i} \\
&> r_{i} - \Delta_{i} + 2c_{i}= \tau + 2c_{i}.
\end{align*}
%\end{small}


%Now, arm elimination condition is being checked at every timestep, in the $m_{i}$-th round as soon as $n_{i}=\ell_{m_{i}}$, arm $i$ gets eliminated. 
Applying Chernoff-Hoeffding bound and considering independence of complementary of the event in (\ref{eq:armelim-casea}),
%\begin{small}
\begin{align*}
%\mathbb{P}\lbrace\hat{r}_{i}\geq r_{i} - 2c_{i}\rbrace &\leq exp(-2(\tau + 2c_{i})^{2}n_{i})\\
\mathbb{P}\lbrace\hat{r}_{i}> r_{i} + 2c_{i}\rbrace &\leq \exp(-4 c_{i}^{2}n_{i})\\
&\leq \exp(-8 * \dfrac{\rho_{\mu}\psi\log ( T\epsilon_{m_{i}})}{2 n_{i}} *n_{i})\\
&\leq \exp\big(-4\rho_{\mu}\psi\log ( T\epsilon_{m_{i}})\big)\\
&\leq \exp\left(-\rho_{\mu}\frac{T\epsilon_{m_{i}}}{2K\log K}\log ( T\epsilon_{m_{i}})\right),\\
&\text{putting the value of $\psi=\frac{T\epsilon_{m_i}}{8K\log K}$}
\end{align*}
%\end{small}
Similarly for the condition in (\ref{eq:armelim-caseb}), $\mathbb{P}\lbrace\hat{r}_{i}< r_{i} - 2c_{i}\rbrace\leq \exp\left(-\frac{T\rho_{\mu}\epsilon_{m_{i}}}{2K\log K}\log ( T\epsilon_{m_{i}})\right)$.

Summing the above two expressions, the probability that arm ${i}$ is not eliminated on or before $m_{i}$-th is $\left(2\exp\left(-\frac{4T\rho_{\mu}\epsilon_{m_{i}}}{8K\log K}\log ( T\epsilon_{m_{i}})\right)\right)$. 


Again for any arm $i$, if it is eliminated from active set $B_{g_{i}}$ then the below two events have to come true,
%\begin{small}
\begin{align}
\hat{r}_{i} + s_{i} < \tau - s_{i}, \label{eq:armelim-var-casea}\\
\hat{r}_{i} - s_{i} > \tau + s_{i}, \label{eq:armelim-var-caseb}
\end{align}
%\end{small}
%
% For \ref{eq:armelim-var-casea} we can see that it eliminates arms that have performed poorly and removes them them from $B_{g_{i}}$. Similarly, \ref{eq:armelim-var-caseb} eliminates arms from $B_{g_{i}}$ that have performed very well compared to threshold $\tau$.
%But, we know that $\epsilon_{m_{i}}=\epsilon_{g_{i}}$ and round consist of $|B_{g_{i}}|\ell_{g_{i}}$ timesteps. 
In the $g_{i}$-th round an arm $i$ can be pulled no more than $\ell_{g_{i}}$ times. So when $n_{i}=\ell_{g_{i}}$, putting the value of $\ell_{g_{i}}\ge\frac{2\psi\log{( T\epsilon_{g_{i}})}}{\epsilon_{g_{i}}}$ in $s_{i}$ we get, 
%\begin{small}
\begin{align*}
s_{i}&=\sqrt{\dfrac{\rho_v \psi \hat{V}_{i} \epsilon_{g_{i}}\log ( T\epsilon_{g_{i}})}{4 n_{i}} + \dfrac{\rho_v \psi\log{( T\epsilon_{g_{i}})}}{4 n_{i}}} \\
&\leq \sqrt{\dfrac{\rho_v\psi \epsilon_{g_{i}}\log ( T\epsilon_{g_{i}})}{4*2 \log(\psi T\epsilon_{g_{i}})} + \dfrac{\rho_v \psi\epsilon_{g_{i}} \log{( T\epsilon_{g_{i}})}}{4*2\psi \log( T\epsilon_{g_{i}})} } \text{, as }\hat{V}_{i}\in [0,1].\\
& \leq \sqrt{\dfrac{\rho_v \epsilon_{g_{i}}}{8} + \dfrac{\rho_v \epsilon_{g_{i}}}{8} } \leq \dfrac{\sqrt{\rho_v \epsilon_{g_{i}}}}{2}< \dfrac{\Delta_{i}}{4} \text{, as }\rho_v\in (0,1].
%& \leq \sqrt{\rho_v \epsilon_{g_{i}+1}} < \dfrac{\Delta_{i}}{4} \text{, as }\rho_v\in (0,1].
\end{align*}
%\end{small}

Again, for ${i} \in A^{'}$ for the elimination condition in (\ref{eq:armelim-var-casea}),
%\begin{small}
\begin{align*}
\hat{r}_{i} &\leq r_{i} + 2s_{i} = r_{i} + 4s_{i} - 2s_{i} \\
&< r_{i} + \Delta_{i} - 2s_{i} = \tau -2s_{i} % \leq \tau - s_{i}
\end{align*}
%\end{small} 


Also, for ${i} \in A^{'}$ for the elimination condition in (\ref{eq:armelim-var-caseb}), 
%\begin{small}
\begin{align*}
\hat{r}_{i}&\geq r_{i} - 2s_{i} = r_{i} - 4s_{i} + 2s_{i} \\
&> r_{i} - \Delta_{i} + 2s_{i}\geq \tau + 2s_{i} % \geq \tau + s_{i}
\end{align*}
%\end{small}


%Since, arm elimination condition is being checked at every timestep, in the $g_{i}$-th round as soon as $n_{i}=\ell_{g_{i}}$, arm $i$ gets eliminated. 
Applying Bernstein inequality and considering independence of complementary of the event in (\ref{eq:armelim-var-casea}),
%\begin{small}
\begin{align}
&\mathbb{P}\lbrace\hat{r}_{i}> r_{i} + 2s_{i}\rbrace\\
&\leq \mathbb{P}\bigg\lbrace \hat{r}_{i} > r_{i}+ ( 2\sqrt{\dfrac{\rho_v\psi \hat{V}_{i}\log( T\epsilon_{g_{i}}) + \rho_v\psi \log{( T\epsilon_{g_{i}})}}{4n_{i}} }) \bigg\rbrace\\
&\leq \mathbb{P}\bigg\lbrace \hat{r}_{i} > r_{i}+ (2\sqrt{\dfrac{\rho_v\psi [\sigma_{i}^{2}+\sqrt{\rho_{v}\epsilon_{g_{i}}} + 1]\log( T\epsilon_{g_{i}})}{4n_{i}}})\bigg\rbrace \label{eq:prob_eq1}\\ 
&+ \mathbb{P}\bigg\lbrace \hat{V}_{i}\geq \sigma_{i}^{2}+\sqrt{\rho_{v}\epsilon_{g_{i}}}\bigg\rbrace \label{eq:prob_eq2}
\end{align}
%\end{small}
 
 
Now, we know that in the $g_{i}$-th round,
%\begin{small}
\begin{align*}
& 2\sqrt{\dfrac{\rho_v\psi [\sigma_{i}^{2}+\sqrt{\rho_{v}\epsilon_{g_{i}}}]\log( T\epsilon_{g_{i}})}{4n_{i}} + \dfrac{\rho_v\psi  \log{(T\epsilon_{g_{i}})}}{4 n_{i}}}\\ &\leq  2\sqrt{\dfrac{\rho_v\psi [\sigma_{i}^{2}+\sqrt{\rho_{v}\epsilon_{g_{i}}}]\log( T\epsilon_{g_{i}})}{\frac{8\psi\log( T \epsilon_{g_{i}})}{\epsilon_{g_{i}}}} + \dfrac{\rho_v\psi \log{( T\epsilon_{g_{i}})}}{\frac{8\psi\log( T \epsilon_{g_{i}})}{\epsilon_{g_{i}}}}}\\
& \leq \dfrac{\sqrt{\rho_v \epsilon_{g_{i}}[\sigma_{i}^{2}+\sqrt{\rho_{v}\epsilon_{g_{i}}} + 1]}}{2}\leq \sqrt{\rho_v \epsilon_{g_{i}}}
\end{align*}
%\end{small}


For the term in (\ref{eq:prob_eq1}), by applying Bernstein inequality, we can write as,
%\begin{small}
\begin{align*}
&\mathbb{P}\bigg\lbrace \hat{r}_{i}> r_{i} + \bigg(2\sqrt{\frac{\rho_v\psi [\sigma_{i}^{2}+\sqrt{\rho_{v}\epsilon_{g_{i}}} + 1]\log( T\epsilon_{g_{i}})}{4n_{i}}  } \bigg)\bigg\rbrace\\
&\leq \exp\bigg(- \dfrac{\bigg(2\sqrt{\frac{\rho_v\psi [\sigma_{i}^{2}+\sqrt{\rho_{v}\epsilon_{g_{i}}}]\log( T\epsilon_{g_{i}})}{4n_{i}} + \frac{\rho_v\psi \log{( T\epsilon_{g_{i}})}}{4 n_{i}}}\bigg)^{2}n_{i}}{2\sigma_{i}^{2}+\frac{4}{3}\sqrt{\frac{\rho_v\psi [\sigma_{i}^{2}+\sqrt{\rho_{v}\epsilon_{g_{i}}}]\log( T\epsilon_{g_{i}})}{4n_{i}}+\frac{\rho_v\psi \log{( T\epsilon_{g_{i}})}}{4 n_{i}}}}\bigg) \\
&\leq \exp\bigg(- \dfrac{\bigg(\rho_v\psi [\sigma_{i}^{2}+\sqrt{\rho_{v}\epsilon_{g_{i}}} + 1]\log( T\epsilon_{g_{i}})\bigg)}{2\sigma_{i}^{2}+\frac{2}{3}\sqrt{\rho_v \epsilon_{g_{i}}}} \bigg)\\
&\leq \exp\bigg(- \dfrac{3\rho_v\psi}{2} \bigg(\dfrac{\sigma_{i}^{2}+\sqrt{\rho_{v}\epsilon_{g_{i}}}+1}{3\sigma_{i}^{2}+\sqrt{\rho_v \epsilon_{g_{i}}}}\bigg) \log( T\epsilon_{g_{i}}) \bigg)\\
&\leq \exp\left(- \dfrac{3\rho_v T\epsilon_{g_i}}{16 K\log K} \left(\dfrac{\sigma_{i}^{2}+\sqrt{\rho_{v}\epsilon_{g_{i}}}+1}{3\sigma_{i}^{2}+\sqrt{\rho_v \epsilon_{g_{i}}}}\right) \log( T\epsilon_{g_{i}}) \right),\\
&\text{ putting the value of $\psi=\frac{T\epsilon_{m_i}}{8K\log K}$}
\end{align*}
%\end{small}
 
  
For the term in (\ref{eq:prob_eq2}), by applying Bernstein inequality, we can write as,
%\begin{small}
\begin{align*}
&\mathbb{P}\bigg\lbrace \hat{V}_{i}\geq \sigma_{i}^{2}+\sqrt{\rho_{v}\epsilon_{g_{i}}}\bigg\rbrace\\
&\leq \mathbb{P}\bigg\lbrace \dfrac{1}{n_{i}}\sum_{t=1}^{n_{i}}(x_{i,t}-r_{i})^{2}-(\hat{r}_{i}-r_{i})^{2}\geq \sigma_{i}^{2}+\sqrt{\rho_{v}\epsilon_{g_{i}}}\bigg\rbrace\\
&\leq \mathbb{P}\bigg\lbrace \dfrac{\sum_{t=1}^{n_{i}}(x_{i,t}-r_{i})^{2}}{n_{i}}\geq \sigma_{i}^{2}+\sqrt{\rho_{v}\epsilon_{g_{i}}} \bigg\rbrace\\
&\leq \mathbb{P}\bigg\lbrace \dfrac{\sum_{t=1}^{n_{i}}(x_{i,t}-r_{i})^{2}}{n_{i}}\geq \sigma_{i}^{2} +\\
&\bigg(2\sqrt{\dfrac{\rho_v\psi [\sigma_{i}^{2}+\sqrt{\rho_{v}\epsilon_{g_{i}}}]\log( T\epsilon_{g_{i}})}{4n_{i}}+\frac{\rho_v\psi  \log{(T\epsilon_{g_{i}})}}{4 n_{i}}}\bigg)\bigg\rbrace\\
&\leq \exp\bigg(- \dfrac{3\rho_v\psi}{2} \bigg(\dfrac{\sigma_{i}^{2}+\sqrt{\rho_{v}\epsilon_{g_{i}}}+1}{3\sigma_{i}^{2}+\sqrt{\rho_v \epsilon_{g_{i}}}}\bigg) \log( T\epsilon_{g_{i}}) \bigg) \\
&\leq \exp\bigg(- \dfrac{3\rho_vT\epsilon_{g_i}}{16K\log K \epsilon_{g_{i}}} \bigg(\dfrac{\sigma_{i}^{2}+\sqrt{\rho_{v}\epsilon_{g_{i}}}+1}{3\sigma_{i}^{2}+\sqrt{\rho_v \epsilon_{g_{i}}}}\bigg) \log( T\epsilon_{g_{i}}) \bigg),\\
&\text{ putting the value of $\psi=\frac{T\epsilon_{m_i}}{8K\log K}$}
\end{align*}
%\end{small}
 
  
Similarly, the condition for the complementary event for the elimination case \ref{eq:armelim-var-caseb} holds such that $\mathbb{P}\lbrace\hat{r}_{i}< r_{i} - 2s_{i}\rbrace \leq 2\exp\left(- \frac{3T\rho_v\epsilon_{g_{i}}}{16K\log K} \left(\frac{\sigma_{i}^{2}+\sqrt{\rho_{v}\epsilon_{g_{i}}}+1}{3\sigma_{i}^{2}+\sqrt{\rho_v \epsilon_{g_{i}}}}\right) \log( T\epsilon_{g_{i}}) \right)$.

Again  summing the above expressions, the probability that an arm ${i}$ is not eliminated on or before $g_{i}$-th round based on the (\ref{eq:armelim-var-casea}) and (\ref{eq:armelim-var-caseb}) elimination condition is  $4\exp\left(- \frac{3T\rho_v\epsilon_{g_{i}}}{16K\log K} \left(\frac{\sigma_{i}^{2}+\sqrt{\rho_{v}\epsilon_{g_{i}}}+1}{3\sigma_{i}^{2}+\sqrt{\rho_v \epsilon_{g_{i}}}}\right) \log( T\epsilon_{g_{i}}) \right)$. 
  
%%%%%%%%%%%%%%%%%%%%%%%%%%%%%%%%%%%%%%%%%%%%%%%%%%%%%%%%%%%%%%%%%%%%%%%%%%%%%%%%%%%%%%
%Not Required for probability of error for AugUCB
%%%%%%%%%%%%%%%%%%%%%%%%%%%%%%%%%%%%%%%%%%%%%%%%%%%%%%%%%%%%%%%%%%%%%%%%%%%%%%%%%%%%%%

%We start with an upper bound on the number of plays $\delta_{\max\lbrace m_{i}, g_{i}\rbrace}$ in the $\max\lbrace m_{i}, g_{i}\rbrace$-th round. We know that the total number of arms surviving in the $\max\lbrace m_{i}, g_{i}\rbrace$-th arm is, 
%
%\begin{small}
%\begin{align*}
%&|B_{\max\lbrace m_{i}, g_{i}\rbrace}|=2K\exp\bigg(-4\rho_{\mu}\log (\psi T\epsilon_{m_{i}}^{2})\bigg)\\ 
%& + 4K\exp\bigg(- \frac{3\rho_v}{2} \big(\frac{\sigma_{i}^{2}+\sqrt{\rho_{v}\epsilon_{g_{i}}}+1}{3\sigma_{i}^{2}+\sqrt{\rho_v \epsilon_{g_{i}}}}\big) \log(\psi T\epsilon_{g_{i}}^{2}) \bigg)
%\end{align*}     
%\end{small}
%
%
%Again for AugUCB, we know that the number of pulls allocated for each surviving arm $i$ in the $m_{i}$-th round is $\ell_{m_{i}}=\frac{2\log (\psi T \epsilon_{m_{i}}^{2})}{\epsilon_{m_{i}}}$ or for the $g_{i}$-th round is $\ell_{g_{i}}=\frac{2\log (\psi T \epsilon_{g_{i}}^{2})}{\epsilon_{g_{i}}}$. Therefore, the proportion of plays $\delta_{\max\lbrace m_{i}, g_{i}\rbrace}$ in the $\max\lbrace m_{i}, g_{i}\rbrace$-th round can be written as,
%
%\begin{small}
%\begin{align*}
%&\delta_{\max\lbrace m_{i}, g_{i}\rbrace}=(|B_{m_{i}}|.\ell_{m_{i}}) + (|B_{g_{i}}|.\ell_{g_{i}})\\
%&\leq 2K\exp\bigg(-4\rho_{\mu}\log (\psi T\epsilon_{m_{i}}^{2})\bigg).\dfrac{2\log (\psi T \epsilon_{m_{i}}^{2})}{\epsilon_{m_{i}}}\\
% & + 4K\exp\bigg(- \dfrac{3\rho_v}{2} \bigg(\dfrac{\sigma_{i}^{2}+\sqrt{\rho_{v}\epsilon_{g_{i}}}+1}{3\sigma_{i}^{2}+\sqrt{\rho_v \epsilon_{g_{i}}}}\bigg) \log(\psi T\epsilon_{g_{i}}^{2})\bigg).\dfrac{2\log (\psi T \epsilon_{g_{i}}^{2})}{\epsilon_{g_{i}}} \\
%& \leq \dfrac{4K\log (\psi T \epsilon_{m_{i}}^{2})}{\epsilon_{m_{i}}}\exp\bigg(-4\rho_{\mu}\log (\psi T\epsilon_{m_{i}}^{2})\bigg)\\
%& + \dfrac{8K\log (\psi T \epsilon_{g_{i}}^{2})}{\epsilon_{g_{i}}}\exp\bigg(- \dfrac{3\rho_v}{2} \bigg(\dfrac{\sigma_{i}^{2}+\sqrt{\rho_{v}\epsilon_{g_{i}}}+1}{3\sigma_{i}^{2}+\sqrt{\rho_v \epsilon_{g_{i}}}}\bigg) \log(\psi T\epsilon_{g_{i}}^{2}) \bigg)
%\end{align*}
%\end{small}

%Now, in the $\max\lbrace m_{i}, g_{i}\rbrace$-th round $\sqrt{\rho_{\mu}\epsilon_{m_{i}}}\leq \frac{\Delta_{i}}{2}$ or $\sqrt{\rho_v\epsilon_{g_{i}}}\leq \frac{\Delta_{i}}{2}$. Hence,
%
%\begin{small}
%\begin{align*}
%&\delta_{\max\lbrace m_{i},g_{i}\rbrace} \leq \dfrac{4K\log (\psi T \frac{\Delta_{i}^{4}}{16\rho_{\mu}^{2}})}{\frac{\Delta_{i}^{2}}{4\rho_{\mu}}}\exp\bigg(-4\rho_{\mu}\log (\psi T\frac{\Delta_{i}^{4}}{16\rho_{\mu}^{2}})\bigg)\\
%& + \dfrac{8K\log (\psi T \frac{\Delta_{i}^{4}}{16\rho_{v}^{2}})}{\frac{\Delta_{i}^{2}}{4\rho_{v}}}\exp\bigg(- \dfrac{3\rho_v}{2} \bigg(\dfrac{\sigma_{i}^{2}+\frac{\Delta_{i}}{2}+1}{3\sigma_{i}^{2}+\frac{\Delta_{i}}{2}}\bigg) \log(\psi T\frac{\Delta_{i}^{4}}{16\rho_{v}^{2}}) \bigg)\\
%%%%%%%%%%%%%%%%%%%%%%%%%%%%%%%%%%%%%%%%
%&\leq 16 C_1\exp\bigg(-4\rho_{\mu}\log (\psi T\frac{\Delta_{i}^{4}}{16\rho_{\mu}^{2}})\bigg)\\
%& + 32C_2\exp\bigg(- \dfrac{3\rho_v}{2} \bigg(\dfrac{2\sigma_{i}^{2}+\Delta_{i}+2}{6\sigma_{i}^{2}+\Delta_{i}}\bigg) \log(\psi T\frac{\Delta_{i}^{4}}{16\rho_{v}^{2}}) \bigg)\\
%&\text{where $C_1=\frac{K\rho_{\mu}\log (\psi T \frac{\Delta_{i}^{4}}{16\rho_{\mu}^{2}})}{\Delta_{i}^{2}}$ and $C_2= \frac{K\rho_v\log (\psi T \frac{\Delta_{i}^{4}}{16\rho_{v}^{2}})}{\Delta_{i}^{2}}$}\\
%%%%%%%%%%%%%%%%%%%%%%%%%%%%%%%%%%%%%%%%
%&\leq 16 C_1\exp\bigg(-4\rho_{\mu}\log (\psi T\frac{\Delta_{i}^{4}}{16\rho^{2}})\bigg)
% + 32C_2\exp\bigg(- \dfrac{3\rho_v}{2} \log(\psi T\frac{\Delta_{i}^{4}}{16\rho_{v}^{2}}) \bigg)
%\end{align*}
%\end{small}
%
%%Summing over all rounds $m=0,1,..,M$,
%Now, putting the values of $\psi$, $\rho_{\mu}$, $\rho_v$ and taking $\Delta_{i}\geq\min_{i\in A}\Delta=\sqrt{\frac{K\log K}{T}}\geq \sqrt{\frac{e}{T}},\forall i\in A$( see \cite{auer2010ucb}), 
%
%\begin{small}
%\begin{align*}
%& \delta_{\max\lbrace m_{i}, g_{i}\rbrace}= \bigg\lbrace 16 C_1\exp\bigg(-4\rho_{\mu}\log (\psi T\frac{\Delta_{i}^{4}}{16\rho_{\mu}^{2}})\bigg)\\
%& + 32C_2\exp\bigg(- \frac{3\rho_v}{2} \log(\psi T\frac{\Delta_{i}^{4}}{16\rho_{v}^{2}}) \bigg) \bigg\rbrace\\
%%%%%%%%%%%%%%%%%%%%%
%&\leq \bigg\lbrace  \frac{2K\log ( T^2 \frac{4\Delta_{i}^{4}}{\log K})}{\Delta_{i}^{2}}\exp\bigg(-\frac{1}{2}\log ( T^2\frac{4\Delta_{i}^{4}}{\log K})\bigg)\\
%& + \frac{32K\log ( T^2 \frac{9\Delta_{i}^{4}}{\log K})}{3\Delta_{i}^{2}}\exp\bigg(- \frac{1}{2} \log( T^2 \frac{9\Delta_{i}^{4}}{\log K}) \bigg) \bigg\rbrace\\
%%%%%%%%%%%%%%%%%%%%%
%&\leq \bigg\lbrace  \frac{4K\log ( T \frac{2\Delta_{i}^{2}}{\sqrt{\log K}})}{\Delta_{i}^{2}}\exp\bigg(-\log ( T\frac{2\Delta_{i}^{2}}{\sqrt{\log K}})\bigg)\\
%& + \frac{64K\log ( T \frac{3\Delta_{i}^{2}}{\sqrt{\log K}})}{3\Delta_{i}^{2}}\exp\bigg(- \log( T \frac{3\Delta_{i}^{2}}{\sqrt{\log K}}) \bigg) \bigg\rbrace\\
%%%%%%%%%%%%%%%%%%%%%
%&\leq \bigg\lbrace  \frac{4KT\log ( \frac{2 K\log K}{\sqrt{\log K}})}{K\log K}\exp\bigg(-\log ( \frac{2K\log K}{\sqrt{\log K}})\bigg)\\
%& + \frac{64TK\log (\frac{3 K\log K}{\sqrt{\log K}})}{3 K\log K}\exp\bigg(- \log( \frac{3 K\log K}{\sqrt{\log K}}) \bigg) \bigg\rbrace\\
%%%%%%%%%%%%%%%%%%%%
%&\leq \bigg\lbrace  \frac{2T\log (2 K\sqrt{\log K})}{K (\log K)^{3/2}}
% + \frac{22T\log ( K\sqrt{\log K})}{ K(\log K)^{3/2}}\bigg) \bigg\rbrace\\
%\end{align*}
%\end{small}
%Now we know that till $m_i$-th round $2c_i > \frac{\Delta_i}{2}$  or till $g_i$ th round $2s_i > \frac{\Delta_i}{2}$. Hence, for the $i$-th arm we can bound the probability of error for any round $m$ by applying Chernoff-Hoeffding and Bernstein inequality,
%\begin{small}
%\begin{align*}
% \Pb\lbrace \xi_1\rbrace  + \Pb\lbrace \xi_2 \rbrace &\geq 1-(\Pb\lbrace |\hat{r}_i -r_i| > 2c_i \rbrace + \Pb\lbrace |\hat{r}_i -r_i| > 2s_i \rbrace)\\ 
%&\geq 1-\left(\Pb\lbrace |\hat{r}_i - r_i| > \frac{\Delta_i}{2} \rbrace + \Pb\lbrace |\hat{r}_i - r_i| > \frac{\Delta_i}{2} \rbrace\right) \\
%&\geq 1-\big(2\exp( -\frac{\Delta_{i}^{2}}{4}n_i ) + 2\exp(- \frac{\Delta_{i}^{2}}{8\sigma_{i}^{2}+ \frac{4}{3}\Delta_i}n_i)\big)\\
%&\geq 1-\bigg(2\exp( -\frac{\Delta_{i}^{2}}{4}\delta_{m_{i}} ) + 2\exp(- \frac{\Delta_{i}^{2}}{8\sigma_{i}^{2}+ \frac{4}{3}\Delta_i}\delta_{g_{i}})\bigg)
%\end{align*}
%\end{small}
%Now, we know that $\E[\Ls(T)]\le1- (\Pb\lbrace \xi_1\rbrace  + \Pb\lbrace \xi_2 \rbrace) $. Summing over all arms $K$ and over all rounds $m=0,1,2,..,M$ we get that,
%\begin{small}
%\begin{align*}
%&\E[\Ls(T)] \leq \sum_{i=1}^{K}\sum_{m=0}^{M}\bigg\lbrace 2\exp\bigg( -\frac{\Delta_{i}^{2}}{4}.\frac{2T\log (2 K\sqrt{\log K})}{K (\log K)^{3/2}}\bigg)\\
%& + 2\exp\bigg(- \frac{\Delta_{i}^{2}}{8\sigma_{i}^{2}+ \frac{4}{3}\Delta_i}.\frac{22T\log ( K\sqrt{\log K})}{ K(\log K)^{3/2}} \bigg)\bigg\rbrace\\
%%%%%%%%%%%%%%%%
%&\E[\Ls(T)] \leq K\left\lceil\log_2\frac{T}{e}\right\rceil\bigg\lbrace\exp\bigg( -\frac{1}{i\max_{i}\Delta_{i}^{-2}}.\frac{T\log (2 K\sqrt{\log K})}{2K (\log K)^{3/2}}\bigg)\\
%& + \exp\bigg(- \frac{3}{i\max_i(6\sigma_{i}^{2}+ \Delta_i)\Delta_{i}^{-2}}.\frac{5T\log ( K\sqrt{\log K})}{ K(\log K)^{3/2}} \bigg)\bigg\rbrace\\
%%%%%%%%%%%%%%%%
%&\E[\Ls(T)] \leq K\left(\log_2\frac{T}{e}+1\right)\bigg\lbrace\exp\bigg( -\frac{T\log (2 K\sqrt{\log K})}{2 H_2 K (\log K)^{3/2}}\bigg)\\
%& + \exp\bigg(- \frac{5T\log ( K\sqrt{\log K})}{H_{2}^{\sigma} K(\log K)^{3/2}} \bigg)\bigg\rbrace\\
%\end{align*}
%\end{small}
%%%%%%%%%%%%%%%%%%%%%%%%%%%%%%%%%%%%%%%%%%%%%%%%%%%%%%%%%%%%%%%%%%%%%%%%%%%%%%%%%%%%%%
%Not Required for probability of error for AugUCB
%%%%%%%%%%%%%%%%%%%%%%%%%%%%%%%%%%%%%%%%%%%%%%%%%%%%%%%%%%%%%%%%%%%%%%%%%%%%%%%%%%%%%%

Hence, for the $i$-th arm we can bound the probability of error till the event $\xi_1$ or $\xi_2$ by,
%\begin{small}
\begin{align*}
 & \Pb\lbrace \xi_1\rbrace  + \Pb\lbrace \xi_2 \rbrace \geq 1-(\Pb\lbrace |\hat{r}_i -r_i| > 2c_i \rbrace + \Pb\lbrace |\hat{r}_i -r_i| > 2s_i \rbrace)\\
&\geq 1- \bigg( \left(2\exp\left(-\frac{T\rho_{\mu}\epsilon_{m_{i}}}{2K\log K}\log ( T\epsilon_{m_{i}})\right)\right)\\
& + 4\exp\left(- \frac{3T\rho_v\epsilon_{g_{i}}}{16K\log K} \left(\frac{\sigma_{i}^{2}+\sqrt{\rho_{v}\epsilon_{g_{i}}}+1}{3\sigma_{i}^{2}+\sqrt{\rho_v \epsilon_{g_{i}}}}\right) \log( T\epsilon_{g_{i}}) \right)\bigg)
\end{align*}
%\end{small}
Now, in the $m_i$-th round or in the $g_i$-th round we know that $\sqrt{\epsilon_{m_{i}}\rho_{\mu}}<\frac{\Delta_i}{2}$ or  $\sqrt{\epsilon_{g_{i}}\rho_{v}}<\frac{\Delta_i}{2}$.
%\begin{small}
\begin{align*}
&\Pb\lbrace \xi_1\rbrace  + \Pb\lbrace \xi_2 \rbrace \geq 1- \bigg( 2\exp\left(-\frac{T\rho_{\mu}\frac{\Delta_{i}^{2}}{4\rho_{\mu}}}{2K\log K}\log ( T\frac{\Delta_{i}^{2}}{4\rho_{\mu}})\right)\\
& + 4\exp\left(- \frac{3T\rho_v\frac{\Delta_{i}^{2}}{4\rho_{v}}}{16 K\log K} \left(\frac{\sigma_{i}^{2}+\frac{\Delta_{i}}{2}+1}{3\sigma_{i}^{2}+\frac{\Delta_{i}}{2}}\right) \log( T\frac{\Delta_{i}^{2}}{4\rho_{v}}) \right)\bigg)\\
%%%%%%%%%%%%%%%%%%%%%%%%%%%%%%%%%%%%%%%%%%%%%%%%%%%%%%%
&\geq 1-\bigg( 2\exp\left(-\frac{T\Delta_{i}^{2}}{8K\log K}\log( 2T\Delta_{i}^{2})\right) \\
& + 4\exp\left(- \frac{3T\Delta_{i}^{2}}{16 K\log K} \left(\frac{2\sigma_{i}^{2}+\Delta_{i}+2}{6\sigma_{i}^{2}+\Delta_{i}}\right) \log( \frac{3}{4} T\Delta_{i}^{2}) \right)\bigg),\\
&\text{putting the values of $\rho_{\mu}$ and $\rho_{v}$.}
\end{align*}
%\end{small}
Now, we know that $\E[\Ls(T)]\le 1- (\Pb\lbrace \xi_1\rbrace  + \Pb\lbrace \xi_2 \rbrace) $. Summing over all arms $K$ and over all rounds $m=0,1,2,..,\max\lbrace m_{i} ,g_{i}\rbrace$ we get that,
%\begin{small}
\begin{align*}
&\E[\Ls(T)] \leq \sum_{i=1}^{K}\sum_{m=0}^{\max\lbrace m_{i} ,g_{i}\rbrace}\bigg\lbrace \bigg( 2\exp\left(-\frac{2T\Delta_{i}^{2}}{8K\log K}\log(2 T\Delta_{i}^{2})\right) \\
& + 4\exp\left(- \frac{3T\Delta_{i}^{2}}{64 K\log K} \left(\frac{2\sigma_{i}^{2}+\Delta_{i}+2}{6\sigma_{i}^{2}+\Delta_{i}}\right) \log( \frac{3}{4} T\Delta_{i}^{2}) \right)\bigg\rbrace\\
%%%%%%%%%%%%%%%%
& \leq K\sum_{m=0}^{M}\bigg\lbrace 2\exp\bigg( -\frac{T}{i\max_{i}\Delta_{i}^{-2}}.\frac{T\log (2 K\log K)}{16K \log K}\bigg)\\
& + 4\exp\bigg(- \frac{6T}{i\max_i(6\sigma_{i}^{2}+ \Delta_i)\Delta_{i}^{-2}}.\frac{\log (\frac{3}{4} K\log K)}{ 64K\log K} \bigg)\bigg\rbrace\\
%%%%%%%%%%%%%%%
&\leq K\left(\log_2\frac{T}{e}+1\right)\bigg\lbrace\exp\bigg( -\frac{T\log ( 2K\log K)}{ 16 H_2 K\log K}\bigg)\\
& + 2\exp\bigg(- \frac{T\log ( \frac{3}{4} K\log K)}{64H_{2}^{\sigma} K(\log K)} \bigg)\bigg\rbrace\\
\end{align*}
%\end{small}
\end{proof}

%	Next we specialize the result of Theorem \ref{Result:Theorem:1} in Corollary \ref{Result:Corollary:1}.
%
%\subsection{Corollary 2}
%
%
%\begin{corollary}
%\label{Result:Corollary:1}
%For $c_{0}=\sqrt{T}$, $\psi=\frac{T}{\log (K)}$, $\rho_{\mu}=\frac{1}{8}$ and $\rho_v=\frac{2}{3}$, the simple regret of AugUCB is given by,
%\begin{small}
%\begin{align*}
%& SR_{AugUCB} \leq \sum_{i=1}^{K} \Delta_{i}\bigg\lbrace\exp\bigg(-\log ( 2T\frac{\Delta_{i}^{2}}{\sqrt{\log K}})-\dfrac{T}{2 H_{2}}\\
%& + \log \big( \dfrac{4\gamma K\log ( 2T \frac{\Delta_{i}^{2}}{\sqrt{\log K}})}{T\Delta_{i}^{2}}\log_{2}\dfrac{T}{e} \big) \bigg)\\
%& +  \exp\bigg(- \bigg(\dfrac{2\sigma_{i}^{2}+\Delta_{i}+2}{6\sigma_{i}^{2}+\Delta_{i}}\bigg)\log( 3T\frac{\Delta_{i}^{2}}{8\sqrt{\log K}}) -\dfrac{3T}{32 H_{2}}\\
%& + \log\big ( \dfrac{64\gamma K\log ( 3T \frac{\Delta_{i}^{2}}{8\sqrt{\log K}})}{3T\Delta_{i}^{2}}\log_{2}\dfrac{T}{e} \big)  \bigg)\bigg\rbrace
%\end{align*}
%\end{small}
%\end{corollary}
%
%\begin{proof}
%Putting $c_{0}=\sqrt{T}$, $\psi=\frac{T}{\log (K)}$, $\rho_{\mu}=\frac{1}{8}$ and $\rho_v=\frac{2}{3}$ in the result obtained in Theorem \ref{Result:Theorem:1}, we get
%\begin{small}
%\begin{align*}
%& SR_{AugUCB} \leq \sum_{i=1}^{K} \Delta_{i}\bigg\lbrace \exp\bigg(-4\rho\log (\psi T\frac{\Delta_{i}^{4}}{16\rho^{2}})-\dfrac{c_{0}\sqrt{T}}{16\rho H_{2}}\\
%& + \log \big( 16\gamma C_1\log_{2}\dfrac{T}{e} \big) \bigg) + \exp\bigg(- \dfrac{3\rho_v}{2} \bigg(\dfrac{2\sigma_{i}^{2}+\Delta_{i}+2}{6\sigma_{i}^{2}+\Delta_{i}}\bigg)\log(\psi T\frac{\Delta_{i}^{4}}{16\rho_{v}^{2}})\\
%& -\dfrac{c_{0}\sqrt{T}}{16\rho_v H_{2}} + \log\big ( 32\gamma C_2\log_{2}\dfrac{T}{e} \big)  \bigg)\bigg\rbrace\\
%%%%%%%%%%%%%%%%%%
%&\leq \sum_{i=1}^{K} \Delta_{i}\bigg\lbrace\exp\bigg(-\dfrac{1}{2}\log ( T^{2}\frac{4\Delta_{i}^{4}}{\log K})-\dfrac{T}{2 H_{2}}\\
%& + \log \big( \dfrac{2\gamma K\log ( T^{2} \frac{4\Delta_{i}^{4}}{\log K})}{T\Delta_{i}^{2}}\log_{2}\dfrac{T}{e} \big) \bigg)\\
%& + \exp\bigg(-  \bigg(\dfrac{2\sigma_{i}^{2}+\Delta_{i}+2}{6\sigma_{i}^{2}+\Delta_{i}}\bigg)\log( T^{2}\frac{\Delta_{i}^{4}}{16.\frac{4}{9}\log K}) -\dfrac{c_{0}\sqrt{T}}{16.\frac{2}{3} H_{2}}\\
%& + \log\big ( \dfrac{32\gamma\rho_v K\log ( T^{2} \frac{\Delta_{i}^{4}}{16.\frac{2}{9}\log K})}{T\Delta_{i}^{2}}\log_{2}\dfrac{T}{e} \big)  \bigg)\bigg\rbrace\\
%%%%%%%%%%%%%%%%%%
%&\leq \sum_{i=1}^{K} \Delta_{i}\bigg\lbrace\exp\bigg(-\log ( 2T\frac{\Delta_{i}^{2}}{\sqrt{\log K}})-\dfrac{T}{2 H_{2}}\\
%& + \log \big( \dfrac{4\gamma K\log ( 2T \frac{\Delta_{i}^{2}}{\sqrt{\log K}})}{T\Delta_{i}^{2}}\log_{2}\dfrac{T}{e} \big) \bigg)\\
%& +  \exp\bigg(- \bigg(\dfrac{2\sigma_{i}^{2}+\Delta_{i}+2}{6\sigma_{i}^{2}+\Delta_{i}}\bigg)\log( 3T\frac{\Delta_{i}^{2}}{8\sqrt{\log K}}) -\dfrac{3T}{32 H_{2}}\\
%& + \log\big ( \dfrac{64\gamma K\log ( 3T \frac{\Delta_{i}^{2}}{8\sqrt{\log K}})}{3T\Delta_{i}^{2}}\log_{2}\dfrac{T}{e} \big)  \bigg)\bigg\rbrace
%\end{align*} 
%\end{small}
%\end{proof}