In this paper $A$ is the set of all arms and $|A|=K$ denotes the number of arms in the set. Any arm is denoted by $i$. The average estimated payoff for any arm is denoted by $\hat{r}_{i}$ whereas the true mean of the distribution from which the rewards are sampled is denoted by $r_{i}$. The optimal arm is denoted by $*$. The '*' superscript is used to denote anything related to optimal arm.  $\Delta_{i}=|\tau-r_{i}|$ and $\hat{\Delta}_{i}=|\tau-\hat{r}_{i}|$.  $n_{i}$ denotes the number of times the arm $i$ has been pulled. $\psi $ denotes the exploration regulatory factor and $\rho,\rho_v$ as arm elimination parameters. $\hat{V}_{i}=\frac{1}{n_i}\sum_{t=1}^{n_{i}}(x_{i,t}-r_{i})^{2}$ denotes the empirical variance and $x_{i,t}$ is the reward obtained at timestep t for arm $i$. Also  $\sigma_{i}^{2}$ denotes the true variance of the arm $i$. It is assumed that the distribution from which rewards are sampled are identical and independent 1-sub-Gaussian distributions which includes Gaussian distributions with variance less than 1 and distributions supported on an interval of length less than 2. We will also assume that all rewards are bounded in $[0,1]$.

%Also we define $\Delta_{i}=r^{*} - r_{i}$ and $\hat{\Delta}_{i}=\hat{r}^{*} - \hat{r}_{i}$. In all cases $\min_{i\in A}{\Delta_{i}}$ is denoted by $\Delta$.
%and the optimal arm is denoted by $*$. The '*' superscript is used to denote anything related to optimal arm
%\paragraph*{}It is assumed that the distribution from which rewards are sampled are identical and independent sub-Gaussian distributions. Throughout the paper, we assume that the distributions $v_{i}$ are sub-Gaussian that is $\int e^{\lambda(x - r)} v_{i} (dx) ≤ e^{\lambda /2}, \forall \lambda \in \mathbb{R}$. Note that these include Gaussian distributions with variance less than 1 and distributions supported on an interval of length less than 2. All the experiments are also conducted with sub-Gaussians having variance as 1. Together with a Chernoff-Hoeffding bound, the sub-Gaussian assumption implies the following concentration inequality, valid for any
%$u > 0$,
%\newline
%\hspace*{8em}$\mathbb{P}\lbrace \hat{r}_{i} - r^{*} > u\rbrace \leq exp(-\dfrac{su^{2}}{2}) $
%\newline
%where s is the number of pulls of $a_{i}$. T is the horizon over which this entire algorithm runs.  $A^{'}$ at any round $m$ denotes the arms still not eliminated.
%\paragraph*{}The paper is organized as follows. We first present the algorithm in section 6. We then provide the proofs of Phase1 which includes regret-bound calculation and arm deletion conditions in section 7. In section 8, we provide proofs for Phase2 along with early stopping conditions. Section 9 deals with regret bound and then we provide error probability and error bounds in section 10. Experimental results are provided in section 11, and we conclude in section 12.